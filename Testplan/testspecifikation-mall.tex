\documentclass[a4paper,10pt]{article}
\usepackage[margin=1.4in]{geometry}
\usepackage[swedish]{babel}
\usepackage[utf8]{inputenc}
\usepackage{titlesec}
\usepackage{titling}
\usepackage{todonotes}



\setlength{\parskip}{1em}
\setlength{\parindent}{0pt}
\titlespacing{\section}{0pt}{\parskip}{-\parskip}
\titlespacing{\subsection}{0pt}{\parskip}{-\parskip}
\titlespacing{\subsubsection}{0pt}{\parskip}{-\parskip}
\titlespacing{\part}{0pt}{\parskip}{-\parskip}

\def\ftitle{---} % Specifikationen ska namnges enligt "Testspecifikation-komponentX" (komponentX är namnet på komponenten som specifikation är skriven för.)
\def\fversion{1.0}
 %Testspecifikation: Den här specifikation ska innehålla information om hur %testet ska genomföras. Den ska också innehålla lämpliga testscenarion. Utöver %det så ska det finnas en sektion som redogör för designen av komponenten och %om den är uppfylld enligt kravspecifikationen. Vidare så ska det finnas en %sektion som redogör för förväntat resultat, enligt kravspecifikation eller %utifrån beskrivningen av enheten.
\begin{document}
\section{Test ID}
\section{Typ av test}
\emph{[Vilket typ av test är det]}
\section{Genomförande}
\emph{[kort beskriving av sektion]}
\section{Testscenarion}
\emph{[Vad är det som ska testas, kan vara flera saker]}
\subsection{Första saken att testa}
\subsubsection{Design av komponent}
\emph{[Beskriv design av komponent och om den är uppfylld enligt Kravspec]}
\subsubsection{Förväntat resultat}
\emph{[Det förväntade resultat enligt kravspec eller beskrivning av enheten]}
\end{document}
