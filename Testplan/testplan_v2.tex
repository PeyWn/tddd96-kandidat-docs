\documentclass[a4paper,10pt]{article}
\usepackage[margin=1.4in]{geometry}
\usepackage[swedish]{babel}
\usepackage[utf8]{inputenc}
\usepackage[T1]{fontenc}
\usepackage{titlesec}
\usepackage{titling}
\usepackage{todonotes}

\usepackage{graphicx}
\usepackage{fancyhdr}

\pagestyle{fancy}
\rhead{\includegraphics[width=1cm]{../Templates/Aeon}}
\lhead{}

\setlength{\parskip}{1em}
\setlength{\parindent}{0pt}
\titlespacing{\section}{0pt}{\parskip}{-\parskip}
\titlespacing{\subsection}{0pt}{\parskip}{-\parskip}
\titlespacing{\subsubsection}{0pt}{\parskip}{-\parskip}
\titlespacing{\part}{0pt}{\parskip}{-\parskip}

\def\ftitle{Testplan}
\def\fversion{1.0}

\externaldocument[arkd-]{../Kravspec/kravspec}


\begin{document}
\begin{titlepage} % Suppresses displaying the page number on the title page and the subsequent page counts as page 1
	\newcommand{\HRule}{\rule{\linewidth}{0.5mm}} % Defines a new command for horizontal lines, change thickness here

	\center % Centre everything on the page

	%------------------------------------------------
	%	Headings
	%------------------------------------------------

	\textsc{\LARGE Linköpings universitet \\ \vspace{0.2em} Institutionen för datavetenskap }\\[2cm]

    \large\today

    \vspace{1cm}


	%------------------------------------------------
	%	Title
	%------------------------------------------------

	\HRule\\[0.4cm]

	{\huge\bfseries Schemaläggningsstöd för  \vspace{.1em} \\ kirurgi  - \ftitle }\\[0.4cm] % Title of your document

	\HRule\\[1cm]

	%------------------------------------------------
	%	Author(s)
	%------------------------------------------------

	\begin{minipage}{0.7\textwidth}
			\large
            \emph{Version: \fversion}
            \vspace{1em}

            \textbf{\\Adam Andersson, Niclas Byrsten, \\Björn Hvass, Hendrik Lindström, \\Martin Persson, Christoffer Sjöbergsson, \\Tor Utterborn}
            

            \vspace{1em}

            Handledare: Jonas Wallgren

            Examinator: Kristian Sandahl
	\end{minipage}
	~

	%------------------------------------------------
	%	Logo
	%------------------------------------------------

	%\vfill\vfill
	%\includegraphics[width=0.2\textwidth]{../Templates/Aeon}\\[1cm] % Include a department/university logo - this will require the graphicx package

	%----------------------------------------------------------------------------------------

	\vfill % Push the date up 1/4 of the remaining page

\end{titlepage}


\section*{\begin{center}Dokumentationshistorik\end{center}}
    \begin{center}
        \begin{tabular}{|c c c c |}
        \hline
        Datum & händelse & iteration & version\\
        \hline
        2018-02-14 & Dokument skapas & 1 &  0.1\\
        \hline
        2018-02-19 & Konvertering till \textbf{\LaTeX} & 1 &  0.2\\
        \hline
        2018-02-19 & Version 1.0 färdigställd & 1 &  1.0\\
        \hline
        \end{tabular}
    \end{center}
    \clearpage
    \tableofcontents
    \clearpage

    \begin{abstract}

    \end{abstract}
    \clearpage

\section{Introduktion}
Det här dokumentet ska översiktligt beskriva de tester som kommer att genomföras under projektets gång.
\subsection{Mål och syfte}
Testerna är till för att säkerhetsställa att det som levereras uppfyller de krav som finns i Kravspecifikation sektion \ref{arkd-sec:Specifika krav}.
\subsection{Avgränsningar}

\section{Moduler}
    \subsection{Server sidan}
        \subsubsection{Databasen}
        \subsubsection{logiklagret}
        \subsubsection{nätverkskommunikationen}
        
\section{Funktioner}
    \subsubsection{FunktionX}
    %Beskrivning
    %Ref till kravspec
    %Beroende
    %Integritets level

\section{Tester}
De olika nivåer av tester som kommer att genomföras listas nedan.
\subsection{Enhetstest}
\label{sec:Enhetstest}
  \textbf{Syfte:}
  När någon skriver kod ska den personen regelbundet testa koden för att fastställa att den fungerar som den ska.

  \textbf{Ingående data:}
        \begin{enumerate}
            \item Enheten som ska testas.
            \item Testspecifikation-mall. \emph{(Se dokumentet testspecifikation-mall)}
            \item Testprotokoll-mall. \emph{(Se dokumentet testprotokoll-mall)}
            \item Bugglista: Den här är listan ska innehålla alla kända problem som systemet har vid testets start, listan kommer att finnas under \emph{issues} på projektets gitlab.
        \end{enumerate}

    \textbf{Utförande:} Testet ska utföras av berörda parter samt en oberoende person. Personerna som utför testet ska först skapa testspecifikation utifrån beskrivningen i testspecifikationsmallen. När det här är klart så utför personerna testet enligt specifikationen. Ett protokoll ska föras enligt mallen för testprotokoll, se testprotokoll-mall dokumentet. Om några problem uppstod ska dessa föras in under \emph{issues} i projektets gitlab.

    \textbf{Resultat:}
        \begin{enumerate}
            \item \label{itm:Testspecifikation-e} Testspecifikation-testtyp-komponent/modul (komponent/modul är namnet på komponenten eller modulen som specifikation är skriven för).
            \item \label{itm:Testprotokoll-e} Testprotokoll-testtyp-komponent/modul (komponent/modul är namnet på komponenten eller modulen som specifikation är skriven för)..
            \item Bugglista: Lisan ska vid behov uppdateras.
        \end{enumerate}
    \textbf{Testkriterium:} Om resultatet från testprotokollet, se punkt \ref{itm:Testprotokoll-e} under \emph{resultat} i sektion \ref{sec:Enhetstest} ovan, stämmer överens med det förväntade resultatet från testspecifikationen, punkt \ref{itm:Testspecifikation-e} under \emph{resultat} i sektion \ref{sec:Enhetstest} ovan, så anses komponenten klarat testet.

\subsection{Integrationstest}
\label{sec:Integrationstest}
\textbf{Syfte:}
När två eller flera enheter som ska sammanföras, där var och en har passerat enhetstesterna, så ska de integrationstestas.

\textbf{Ingående data:}
    \begin{enumerate}
        \item Modulen som ska integrationstestas.
        \item Testprotokoll för alla enheter som finns i modulen som ska integrationstestas.
        \item Testspecifikation-mall. \emph{(Se dokumentet testspecifikation-mall)}
        \item Testprotokoll-mall. \emph{(Se dokumentet testprotokoll-mall)}
        \item Bugglista: Den här är listan ska innehålla alla kända problem som systemet har vid testets start, listan kommer att finnas under \emph{issues} på projektets gitlab.
    \end{enumerate}

\textbf{Utförande:} Testet ska utföras av berörda parter, det bör max finnas en representant per komponent som ska integrationstestas. Personerna som utför testet ska först skapa testspecifikation utifrån beskrivningen i testspecifikationsmallen. När det här är klart så utför personerna testet enligt specifikationen. Ett protokoll ska föras enligt mallen för testprotokoll, se testprotokoll-mall dokumentet. Om några problem uppstod ska dessa föras in under \emph{issues} i projektets gitlab.

\textbf{Resultat:}
    \begin{enumerate}
        \item \label{itm:Testspecifikation-i} Testspecifikation-testtyp-komponent/modul (komponent/modul är namnet på komponenten eller modulen som specifikation är skriven för).
        \item \label{itm:Testprotokoll-i} Testprotokoll-testtyp-komponent/modul (komponent/modul är namnet på komponenten eller modulen som specifikation är skriven för)..
        \item Bugglista: Lisan ska vid behov uppdateras.
    \end{enumerate}

\textbf{Testkriterium:} Om resultatet från testprotokollet, se punkt \ref{itm:Testprotokoll-i} under \emph{resultat} i sektion \ref{sec:Integrationstest} ovan, stämmer överens med det förväntade resultatet från testspecifikationen, punkt \ref{itm:Testspecifikation-i} under \emph{resultat} i sektion \ref{sec:Integrationstest} ovan, så anses komponenten klarat testet.

\subsection{Systemtest}
\label{sec:Systemtest}
\textbf{Syfte:}

\textbf{Ingående data:}
   \begin{enumerate}
       \item Systemet som ska testas.
       \item Testprotokoll för alla systemet moduler.
       \item Testspecifikation-mall. \emph{(Se dokumentet testspecifikation-mall)}
       \item Testprotokoll-mall. \emph{(Se dokumentet testprotokoll-mall)}
       \item Bugglista: Den här är listan ska innehålla alla kända problem som systemet har vid testets start, listan kommer att finnas under \emph{issues} på projektets gitlab.
   \end{enumerate}

\textbf{Utförande:} Testet ska utföras av hela projektgruppen. Gruppen utser två till tre personer som ska skapa en testspecifikation utifrån beskrivningen i testspecifikationsmallen. När det här är klart så utför personerna testet enligt specifikationen. Ett protokoll ska föras enligt mallen för testprotokoll, se testprotokoll-mall dokumentet. Om några problem uppstod ska dessa föras in under \emph{issues} i projektets gitlab.

\textbf{Resultat:}
    \begin{enumerate}
        \item \label{itm:Testspecifikation-s} Testspecifikation-testtyp-komponent/modul (komponent/modul är namnet på komponenten eller modulen som specifikation är skriven för).
        \item \label{itm:Testprotokoll-s} Testprotokoll-testtyp-komponent/modul (komponent/modul är namnet på komponenten eller modulen som specifikation är skriven för)..
        \item Bugglista: Lisan ska vid behov uppdateras.
    \end{enumerate}

\textbf{Testkriterium:} Om resultatet från testprotokollet, se punkt \ref{itm:Testprotokoll-s} under \emph{resultat} i sektion \ref{sec:Systemtest} ovan, stämmer överens med det förväntade resultatet från testspecifikationen, punkt \ref{itm:Testspecifikation-s} under \emph{resultat} i sektion \ref{sec:Systemtest} ovan, så anses komponenten klarat testet.

\subsection{Acceptanstest}
\textbf{Syfte:}
Två stycken acceptanstest kommer att genomföras. För det första ska ett test utföras på systemets grafiska gränssnitt.  För det andra så ska ett test utföars på hela systemet efter att systemtestet har blivit godkänd.

\textbf{Ingående data:}
        \begin{enumerate}
            \item Testprotokoll-systemet.
            \item Modulen(a) som ska testas.
            \item Testspecifikation-mall. \emph{(Se dokumentet testspecifikation-mall)}
            \item Testprotokoll-mall. \emph{(Se dokumentet testprotokoll-mall)}
            \item Bugglista: Den här är listan ska innehålla alla kända problem som systemet har vid testets start, listan kommer att finnas under \emph{issues} på projektets gitlab.
        \end{enumerate}

\textbf{Utförande:} Testet ska utföras av hela projektgruppen. Gruppen utser två till tre personer som ska skapa en testspecifikation utifrån beskrivningen i testspecifikationsmallen. När det här är klart så utför personerna testet enligt specifikationen. Ett protokoll ska föras enligt mallen för testprotokoll, se testprotokoll-mall dokumentet. Om några problem uppstod ska dessa föras in under \emph{issues} i projektets gitlab.

\textbf{Resultat:}
    \begin{enumerate}
        \item \label{itm:Testspecifikation-a} Testspecifikation-testtyp-komponent/modul (komponent/modul är namnet på komponenten eller modulen som specifikation är skriven för).
        \item \label{itm:Testprotokoll-a} Testprotokoll-testtyp-komponent/modul (komponent/modul är namnet på komponenten eller modulen som specifikation är skriven för)..
        \item Bugglista: Lisan ska vid behov uppdateras.
    \end{enumerate}

\textbf{Testkriterium:} Om resultatet från testprotokollet, se punkt \ref{itm:Testprotokoll-s} under \emph{resultat} i sektion \ref{sec:Systemtest} ovan, stämmer överens med det förväntade resultatet från testspecifikationen, punkt \ref{itm:Testspecifikation-s} under \emph{resultat} i sektion \ref{sec:Systemtest} ovan, så anses komponenten klarat testet.
\end{document}
