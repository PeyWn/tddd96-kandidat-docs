\documentclass[a4paper,10pt]{article}
\usepackage[margin=1.4in]{geometry}
\usepackage[swedish]{babel}
\usepackage[utf8]{inputenc}
\usepackage[T1]{fontenc}
\usepackage{titlesec}
\usepackage{titling}
\usepackage{todonotes}

\usepackage{graphicx}
\usepackage{fancyhdr}

\pagestyle{fancy}
\rhead{\includegraphics[width=1cm]{../Templates/Aeon}}
\lhead{}

\setlength{\parskip}{1em}
\setlength{\parindent}{0pt}
\titlespacing{\section}{0pt}{\parskip}{-\parskip}
\titlespacing{\subsection}{0pt}{\parskip}{-\parskip}
\titlespacing{\subsubsection}{0pt}{\parskip}{-\parskip}
\titlespacing{\part}{0pt}{\parskip}{-\parskip}

\def\ftitle{Testplan}
\def\fversion{1.0}

\externaldocument[arkd-]{../Kravspec/kravspec}


\begin{document}
\begin{titlepage} % Suppresses displaying the page number on the title page and the subsequent page counts as page 1
	\newcommand{\HRule}{\rule{\linewidth}{0.5mm}} % Defines a new command for horizontal lines, change thickness here

	\center % Centre everything on the page

	%------------------------------------------------
	%	Headings
	%------------------------------------------------

	\textsc{\LARGE Linköpings universitet \\ \vspace{0.2em} Institutionen för datavetenskap }\\[2cm]

    \large\today

    \vspace{1cm}


	%------------------------------------------------
	%	Title
	%------------------------------------------------

	\HRule\\[0.4cm]

	{\huge\bfseries Schemaläggningsstöd för  \vspace{.1em} \\ kirurgi  - \ftitle }\\[0.4cm] % Title of your document

	\HRule\\[1cm]

	%------------------------------------------------
	%	Author(s)
	%------------------------------------------------

	\begin{minipage}{0.7\textwidth}
			\large
            \emph{Version: \fversion}
            \vspace{1em}

            \textbf{\\Adam Andersson, Niclas Byrsten, \\Björn Hvass, Hendrik Lindström, \\Martin Persson, Christoffer Sjöbergsson, \\Tor Utterborn}
            

            \vspace{1em}

            Handledare: Jonas Wallgren

            Examinator: Kristian Sandahl
	\end{minipage}
	~

	%------------------------------------------------
	%	Logo
	%------------------------------------------------

	%\vfill\vfill
	%\includegraphics[width=0.2\textwidth]{../Templates/Aeon}\\[1cm] % Include a department/university logo - this will require the graphicx package

	%----------------------------------------------------------------------------------------

	\vfill % Push the date up 1/4 of the remaining page

\end{titlepage}


\section*{\begin{center}Dokumentationshistorik\end{center}}
    \begin{center}
        \begin{tabular}{|c c c c |}
        \hline
        Datum & händelse & iteration & version\\
        \hline
        2018-02-14 & Dokument skapas & 1 &  0.1\\
        \hline
        2018-02-19 & Konvertering till \textbf{\LaTeX} & 1 &  0.2\\
        \hline
        2018-02-19 & Version 1.0 färdigställd & 1 &  1.0\\
        \hline
        \end{tabular}
    \end{center}
    \clearpage
    \tableofcontents
    \clearpage

    \begin{abstract}

    \end{abstract}
    \clearpage

\section{Introduktion}
Det här dokumentet ska översiktligt beskriva de tester som kommer att genomföras under projektets gång.
\subsection{Mål och syfte}
Testerna är till för att säkerhetsställa att det som levereras uppfyller de krav som finns i Kravspecifikation sektion \ref{arkd-sec:Specifika krav}.
\subsection{Avgränsningar}

\section{Moduler}

\section{Funktioner}

    \subsubsection{Boka Operation}
    %Beskrivning
    %Ref till kravspec
    %Beroende
    %Integritets level
    
I denna funktion skall man kunna välja tid, se vilka lediga bokningsbara tider som existerar samt bli informerad om när kompetens, material och utrustning saknas
Bokningen skall slutföras med en bekräftelseruta

Operationstider
Huvudåtgärder (KVÅ-koder) med operationstider. Förberedelsetid, operationstid och avvecklingstid. 

Det går att mata in:

\begin{itemize}
	\item Personnummer
	\item Efternamn, Förnamn
	\item Diagnoskod
	\item Åtgärdskod enligt KVÅ-standard
	\item Angiven KVÅ-kod som är huvudåtgärd
	\item Operation av parigt organ (lateralitet)
	\item Datum och tid för beslut om kirurgisk åtgärd
	\item Namn på beslutsfattare inkl. titel
	\item Beslutande klinik
	\item Medicinsk brådskandegrad
	\item Behandlingsnummer
	\item Akut eller elektiv operation	
	
\end{itemize}


Funktion för att Boka sal: \\
Ingående data 
Namn på lokal
Typ av lokal
Tillgänglig salstid
Möjliga operationer att genomföra i lokalen
Ingreppsbeskrivning

Funktion för att Avboka operation: \\
-	Välja tid
-	Bekräftelseruta

Funktion för Tidslinje för operation \\
-	Visa beslutsdata

Funktion för översikt av beslutade operationer \\
-	Visa beslutsdata
Funktion för schemaöversikt för planerade operationer
-	Filterfunktion för schemaöversikt 

Funktion för att Reservera material, lokal, kompetens och utrustning \\
Databashanterings funktion

Funktion för reservation av: \\
Ingående data 
	Material
	Artikelnamn
	Artikelnummer
	Typ av material
	Reserverade tider för flergångsmaterial (tidsåtgång: operation + sterilisering)
Funktion för reservation Utrustning: \\
Ingående data
	Namn
	MTÖ-nummer 
	Typ av utrustning
	Reserverade tider.
Kompetenser
Titel
Efternamn, förnamn
Kliniktillhörighet
Klinisk specialitet
Kompetens (Kvå-kod, huvudåtärd)
Schema


\section{Tester}
De olika nivåer av tester som kommer att genomföras listas nedan.
\subsection{Enhetstest}
  \textbf{Syfte:}
  När någon skriver kod ska den personen regelbundet testa koden för att fastställa att den fungerar som den ska.

  \textbf{Ingående data:}
        \begin{enumerate}
            \item Enheten som ska testas.
            \item \label{itm:Testspecifikation} Testspecifikation: Den här specifikation ska innehålla information om hur testet ska genomföras. Den ska också, om det behövs för enheten, innehålla lämpliga testscenarion. Utöver det så ska det finnas en sektion som redogör för designen av komponenten och om den är uppfylld enligt kravspecifikationen. Vidare så ska det finnas en sektion som redogör för förväntat resultat, enligt kravspecifikation eller utifrån beskrivningen av enheten.
            \item Bugglista: Den här är listan ska innehålla alla kända problem som systemet har vid testets start, listan kommer att finnas under \emph{issues} på projektets gitlab.
        \end{enumerate}
    \textbf{Utförande:} Personen eller personerna som utför testet ska först skapa testspecifikation utifrån beskrivningen ovan, punkt \ref{itm:Testspecifikation} under \emph{Ingående data}. När det här är klart så utför personerna testet. Ett protokoll ska föras enligt mallen \todo{ref till protokoll-mall}. Om några problem uppstod ska dessa föras in under \emph{issues} i projektets gitlab.

    \textbf{Resultat:}
        \begin{enumerate}
            \item \label{tim:Testprotokoll} Testprotokoll.
            \item Bugglista: Lisan ska vid behov uppdateras.
        \end{enumerate}
    \textbf{Kriterium för passering:} Om resultatet från testprotokollet, se punkt \ref{itm:Testprotokoll} under resultat ovan, stämmer överens med det förväntade resultatet från testspecifikationen.

\subsection{Integrationstest}
När två eller flera enheter som ska sammanföras, där var och en har passerat enhetstesterna, så ska de integrationstestas.
   \textbf{Syfte:}
    \\ \textbf{Ingående data:}
    \\ \textbf{Utförande:}
    \\ \textbf{Resultat:}
    \\ \textbf{Kriterium för passering:}
\subsection{Systemtest}
   \textbf{Syfte:}
    \\ \textbf{Ingående data:}
    \\ \textbf{Utförande:}
    \\ \textbf{Resultat:}
    \\ \textbf{Kriterium för passering:}
\subsection{Acceptanstest}
Två stycken acceptanstest kommer att genomföras, dels en pappersprototyp dels ett på systemet i sig när systemtestet har blivit godkänd.
    \\ \textbf{Syfte:}
    \\ \textbf{Ingående data:}
    \\ \textbf{Utförande:}
    \\ \textbf{Resultat:}
    \\ \textbf{Kriterium för passering:}




\end{document}
