\documentclass[a4paper,10pt]{article}
\usepackage[margin=1.4in]{geometry}
\usepackage[swedish]{babel}
\usepackage[utf8]{inputenc}
\usepackage[T1]{fontenc}
\usepackage{titlesec}
\usepackage{titling}
\usepackage{todonotes}

\usepackage{graphicx}
\usepackage{fancyhdr}

\pagestyle{fancy}
\rhead{\includegraphics[width=1cm]{../Templates/Aeon}}
\lhead{}

\setlength{\parskip}{1em}
\setlength{\parindent}{0pt}
\titlespacing{\section}{0pt}{\parskip}{-\parskip}
\titlespacing{\subsection}{0pt}{\parskip}{-\parskip}
\titlespacing{\subsubsection}{0pt}{\parskip}{-\parskip}
\titlespacing{\part}{0pt}{\parskip}{-\parskip}

\usepackage{multicol}
\usepackage[backend=bibtex,sorting=none]{biblatex}

\def\ftitle{Testplan}
\def\fversion{1.0}

\addbibresource{bib-testplan.bib}

\externaldocument[krav-]{../Kravspec/kravspec}
\externaldocument[arki-]{../Arkitekturdokument/arkitekturdokument}




\begin{document}
\begin{titlepage} % Suppresses displaying the page number on the title page and the subsequent page counts as page 1
	\newcommand{\HRule}{\rule{\linewidth}{0.5mm}} % Defines a new command for horizontal lines, change thickness here

	\center % Centre everything on the page

	%------------------------------------------------
	%	Headings
	%------------------------------------------------

	\textsc{\LARGE Linköpings universitet \\ \vspace{0.2em} Institutionen för datavetenskap }\\[2cm]

    \large\today

    \vspace{1cm}


	%------------------------------------------------
	%	Title
	%------------------------------------------------

	\HRule\\[0.4cm]

	{\huge\bfseries Schemaläggningsstöd för  \vspace{.1em} \\ kirurgi  - \ftitle }\\[0.4cm] % Title of your document

	\HRule\\[1cm]

	%------------------------------------------------
	%	Author(s)
	%------------------------------------------------

	\begin{minipage}{0.7\textwidth}
			\large
            \emph{Version: \fversion}
            \vspace{1em}

            \textbf{\\Adam Andersson, Niclas Byrsten, \\Björn Hvass, Hendrik Lindström, \\Martin Persson, Christoffer Sjöbergsson, \\Tor Utterborn}
            

            \vspace{1em}

            Handledare: Jonas Wallgren

            Examinator: Kristian Sandahl
	\end{minipage}
	~

	%------------------------------------------------
	%	Logo
	%------------------------------------------------

	%\vfill\vfill
	%\includegraphics[width=0.2\textwidth]{../Templates/Aeon}\\[1cm] % Include a department/university logo - this will require the graphicx package

	%----------------------------------------------------------------------------------------

	\vfill % Push the date up 1/4 of the remaining page

\end{titlepage}


\section*{\begin{center}Dokumentationshistorik\end{center}}
    \begin{center}
        \begin{tabular}{|c c c c |}
        \hline
        Datum & händelse & iteration & version\\
        \hline
        2018-02-14 & Dokument skapas & 1 &  0.1\\
        \hline
        2018-02-19 & Konvertering till \textbf{\LaTeX} & 1 &  0.2\\
        \hline
        2018-03-05 & Version 1.0 färdigställd & 1 &  1.0\\
        \hline
        \end{tabular}
    \end{center}
    \clearpage
    \tableofcontents
    \clearpage

\section{Introduktion}
Det här dokumentet ska översiktligt beskriva de tester som kommer att genomföras under projektets gång.


\subsection{Mål och syfte}
Testerna är till för att säkerställa att det som levereras uppfyller de krav som
finns i kravspecifikation sektion \ref{krav-sec:Specifika krav}.

%\subsection{Avgränsningar}

\section{Moduler}
Systemet är uppdelat i två delar, webbservern och webbklienten. Dessa delarna är
i sin tur uppdelade i ett antal olika moduler. En övergripande systemskiss finns
i arkitekturdokumentet under rubrik \ref{arki-sec:Systemskiss}. Mer information
angående webbklienten och dess moduler se sektion \ref{arki-sec:webbklient}
medans information angående webbservern finns under sektion \ref{arki-sec:webbserver}.

\section{Funktioner}
Denna sektion beskriver de funktioner som ska genomgå testning. Varje funktions
användningsområde presenteras tillsammans med den indata som funktionen ska
behandla. Varje funktion har även blivit tilldelad en integritets nivå som
beskriver hur kritisk funktionen är för systemet.
Se kravspecifikationen \ref{krav-subsec:Funktionellakrav} för mer ingående
information om kraven som berör funktionen.

    \subsection{Boka Operation}
    %Beskrivning
    %Ref till kravspec
    %Beroende
    %Integritets level

I denna funktion skall man kunna välja tid, se vilka lediga bokningsbara tider som existerar samt bli informerad om när kompetens, material och utrustning saknas. Förberedelser och efterarbete beskrivs med hjälp av KVÅ-kodning.

Beslutsdata:
\begin{multicols}{3}
\begin{itemize}
	\item Personnummer
	\item Efternamn, Förnamn
	\item Diagnoskod
	\item Åtgärdskod enligt KVÅ-standard
	\item vilken KVÅ-kod som är huvudåtgärd
	\item Operation av parigt organ (lateralitet)
	\item Datum och tid för beslut om kirurgisk åtgärd
	\item Namn på beslutsfattare inkl. titel
	\item Beslutande klinik
	\item Medicinsk brådskandegrad
	\item Behandlingsnummer
	\item Akut eller elektiv operation
\end{itemize}
\end{multicols}

Operationstider (KVÅ-koder)
\begin{multicols}{3}
\begin{itemize}
	\item Förberedelsetid
	\item Operationstid
	\item Avvecklingstid.
\end{itemize}
\end{multicols}

\subsubsection{Boka sal}
Det ska gå att välja namn på den lokal man vill boka och välja vilken typ av lok
al det är. Man ska även kunna se tillgängliga tider för salen. Vid
bokning skall man få information över vilka operationer som går att genomföra i
lokalen samt en inggreppbeskrivning.

Ingående data
\begin{multicols}{2}
\begin{itemize}
	\item Namn på lokal
	\item Typ av lokal
	\item Tillgängliga tider för sal
	\item Möjliga operationer att genomföra i lokalen
	\item Ingreppsbeskrivning
\end{itemize}
\end{multicols}

\subsubsection{Avboka sal}

Det ska vara möjligt att avboka en tid som man sedan tidigare bokat.

-	Välja tid

\subsection{Inloggning}
I denna funktion ska följande uppfyllas:
\begin{enumerate}
	\item Krävas användarnamn och lösenord för att komma in i systemet.
  \item En användare ska ha ett användarnamn och ett lösenord.
  \item Det ska finnas en databas som kan hantera all ingående data.
\end{enumerate}

\subsection{Tidslinje för operation}

Ska presentera den beslutsdata som finns att tillgå.

-	Visa beslutsdata

\subsection{Översikt av beslutade operationer}

Ska presentera beslutade operationer.

-	Visa beslutsdata
\subsection{Schemaöversikt för planerade operationer}

En överblick för de operationer som ej blivit beslutade men som är planerade, skall gå att filtrera efter önskemål angivna i kravspec.

-	Filterfunktion för schemaöversikt

\clearpage
\subsection{Reservera material, lokal, kompetens och utrustning}

Databasfunktion som kan reservera de olika resurser som behövs för operation samt post- eller pre-operativa processer.
\subsubsection{Material}
Ingående data
\begin{multicols}{2}
\begin{itemize}
	\item Material
	\item Artikelnamn
	\item Typ av material
	\item Reserverade tider för material som måste steriliseras \\Tidsåtgång: \\ Operationstid + steriliseringstid
\end{itemize}
\end{multicols}

\subsubsection{Utrustning}
Ingående data
\begin{multicols}{2}
\begin{itemize}
	\item Namn
	\item MTÖ-nummer
	\item Typ av utrustning
	\item Reserverade tider.
\end{itemize}
\end{multicols}

\subsubsection{Kompetenser}
Ingående data
\begin{multicols}{2}
\begin{itemize}
	\item Titel
	\item Efternamn, förnamn
	\item Kliniktillhörighet
	\item Klinisk specialitet
	\item Kompetens (Kvå-kod, huvudåtgärd)
	\item Schema
\end{itemize}
\end{multicols}

\section{Tester}
Den här sektionen redogör för de olika testningsprocedurerna som ska utföras för
de olika testnivåer.

\subsection{Enhetstest}
\label{sec:Enhetstest}
  \textbf{Syfte:}
  Efter det att en funktion har implementerats ska koden genomgå ett enhetstest.
  Testet ska utföras för att säkerställa att funktionen fungerar enligt
  specifikationen \cite{kravspec}. Vidare används testet för att fastställa att
  funktionen är följer designkraven.

  \textbf{Ingående data:}
    \begin{enumerate}
      \item Enheten som ska testas.
      \item Testspecifikation-mall.
      \emph{(Se dokumentet: testspecifikation-mall)}
      \item Testprotokoll-mall. \emph{(Se dokumentet: testprotokoll-mall)}
      \item Bugglista: Den här är listan ska innehålla alla kända problem som
      systemet har vid testets start, listan kommer att finnas under
      \emph{issues} på projektets gitlab.
    \end{enumerate}

  \textbf{Utförande:} Testet ska utföras av berörda parter samt en oberoende
  person. Personerna som utför testet ska först skapa testspecifikation utifrån beskrivningen i testspecifikationsmallen. När det här är klart så utför
  personerna testet enligt specifikationen \cite{kravspec}. Ett protokoll ska
  föras enligt mallen för testprotokoll, se testprotokoll-mall dokumentet. Om
  några problem uppstod som inte kunde lösas på plats ska dessa föras in under
  \emph{issues} i projektets gitlab.

  \textbf{Resultat:}
    \begin{enumerate}
      \item \label{itm:Testspecifikation-e}
      Testspecifikation-testtyp-komponent/modul (komponent/modul är namnet på
      komponenten eller modulen som specifikation är skriven för).
      \item \label{itm:Testprotokoll-e} Testprotokoll-testtyp-komponent/modul
      (komponent/modul är namnet på komponenten eller modulen som specifikation
      är skriven för)..
      \item Bugglista: Listan ska vid behov uppdateras.
    \end{enumerate}

  \textbf{Testkriterium:} Om resultatet från testprotokollet, se punkt
   \ref{itm:Testprotokoll-e} under \emph{resultat} i sektion \ref{sec:Enhetstest}
  ovan, stämmer överens med det förväntade resultatet från testspecifikationen,
  punkt \ref{itm:Testspecifikation-e} under \emph{resultat} i sektion
  \ref{sec:Enhetstest} ovan, så anses komponenten klarat testet.

\subsection{Integrationstest}
\label{sec:Integrationstest}
\textbf{Syfte:}
Ett integrationstest ska utföras för att säkerställa att en modul och dess
komponenter fungerar enligt specifikationen \cite{kravspec}. Testet säkerställer
också att modulen följer designkraven.

\textbf{Ingående data:}
  \begin{enumerate}
    \item Modulen som ska integrationstestas.
    \item Testprotokoll för alla enheter som finns i modulen som ska
    integrationstestas.
    \item Testspecifikation-mall. \emph{(Se dokumentet testspecifikation-mall
    \cite{testspec-mall})}
    \item Testprotokoll-mall. \emph{(Se dokumentet testprotokoll-mall
    \cite{testprot-mall})}
    \item Bugglista: Den här är listan ska innehålla alla kända problem som
    systemet har vid testets start, listan kommer att finnas under \emph{issues}
    på projektets gitlab.
  \end{enumerate}

\textbf{Utförande:} Testet ska utföras av berörda parter, det bör max finnas en
representant per komponent som ska integrationstestas. Personerna som utför
testet ska börja med att skapa testspecifikation utifrån den beskrivning som
finns i testspecifikationsmallen. När det är klart så utför personerna
testet enligt specifikationen \cite{kravspec}. Ett protokoll ska föras enligt
mallen för test
protokoll, se testprotokoll-mall dokumentet. Om några problem uppstod som inte
kunde lösas på plats ska dessa föras in under \emph{issues} i projektets gitlab.

\textbf{Resultat:}
    \begin{enumerate}
        \item \label{itm:Testspecifikation-i}
        Testspecifikation-testtyp-komponent/modul (komponent/modul är namnet på
        komponenten eller modulen som specifikation är skriven för).
        \item \label{itm:Testprotokoll-i} Testprotokoll-testtyp-komponent/modul
        (komponent/modul är namnet på komponenten eller modulen som
        specifikation är skriven för)..
        \item Bugglista: Listan ska vid behov uppdateras.
    \end{enumerate}

\textbf{Testkriterium:} Om resultatet från testprotokollet, se punkt \ref{itm:Testprotokoll-i} under \emph{resultat} i sektion
\ref{sec:Integrationstest} ovan, stämmer överens med det förväntade resultatet
från testspecifikationen, punkt \ref{itm:Testspecifikation-i} under
\emph{resultat} i sektion \ref{sec:Integrationstest} ovan, så anses komponenten
klarat testet.

\subsection{Systemtest}
\label{sec:Systemtest}
\textbf{Syfte:}
Testet ska utföras för att säkerställa att systemet behandlar och utan problem
kan utföra alla de funktioner som krävs av systemet enligt specifikationen
\cite{kravspec}.

\textbf{Ingående data:}
   \begin{enumerate}
       \item Systemet som ska testas
       \item Testprotokoll för alla systemets moduler
       \item Testspecifikation-mall.
       \emph{(Se dokumentet testspecifikation-mall \cite{testspec-mall})}
       \item Testprotokoll-mall.
       \emph{(Se dokumentet testprotokoll-mall \cite{testprot-mall})}
       \item Bugglista: Den här är listan ska innehålla alla kända problem som
       systemet har vid testets start, listan kommer att finnas under
       \emph{issues} på projektets gitlab.
   \end{enumerate}

\textbf{Utförande:} Testet ska utföras av hela projektgruppen. Gruppen utser två
till tre personer som skapar en testspecifikation utifrån beskrivningen i
testspecifikationsmallen. När det är klart så utför personerna testet enligt
specifikationen \cite{kravspec}. Ett protokoll ska föras enligt mallen för
testprotokoll, se testprotokoll-mall-dokumentet. Om några problem uppstod som
inte kunde lösas på plats ska dessa föras in under \emph{issues} i projektets
gitlab.

\textbf{Resultat:}
    \begin{enumerate}
        \item \label{itm:Testspecifikation-s}
        Testspecifikation-testtyp-komponent/modul (komponent/modul är namnet på
        komponenten eller modulen som specifikation är skriven för)
        \item \label{itm:Testprotokoll-s} Testprotokoll-testtyp-komponent/modul
        (komponent/modul är namnet på komponenten eller modulen som
        specifikation är skriven för)
        \item Bugglista: Listan ska vid behov uppdateras
    \end{enumerate}

\textbf{Testkriterium:} Om resultatet från testprotokollet, se punkt
\ref{itm:Testprotokoll-s} under \emph{resultat} i sektion \ref{sec:Systemtest}
ovan, stämmer överens med det förväntade resultatet från testspecifikationen,
punkt \ref{itm:Testspecifikation-s} under \emph{resultat} i sektion
\ref{sec:Systemtest} ovan, så anses komponenten klarat testet.

\subsection{Acceptanstest}
\textbf{Syfte:}
Två stycken acceptanstest ska att genomföras. Ett test ska utföras på systemets
grafiska gränssnitt.  Det andra testet ska utföars på hela systemet efter att
systemtestet blivit godkänt.

\textbf{Ingående data:}
        \begin{enumerate}
            \item Testprotokoll-system
            \item Modulen(a) som ska testas
            \item Testspecifikation-mall
            \emph{(Se dokumentet testspecifikation-mall \cite{testspec-mall})}
            \item Testprotokoll-mall \emph{(Se
            dokumentet testprotokoll-mall \cite{testprot-mall})}
            \item Bugglista: Den här är listan ska innehålla alla kända problem
            som systemet har vid testets start, listan kommer att finnas under
            \emph{issues} på projektets gitlab
        \end{enumerate}

\textbf{Utförande:} Testet ska utföras av hela projektgruppen. Gruppen utser
två till tre personer som skapar en testspecifikation utifrån beskrivningen i
testspecifikationsmallen. När det är klart så utför personerna testet enligt
specifikationen \cite{kravspec}. Ett protokoll ska föras enligt mallen för
testprotokoll, se testprotokoll-mall dokumentet. Om några problem uppstod som
inte kunde lösas på plats ska dessa föras in under \emph{issues} i projektets
gitlab.

\textbf{Resultat:}
    \begin{enumerate}
        \item \label{itm:Testspecifikation-a}
        Testspecifikation-testtyp-komponent/modul (komponent/modul är namnet på
        komponenten eller modulen som specifikation är skriven för)
        \item \label{itm:Testprotokoll-a} Testprotokoll-testtyp-komponent/modul
        (komponent/modul är namnet på komponenten eller modulen som
        specifikation är skriven för)
        \item Bugglista: Lisan ska vid behov uppdateras
    \end{enumerate}

\textbf{Testkriterium:} Testet anses vara godkänt om och endast om alla krav i specifikationen är behandlade och uppfyllda.
\clearpage
\printbibliography
\end{document}
