\documentclass[a4paper,10pt, twoside]{article}
\usepackage[margin=1.4in]{geometry}
\usepackage[swedish]{babel}
\usepackage[utf8]{inputenc}
\usepackage{titlesec}
\usepackage{titling}
\usepackage{todonotes}



\setlength{\parskip}{1em}
\setlength{\parindent}{0pt}
\titlespacing{\section}{0pt}{\parskip}{-\parskip}
\titlespacing{\subsection}{0pt}{\parskip}{-\parskip}
\titlespacing{\subsubsection}{0pt}{\parskip}{-\parskip}
\titlespacing{\part}{0pt}{\parskip}{-\parskip}


\title{DesignTest 1}
\author{Testansvarig: Björn Hvass\\Observatör: Christoffer Sjöbergsson}
\date{\today}

\begin{document}
\maketitle\clearpage

\section{Resurser}
    Personal från sjukhuset:
    \begin{itemize}
        \item Projektledare
        \item Schemaläggare (2st)
        \item Designers UX (2st)
        \item Tekniska kompetenser (3st)
    \end{itemize}

\section{Test log}
\emph{
    \begin{itemize}
    \item Björn går igenom alla desigelement som ska utvärderas.
    \item Gruppen frågande ut.
    \item En av schemaläggarna funderar över strukturen och skulle föredra att planeringen sker utifrån en sal istället för en patient
    \item På grund av att prototypen var ofullständig så uppstod förvirring i hur vi hade tänkt, efter en längre förklaring och demonstrerande av olika idéer kom gruppen fram till att det som designats följer kraven som satts upp och att detta är rätt. Däremot finns det ytterligare krav som inte identifierats tidigare.
    \item Kraven som dök upp var i relation till att planeringen bör gå att utföra baklänges. Med detta menas att det finns önskemål om att istället för att filtrera schemat utifrån ett valt beslut så vill de kunna filtrera listan med beslut utifrån en vald sal och dag.
    \end{itemize}

}


\section{Resultat}
Loggen ovan är resultatet från testet.

\section{Buggar}
Inga buggar hittades. Däremot identifierades fler krav, dessa kommer att utvärderas och om det bedöms lämpligt kommer de implementeras i kravspecifikationen.
\end{document}
