\documentclass[a4paper,10pt, twoside]{article}
\usepackage[margin=1.4in]{geometry}
\usepackage[swedish]{babel}
\usepackage[utf8]{inputenc}
\usepackage{titlesec}
\usepackage{titling}
\usepackage{todonotes}



\setlength{\parskip}{1em}
\setlength{\parindent}{0pt}
\titlespacing{\section}{0pt}{\parskip}{-\parskip}
\titlespacing{\subsection}{0pt}{\parskip}{-\parskip}
\titlespacing{\subsubsection}{0pt}{\parskip}{-\parskip}
\titlespacing{\part}{0pt}{\parskip}{-\parskip}


\title{Demo2}
\author{Testansvarig: Kurt Karllson\\Observatör: Anna Alm.}
\date{\today}

\begin{document}

\section{Resurser}
\begin{itemize}
  \item Gunnar IT
  \item Jeremias
  \item Daniel Kirurg
  \item Två Planerare
  \item En observatör
\end{itemize}

\section{Test log}

\begin{itemize}
    \item Christoffer börjar med att start upp webbsidan.
    \item Christoffer visar beslut vy
    \item fråga: boka för sent (i det röda), svar ja
    \item fråga: hur man ska välja dag, svar Planerare båda fungerar för mig. Planerare missar ingen information så möjligtvis ett onödigt tryck med knapp. Planerare skulle ha en vint med mer information.
    \item fråga det ska synas om det inte går och boka en dag! OBS! svar: Christoffer skulle kunna gråa ut den dagen.
    \item tex på röda dager så är det dumt och boka samma med röd dagar.
    \item Planerare kontinuerligt pussel att få ihop för att fylla dagar.
    \item Planerare veckovy är intressant, då den kan visa mycket information om vad en vecka innehåller. Ser ut som Outlook ungefär idag. Kan vara en låsning som finns i mitt eget huvud men för mig i mitt vardagliga arbete så har det varit av mycket hjälp.
    \item Planerare man gör en bedömning att ett beslut kan skjutas för att ge plats till ett beslut som har mer prio.
    \item fråga men ni gör mycket ommöbleringar OBS!, svar ja, med det är det som vi känner skulle vara så bra men en dator som kan ge tillgång till så många olika tillgångar faktorer.
    \item Planerare jag vill kunna se allt i eventbokningen kalendern som jag ser här *pekar på sin antecknings pärm*
    \item visar material listan
    \item Planerare det hade varit bra och se hur det gick så man får referens punkter
    \item Christoffer förklarar tanken bakom tidslinjen
    \item diskussion om arbetsflödet välj patient dag resurs bokar in.
    \item Planerare jag tänkte spontant, man väljer en dag, väljer en operatör och sal så filtrerar listan med beslut.
    \item Christoffer man skulle kunna design det lite bakifrån, att man väljer.
    \item Daniel detta är det som är så intressant eftersom det vi finner så många olika aspekter som vi faktiskt fångar upp när man har något att utgå ifrån.
    \item Gunnar det är är så roliga och det som är halva syftet med att göra en prototyp.
    \item Gunnar om man skulle vilja göra en förändring i systemet en start koppling mellan sal och doktor.
    \item (OBS!) personliga anteckningar per beslut vore trevligt.
    \item Christoffer förklarar lite om hur data är organiserad och hur man kan söka i den datan (Fast med flera och lättare ord :) )
    \item Gunnar det var intressant för då kan man sortera salar på beslut.
    \item Christoffer det är väl planen att vi kommer få stöd för.
    \item fråga Jeremias finns det någon lista på vad en doktor klarar av. Svar: Planerare ja, i mitt huvud.
    \item Gunnar doktorer är specialiserar sig så även om flera har kompetens finns det specialfall då en specialist.
    \item Christoffer visar pilar och boxar diagrammet Gunnar blir glad.
    \item Planerare kan man ta antal dagar till sista garantidagen på passerade. Planerare 2 jag vill ha hur många dagar dom har väntat. (OBS!)
    \item där det står passerat så kan man skriva minus. (OBS!)
    \item Planerare det finns olika tids som man ska ta hänsyn till.
    \item Daniel  vi har medvetet inte lagt in andra tidsramar en det som vi har nu.
\end{itemize}

\section{Resultat}
\emph{[Vad blev resultat och blev den godkänd]}
\end{document}
