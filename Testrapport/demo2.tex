\documentclass[a4paper,10pt, twoside]{article}
\usepackage[margin=1.4in]{geometry}
\usepackage[swedish]{babel}
\usepackage[utf8]{inputenc}
\usepackage{titlesec}
\usepackage{titling}
\usepackage{todonotes}



\setlength{\parskip}{1em}
\setlength{\parindent}{0pt}
\titlespacing{\section}{0pt}{\parskip}{-\parskip}
\titlespacing{\subsection}{0pt}{\parskip}{-\parskip}
\titlespacing{\subsubsection}{0pt}{\parskip}{-\parskip}
\titlespacing{\part}{0pt}{\parskip}{-\parskip}


\title{Demo2}
\author{Testansvarig: Kurt Karllson\\Observatör: Anna Alm.}
\date{\today}

\begin{document}
\maketitle\clearpage

\section{Resurser}
\emph{[Vilka resurser kommer att användas i testet, med Referenser om så behövs]}
\begin{itemize}
    \item
\end{itemize}

\section{Test log}

\begin{itemize}
    \item chr börjar med att start upp webbsidan.
    \item chr visar beslut vy
    \item fråga: boka för sent (i det röda), svar ja
    \item fråga: hur man ska välja dag, svar sys båda fungerar för mig. sys missar ingen information så möjligvis ett onödigt tryck med knapp. sys skulle ha en vint med mer information.
    \item fråga det ska synas om det inte går och boka en dag! OBS! svar: chr skulle kunna gråa ut den dagen.
    \item tex på röda dager så är det dumt och boka samma med röd dagar.
    \item sys kontinuerligt pussel att få ihop för att fylla dagar.
    \item sys veckovy är intressant, då den kan visa mycket information om vad en vecka innehåller. Ser ut som outlook ungeför idag. Kan vara en låsning som finns i mitt eget huvud men för mig i mitt vardagiga arbete så har det varit av mycket hjälp.
    \item sys man gör en bedömning att ett beslut kan skjutas för att ge plats till ett beslut som har mer prio.
    \item fråga men ni gör mycket ommöbleringar OBS!, svar ja, med det är det som vi känner skulle vara så bra men en dator som kan ge tillgång till så många olika tillgångar faktorer.
    \item sys jag vill kunna se allt i eventbokningen kalendern som jag ser här *pekar på sin anteknings perm*
    \item visar material listan
    \item sys det hade varit bra och se hur det gick så man får referens punkter
    \item chr förklarar tanken bakom tidslinjen
    \item disskution om arbetsflödet välj passient dag resu bokar in.
    \item sys jag tänkte spontant, man väljer en dag, väljer en operatör och sal så filtrerar listan med beslut.
    \item chr man skulle kunna design det lite bakifrån, att man väljer.
    \item dr detta är det som är så interssant eftersom det vi finner så många olika aspekter som vi faktiskt fångar upp när man har något att utgå ifrån.
    \item Gunnar det är är så roliga och det som är halva syftet med att göra en prototyp.
    \item Gunnar om man skulle vilja göra en förändring i systemet en start koppling mellan sal och doktor.
    \item (OBS!) personliga antekningar per beslut vore trevligt.
    \item chr förklarar lite om hur data är organiserad och hur man kan söka i den datan (Fast med flera och lättare ord :) )
    \item Gunnar det var intressant för då kan man sortera salar på beslut.
    \item chr det är väll planen att vi kommer få stöd för.
    \item fråga jer finns det någon lista på vad en doktor klarar av. Svar: Sys ja, i mitt huvud.
    \item Gunnar doktorer är specialiserar sig så även om flerar har kompitens finns det specialfall då en speciallist.
    \item chr visar pilar och boxar digramet Gunnar blir glad.
    \item sys kan man ta antal dagar till sista garantidagen på passerade. sys2 jag vill ha hur många dagar dom har väntat. (OBS!)
    \item där det står passerat så kan man skriva minus. (OBS!)
    \item sys det finns olika tids som man ska ta hänsyn till.
    \item dr vi har medvetet inte lagt in andra tidsramar en det som vi har nu.
\end{itemize}

\section{Resultat}
\emph{[Vad blev resultat och blev den godkänd]}
\end{document}
