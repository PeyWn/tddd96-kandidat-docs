\documentclass[a4paper,10pt, twoside]{article}
\usepackage[margin=1.4in]{geometry}
\usepackage[swedish]{babel}
\usepackage[utf8]{inputenc}
\usepackage[T1]{fontenc}
\usepackage{titlesec}
\usepackage{titling}
\usepackage{todonotes}

\usepackage{graphicx}
\usepackage{fancyhdr}

\pagestyle{fancy}
\rhead{\includegraphics[width=1cm]{../Templates/Aeon}}
\lhead{}

\setlength{\parskip}{1em}
\setlength{\parindent}{0pt}
\titlespacing{\section}{0pt}{\parskip}{-\parskip}
\titlespacing{\subsection}{0pt}{\parskip}{-\parskip}
\titlespacing{\subsubsection}{0pt}{\parskip}{-\parskip}
\titlespacing{\part}{0pt}{\parskip}{-\parskip}

\def\ftitle{Testrapport Iteration 3}
\def\fversion{0.1}

\begin{document}
\begin{titlepage} % Suppresses displaying the page number on the title page and the subsequent page counts as page 1
	\newcommand{\HRule}{\rule{\linewidth}{0.5mm}} % Defines a new command for horizontal lines, change thickness here

	\center % Centre everything on the page

	%------------------------------------------------
	%	Headings
	%------------------------------------------------

	\textsc{\LARGE Linköpings universitet \\ \vspace{0.2em} Institutionen för datavetenskap }\\[2cm]

    \large\today

    \vspace{1cm}


	%------------------------------------------------
	%	Title
	%------------------------------------------------

	\HRule\\[0.4cm]

	{\huge\bfseries Schemaläggningsstöd för  \vspace{.1em} \\ kirurgi  - \ftitle }\\[0.4cm] % Title of your document

	\HRule\\[1cm]

	%------------------------------------------------
	%	Author(s)
	%------------------------------------------------

	\begin{minipage}{0.7\textwidth}
			\large
            \emph{Version: \fversion}
            \vspace{1em}

            \textbf{\\Adam Andersson, Niclas Byrsten, \\Björn Hvass, Hendrik Lindström, \\Martin Persson, Christoffer Sjöbergsson, \\Tor Utterborn}
            

            \vspace{1em}

            Handledare: Jonas Wallgren

            Examinator: Kristian Sandahl
	\end{minipage}
	~

	%------------------------------------------------
	%	Logo
	%------------------------------------------------

	%\vfill\vfill
	%\includegraphics[width=0.2\textwidth]{../Templates/Aeon}\\[1cm] % Include a department/university logo - this will require the graphicx package

	%----------------------------------------------------------------------------------------

	\vfill % Push the date up 1/4 of the remaining page

\end{titlepage}


\newpage
\section{Status}
Ett gränssnitt med viss funktionalitet har utvecklats. Under iteration 2 så har det nuvarande gränsnittet testats 
mot kund.  

\section{Testrapport iteration 3}
Nedan så listas en sammanfattning av de tester som har gjorts av gruppen under iteration 3.  

\section{Acceptanstest - DesignTest 1}
\subsection{Testaktivitet}
Systemets front-end del ska visas upp för kund under ett möte och en diskussion ska föras kring designen.
\subsection{Testresultat}
Testet resulterade i ett testprotokoll från mötet med kund.

\subsection{Avvikelser}
Inga avvikelser.
\subsection{Utvärdering}
Eftersom testet var utformat mer som en undersökning så anses det godkänt då det genomfördes enligt specifikation.

\clearpage
\appendix
\section{Testprotokoll DesignTest 1}
Testansvarig: Björn Hvass\\Observatör: Christoffer Sjöbergsson\\Datum: 23 april 2018

\subsection{Resurser}
Personal från sjukhuset:
\begin{itemize}
	\item Projektledare
    \item Schemaläggare (2st)
	\item Designers UX (2st)
    \item Tekniska kompetenser (3st)
\end{itemize}

\subsection{Test log}
\emph{
\begin{itemize}
    \item Björn går igenom alla designelement som ska utvärderas.
   	\item En av schemaläggarna funderar över strukturen och skulle föredra att planeringen sker utifrån en sal 			
   	istället för en patient
    \item På grund av att prototypen var ofullständig så uppstod förvirring i hur vi hade tänkt, efter en längre 		
    förklaring och demonstrerande av olika idéer kom gruppen fram till att det som designats följer kraven som 		
    satts upp och att detta är rätt. Däremot finns det ytterligare krav som inte identifierats tidigare.
    \item Kraven som dök upp var i relation till att planeringen bör gå att utföra baklänges. Med detta menas att det 	
    finns önskemål om att istället för att filtrera schemat utifrån ett valt beslut så vill de kunna filtrera 				
    listan med beslut utifrån en vald sal och dag.
\end{itemize}
}
\clearpage

\section{Testspecifikation DesignTest 1}
Nedan listas specifikationen för testet.

\subsection{Typ av test}
Testets avsikt var att stämma av de designmässiga aspekterna av front-end med kund, typen av test blir såldes ett 			
acceptanstest med modifikation.

\subsection{Genomförande}
Systemets front-end del visades upp för kund under ett möte och en diskussion fördes kring designen.

\subsection{Testscenarion}
Designimplementation med fokus på sidopanelen och kalendervyn.

\subsection{Förväntat resultat}
En testrapport som speglar diskussionen som fördes under kundmötet.

\end{document}
