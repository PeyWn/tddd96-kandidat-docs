\documentclass[a4paper,10pt]{article}
\usepackage[margin=1.4in]{geometry}
\usepackage[swedish]{babel}
\usepackage[utf8]{inputenc}
\usepackage[T1]{fontenc}
\usepackage{titlesec}
\usepackage{titling}
\usepackage{todonotes}

\usepackage{graphicx}
\usepackage{fancyhdr}

\pagestyle{fancy}
\rhead{\includegraphics[width=1cm]{../Templates/Aeon}}
\lhead{}

\setlength{\parskip}{1em}
\setlength{\parindent}{0pt}
\titlespacing{\section}{0pt}{\parskip}{-\parskip}
\titlespacing{\subsection}{0pt}{\parskip}{-\parskip}
\titlespacing{\subsubsection}{0pt}{\parskip}{-\parskip}
\titlespacing{\part}{0pt}{\parskip}{-\parskip}

\def\ftitle{Testrapport}
\def\fversion{0.1}

\begin{document}
\begin{titlepage} % Suppresses displaying the page number on the title page and the subsequent page counts as page 1
	\newcommand{\HRule}{\rule{\linewidth}{0.5mm}} % Defines a new command for horizontal lines, change thickness here

	\center % Centre everything on the page

	%------------------------------------------------
	%	Headings
	%------------------------------------------------

	\textsc{\LARGE Linköpings universitet \\ \vspace{0.2em} Institutionen för datavetenskap }\\[2cm]

    \large\today

    \vspace{1cm}


	%------------------------------------------------
	%	Title
	%------------------------------------------------

	\HRule\\[0.4cm]

	{\huge\bfseries Schemaläggningsstöd för  \vspace{.1em} \\ kirurgi  - \ftitle }\\[0.4cm] % Title of your document

	\HRule\\[1cm]

	%------------------------------------------------
	%	Author(s)
	%------------------------------------------------

	\begin{minipage}{0.7\textwidth}
			\large
            \emph{Version: \fversion}
            \vspace{1em}

            \textbf{\\Adam Andersson, Niclas Byrsten, \\Björn Hvass, Hendrik Lindström, \\Martin Persson, Christoffer Sjöbergsson, \\Tor Utterborn}
            

            \vspace{1em}

            Handledare: Jonas Wallgren

            Examinator: Kristian Sandahl
	\end{minipage}
	~

	%------------------------------------------------
	%	Logo
	%------------------------------------------------

	%\vfill\vfill
	%\includegraphics[width=0.2\textwidth]{../Templates/Aeon}\\[1cm] % Include a department/university logo - this will require the graphicx package

	%----------------------------------------------------------------------------------------

	\vfill % Push the date up 1/4 of the remaining page

\end{titlepage}


\section*{\begin{center}Dokumentationshistorik\end{center}}
    \begin{center}
        \begin{tabular}{|c c c c |}
        \hline
        Datum & händelse & iteration & version\\
        \hline
        2018-02-27 & Dokument skapas & 2 &  0.1\\
        \hline
        \end{tabular}
    \end{center}
    \clearpage
    \tableofcontents
    \clearpage


\section{Enhetstest}
\subsection{Testaktivitet}
\subsection{Testresultat}
\subsection{Avvikelser}
\subsection{Utvärdering}
\section{Integrationstest}
\subsection{Testaktivitet}
\subsection{Testresultat}
\subsection{Avvikelser}
\subsection{Utvärdering}
\section{Systemtest}
\subsection{Testaktivitet}
\subsection{Testresultat}
\subsection{Avvikelser}
\subsection{Utvärdering}
\section{Acceptanstest}
\subsection{Pappersprototyp}
Två prototyperna testades genom att Daniel från Region Östergötland fick i uppdrag att boka in en operation i pappersprototypen. När han gjorde ett val så uppdaterade Martin gränssittet för att reflektera förändringen till följd av valet. Under testet så talade Daniel om vad han såg och hur han tänkte att han borde göra.
\subsection{Testresultat}
Utifrån de två prototyperna så listas de viktigaste punkterna som saknades i prototyperna nedan:
\begin{itemize}
\item Det kan vara bra med en kom ihåg mig ruta
\end{itemize}
\subsection{Utvärdering}
\emph{Blev testet godkänt}


\end{document}
