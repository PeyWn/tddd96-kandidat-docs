\documentclass[a4paper,10pt]{article}
\usepackage[margin=1.4in]{geometry}
\usepackage[swedish]{babel}
\usepackage[utf8]{inputenc}
\usepackage[T1]{fontenc}
\usepackage{titlesec}
\usepackage{titling}
\usepackage{todonotes}

\usepackage{graphicx}
\usepackage{fancyhdr}

\pagestyle{fancy}
\rhead{\includegraphics[width=1cm]{../Templates/Aeon}}
\lhead{}

\setlength{\parskip}{1em}
\setlength{\parindent}{0pt}
\titlespacing{\section}{0pt}{\parskip}{-\parskip}
\titlespacing{\subsection}{0pt}{\parskip}{-\parskip}
\titlespacing{\subsubsection}{0pt}{\parskip}{-\parskip}
\titlespacing{\part}{0pt}{\parskip}{-\parskip}

\def\ftitle{Testspecifikation-pappersprototyp}
\def\fversion{1.0}
 %Testspecifikation: Den här specifikation ska innehålla information om hur %testet ska genomföras. Den ska också innehålla lämpliga testscenarion. Utöver %det så ska det finnas en sektion som redogör för designen av komponenten och %om den är uppfylld enligt kravspecifikationen. Vidare så ska det finnas en %sektion som redogör för förväntat resultat, enligt kravspecifikation eller %utifrån beskrivningen av enheten.
\begin{document}
\section{Pappersprototyp}
Två stycken pappersprototyper hade tagits fram för testning.
\section{Typ av test}
Det här testet skulle fokusera på designen av gränssittet
\section{Genomförande}
De två första prototyperna testades genom att Daniel(RÖ) fick i uppdrag att boka in en operation i pappersprototypen. När han gjorde ett val så uppdaterade Martin gränssittet för att reflektera förändringen till följd av valet. Under testet så talade Daniel om vad han såg och hur han tänkte att han borde göra. Vid olika tillfällen så flikade de övriga personerna från RÖ in med tankar och åsikter.
\section{Testscenarion}

\subsection{Prototyp 1}
\subsubsection{Design av komponent}
\emph{[Beskriv design av komponent och om den är uppfylld enligt Kravspec]}
\subsubsection{Förväntat resultat}
\emph{[Det förväntade resultat enligt kravspec eller beskrivning av enheten]}
\subsection{Prototyp 2}
\subsubsection{Design av komponent}
\emph{[Beskriv design av komponent och om den är uppfylld enligt Kravspec]}
\subsubsection{Förväntat resultat}
\emph{[Det förväntade resultat enligt kravspec eller beskrivning av enheten]}
\end{document}
