\documentclass[a4paper,10pt]{article}
\usepackage[margin=1.4in]{geometry}
\usepackage[swedish]{babel}
\usepackage[utf8]{inputenc}
\usepackage{titlesec}
\usepackage{titling}
\usepackage{todonotes}



\setlength{\parskip}{1em}
\setlength{\parindent}{0pt}
\titlespacing{\section}{0pt}{\parskip}{-\parskip}
\titlespacing{\subsection}{0pt}{\parskip}{-\parskip}
\titlespacing{\subsubsection}{0pt}{\parskip}{-\parskip}
\titlespacing{\part}{0pt}{\parskip}{-\parskip}


\title{Projektplan}
\author{Grupp 6}


\begin{document}
\begin{titlingpage}
    \maketitle

\end{titlingpage}
\section*{\begin{center}Dokumentationshistorik\end{center}}
\begin{center}
\begin{tabular}{|c c c c |}
 \hline
 Datum & händelse & iteration & version\\
 \hline
 2018-01-16 & Dokument skapas & 1 &  0.1\\
 \hline
\end{tabular}
\clearpage
\end{center}
\tableofcontents
\clearpage
\section{Projektbeskrivning}
Här ges en översiktlig beskrivning av projektet. Underrubrikerna beskriver varför projektet utförs och vad målet med projektet är, samt under vilken tidsrymd projektet utförs och vilka begränsningar projektet har.
\subsection{Bakgrund}
På civilingenjörsutbildningen i datavetenskap samt mjukvaruteknik så utför man, under vårterminen på det 3:e året, ett kandidatarbete med kurskod TDDD96. Arbetet går ut på att studenter ska övas i att utföra ett mjukvaruprojekt där man arbetar i grupp mot en kund och ska producera en produkt. Gruppen som ligger bakom denna projektplan har kandidatarbetet att bygga ett schemaläggningsstöd för kirurgi till Region Östergötland.

Region Östergötland utförde under hösten 2017 ett projekt, GOLI(a)T-projektet, med syfte att ta fram förslag på en gemensam operationsprocess för de olika verksamheter som ingår i regionen.

För att ytterligare effektivisera arbetet under den framtagna operationsprocessens planeringsfas och minska väntetider efterfrågar Region Östergötland nu ett grafiskt schemaläggningsverktyg som kan användas av operationsplanerare för att koordinera vårdresurser.
 
När ett beslut om operation har fattats måste en tid för operation bokas. Det är nödvändigt att rätt specialkompetens, utrustning och lokaler för operationen finns tillgängliga. Schemaläggningsverktygets syfte kommer därför vara att visa tider där dessa villkor är uppfyllda och då operationen kan bokas in.

Det system som regionen för närvarande använder håller på att tas ur drift, och fram till det att ett nytt system kan upphandlas så behövs en temporär lösning. Den lösningen kan också vara till hjälp vid upphandlingen genom att belysa vilken funktionalitet verksamheterna inom regionen efterfrågar.
\subsection{Projektets begränsningar}
Projektet är begränsat till att uppfylla de krav som specificeras i kravspecifikationen. Budgeten för projektet är begränsad till 400 timmars arbete per projektmedlem.
\subsection{Projektets mål}
Målet med projektet är i första hand att ge gruppens medlemmar erfarenhet med att jobba mot en kund med ett större projekt i form av ett utvecklingsteam på 7 medlemmar. Detta innefattar att varje medlem ska få testa på att ta ansvar och vara drivande inom ett visst område beroende på dess roll i teamet. I målet ingår även att teamet ska arbeta för att skapa största möjliga värde för kunden.

Ett annat mål är att projektgruppen ska skapa en produkt åt kund, den optimala slutstatusen på detta mål är om teamet lyckas skapa en färdig produkt som kund är redo att ta i drift vid projektets slut.
\subsection{Start- och slutdatum}
2018-01-15. 
Datumet då projektet tog sin början då gruppmedlemmarna träffades vid kursintroduktionen.

2018-01-22
Efter att gruppen valt roller samt diskuterat de olika projekten som fanns tillgodo tilldelades de ett projekt, då kunde projektet starta igång på riktigt.

2018-05-23
Slutredovisning av projektet.

2018-05-28 
Slutversion av kandidatarbetet inlämnat till examinator.
\section{Tid- och resursplan}
Gruppen kommer arbeta dels med den tid (budget) som är tilldelad detta projekt men även de resurser som gruppen har i form utav förkunskaper hos de olika medlemmarna. Med hjälp av tiden och kunskapen så kommer ett antal aktiviteter att utföras, aktiviteter som sedan ligger till grund för milstolpar och beslutspunkter.
\subsection{Milstolpar}
Inom projektet så kommer fyra stycken iterationer att göras mot handledare och examinator. Då vi arbetar med Scrum-metoden så kommer vi även att ha sprintar i intervall om två veckor, dessa sammanfaller ibland med de fyra större iterationerna. 
\subsubsection{Iteration 1}
Deadline: 2018-02-19
Dokumentation från förstudien ska lämnas in till handledare och examinator, den dokumentation som ska levereras är:
\begin{itemize}
\item Projektplan
\item Kravspecifikation
\item Kvalitetsplan
\item Statusrapport
\item Systemanatomier
\item Påbörjad arkitekturdokument
\item Påbörjad testplan
\end{itemize}
\subsubsection{Iteration 2}
Deadline: 2018-03-05
Ytterligare dokumentation ska levereras till handledare och examinator:
\begin{itemize}
\item Preliminär rapporthalva utav kandidatarbetet
\item Arkitekturdokument
\item Testplan och testrapport för iteration två
\item Utvärdering av iteration två
\end{itemize}
Sedan ska även de tidigare inlämnade dokumenten skickas in så att handledare och examinator kan se hur dessa har förändrats.
\subsubsection{Iteration 3}
Deadline: 2018-04-23
\subsubsection{Iteration 4}
Deadline: 2018-05-07
Utkast två av kandidatarbetet ska lämnas in.
\subsubsection{Iteration 5}
Deadline: 2018-05-28
Slutversion av kandidatarbetet ska lämnas in till Kristian.
\subsection{Beslutspunkter}
De beslutspunkter som existerar i projektet är främst när vi har sprint-planering inför varje sprint, då bestämmer gruppen vad som ska utföras de kommande två veckorna och hur de ska utföras. Sedan kommer det även finnas mindre beslutspunkter såsom dagliga Scrum-möten där man diskuterar problem och hur de ska lösas. 
Datum för sprint-planering tillika start av en sprint:
\begin{itemize}
\item 2018-02-19
\item 2018-03-05
\item 2018-03-19
\item 2018-04-09
\item 2018-04-23
\item 2018-05-07
\item 2018-05-21
\end{itemize}
\subsection{Delresultat}
Under projektets gång så kommer ett antal delresultat att arbetas fram, dessa är till stor del kopplade med antingen milstolpar och beslutspunkter (sprintar). Ser man till de milstolpar som existerar i projektet så är det där angivet vad det är för resultat som ska ha uppnåtts till angivet datum. Det kommer även att finnas resultat vid slutet av varje sprint, dessa är samma datum som finns under beslutspunkter, eftersom den nya sprinten planeras när den gamla redovisas och utvärderas. Dessa sprint-resultat kommer att utarbetas vid respektive sprint-planering men de kommer utgå ifrån den kravspecifikation som existerar samt de iterationer som finns under rubriken Milstolpar.
\subsection{Aktiviteter}
Under projektets gång så kommer ett antal aktiviteter att utföras för att nå de mål som satts upp för projektet, dessa aktiviteter innefattar:
\begin{itemize}
\item Grupparbetstider: Gruppen träffas och diskuterar tillsammans omkring lösningar på problem samt delar upp de olika uppgifter som ska utföras.
\item Föreläsningar: I syfte att utbilda projektgruppen inom hur man arbetar i projekt, rapportskrivning m.m.
\item Seminarier: Gruppen deltar aktivt i olika problem som man kan ställas inför, t.ex. opponering, att pitcha en idé m.m.
\item Kundmöten: Hela gruppen eller några stycken träffar kund och stämmer av läget eller diskuterar t.ex. implementation eller kravspecifikation.
\item Handledarmöten: Gruppen träffar sin handledare en gång i veckan för att uppdatera om status på projektet och ha möjligheten att få handledning gällande saker som gruppen har fastnat på eller funderar över.
\item Dokumentskrivning: Kontinuerlig dokumentering av projektet för att sedan lämnas in till handledare och examinator i flera iterationer.
\item Scrum-möten: Gruppen har valt att implementera Scrum-metoden och kommer således ha dagliga Scrum-möten, Sprint-planering, sprintgenomgång och återblick.
\item Utveckling: Gruppen kommer lägga en större del av projektet på utveckling och implementation av den programvara som ska levereras till kund.
\end{itemize}
\subsection{Resurser}
Resursplanen bygger på de sju gruppmedlemmarnas tidsbudget á 400 timmar. Milstolpar och beslutspunkter har grundats på den sammanslagna budgeten av 2800 timmar uppdelat över 4 stycken iterationer under projektets gång. Ser man till de antal veckor som gruppen har på sig att uppfylla dessa timmar så ska varje person arbeta 20 timmar i veckan, dessa timmar (resurser) kommer inte att vara uppdelade på exakt detta sätt då vissa veckor kommer man behöva lägga större resurser medans man lägger färre resurser under andra veckor. 
Utöver resursen tid så är även projektgruppens förkunskaper en bra resurs för att uppnå de mål som finns för projektet. Samtliga medlemmar har studerat i snart tre år på civilingenjörsutbildningen i datavetenskap eller mjukvaruteknik. Detta innebär att medlemmarna har gedigna kunskaper inom området mjukvaruutveckling vilket leder till att de kan använda dessa kunskaper (resurser) i detta projekt.
Projektgruppen har läst en teorikurs i programutvecklingsmetodik och har valt att tillämpa arbetssättet Scrum, detta är en användbar resurs för gruppen då det är en bevisat effektiv metod för att genomföra projekt.
Vid gruppmöten så kommer lokaler på universitet att utnyttjas.
\section{Projektorganisation}
Projektorganisationen är uppdelad i ett antal roller där alla i projektet har ansvar för en roll. Rollerna är listade nedan med en beskrivning på vad rollen har för ansvar, vad som ska göras och vilka områden som respektive roll har beslutsrätt inom.
\subsection{Roller}
I projektet existerar det sju stycken olika roller som är menat att uppfylla de olika uppgifter som behöver genomföras. De projektroller som finns är:
\begin{itemize}
\item Teamledare
\item Analysansvarig
\item Arkitekt
\item Utvecklingsledare
\item Testledare
\item Kvalitetssamordnare \& Dokumentansvarig
\item Konigurationsansvarig
\end{itemize}
\subsubsection{Teamledare}
Teamledaren är den person som ska leda arbetet i projektet och coacha projektmedlemmarna i deras arbete samt ansvarar för att gruppen ska ha en bra arbetsmiljö. Rollen har ansvar för att de mål som sätts upp för projektet ska uppfyllas och att de processer som är uppsatta efterföljs. Teamledaren har även ansvaret för att denna projektplan skrivs av projektgruppen. Till skillnad från rollen projektledare så planerar inte en teamledare arbetet för varje person, det är projektgruppens ansvar att göra detta.

Utåt sett så är det teamledaren som är representanten för teamet, med detta menas det att det är teamledaren som har hand om de externa kontakterna, i alla fall till en början.

Vid behov så är det teamledaren som har sista ordet.
\subsubsection{Analysansvarig}
Rollen som analysansvarig går ut på att vara projektgruppens kontakt till kunden. Med hjälp utav dialog och iterationer så ska analysansvarig kunna ta reda på kundens behov, förhandla dessa och föra detta vidare till övriga gruppen. 

Analysansvarig håller mycket kontakt med arkitekten för att kunna förmedla de krav kunden har och se om detta är möjligt att åstadkomma. Om arkitekten ser förhinder i att uppnå de krav som kunden har så får analysansvarig förhandla mot kunden för att antingen ändra de krav som finns eller för att finna en annan lösning på det problem som gruppen ska lösa.
En kontinuerlig dialog förs även med testledaren, detta för att diskutera de olika sätt som produkten kan testas för att säkerställa att de krav som är uppställda är uppnådda.

Det är även analysansvarig som har ansvaret för att dokumentera de krav som kunden har på produkten.
\subsection{Arkitekt}
Arkitekten är den person som ansvarar för att en stabil arkitektur för produkten plockas fram. Rollen är en styrande röst i tekniska frågor och teknikval då den har koll på och identifierar komponenter samt gränssnitt i produkten.

Arkitekten har god kontakt med analysansvarig för att kunna skapa en arkitektur som uppnår de krav som kunden har men även har ansvaret att, genom analysansvarig, ge feedback till kund om vad som är tekniskt möjligt utifrån gruppens kunskaper och tekniska möjligheter. Både vid feedback mot kund och vid kommunikation med resterande projektgrupp så är det viktigt att arkitekten kan kommunicera bärande idéer.

Det är arkitektens ansvar att dokumentation av arkitektur skapas.
\subsection{Utvecklingsledare}
Utvecklingsledaren ansvarar för den detaljerade designen i produkten. Utvecklingsledaren ansvarar för att utvecklingsarbetet flyter på smidigt och därmed leder och, vid behov, fördelar utvecklingsarbetet. I detta projekt så är alla med och hjälper till vid utvecklingen, så vid utvecklingsstadiet i projektet så leds alla i projektgruppen av utvecklingsledaren oavsett vilken roll man innehar.

Det ligger på utvecklingsledaren att fatta beslut om den utvecklingsmiljö som ska användas utav projektgruppen. 
\subsection{Testledare}
Rollen som testledare går ut på att processen kring testning ska vara strukturerad och utföras på ett bra sätt. Testledaren beslutar om produktens status och sköter den dynamiska verifieringen och valideringen av detta genom exekvering. Testledaren har kommunikation med analysansvarig vid testning för att se att de krav som är uppsatta för produkten uppfylls. Rollen har även samarbete tillsammans med rollen kvalitetssamordnare \& dokumentansvarig för att testa de kvalitetskrav som finns för produkten.

Testledaren ansvarar för dokumentation av testplan och testrapport.
\subsection{Kvalitetssamordnare \& Dokumentansvarig}
Den sammanslagna rollen kvalitetssamordnare \& dokumentansvarig ansvarar delvis för att kvalitetsarbete utförs överallt av alla roller och att dokumentmallar ska finnas tillgängliga för projektet.

Som kvalitetssamordnare så har man initiativ- och uppföljningsansvar. Man ska planera och budgetera ihop med övriga gruppen och fundera över hur mycket kvalitet får kosta i projektet. Det åligger kvalitetssamordnare att skriva en kvalitetsplan för projektet.

Rollen som dokumentansvarig går ut på att leveranser till deadlines ska upprätthållas och att arbeta mycket med konfigurationsansvarig (och kvalitetssamordnare) för att bestämma vilka arbetsprodukter som ska ingå i en utgåva och vilken kvalitet de ska ha. Att ta fram en logotyp för projektgruppen/produkten är också en uppgift som ligger på dokumentansvarig.
\subsubsection{Konfigurationsansvarig}
Konfigurationsansvarig bestämmer vilka arbetsprodukter som ska versionshanteras och vilka av dessa som ska ingå i en release av en produkt. Ansvaret för att välja och underhålla verktyg för versions- och konfigurationshantering ligger också på denna roll. Konfigurationsansvarig ska även kontrollera att dessa verktyg används på rätt sätt av resten utav projektgruppen.

Konfigurationsansvarig arbetar mycket med utvecklingsledaren och dokumentansvarig så att utveckling och release av arbetsprodukter ska ske på ett konsekvent och korrekt sätt.
\subsection{Kunskaper och färdigheter}
Fem av gruppens medlemmar läser D-programmet vid Linköpings universitet och resterande två läser U-programmet vid samma universitet. Gruppens medlemmar har under sina respektive studietider bekantat sig med ett flertal olika programmeringsspråk och besitter därför en bred teknisk kompetens. 

De har även tidigare arbetat i projekt av liknande typ som detta. D-studenterna har genomfört ett projekt i konstruktion med mikrodator och U-studenterna har genomfört ett AI-projekt. Samtliga gruppmedlemmar är därför medvetna om vilket ansvar som följer av att vara delaktig i större mjukvaruprojekt.
\subsection{Kompetensutveckling}
Varje gruppmedlem ansvarar själv för att tillgodogöra sig relevant kunskap när projektet kräver det. För mer ingående kunskap inom områden som innefattas av någon projektroll ansvarar gruppmedlemmen med respektive projektroll, och denne kan förväntas assistera eller utbilda övriga gruppmedlemmar vid behov.
\subsection{Kommunikation och rapporter}
Gruppen har satt upp som mål att i så stor mån det går träffas under de schemalagda tiderna som finns för projektet, detta för att man ungefär en gång per dag ska kunna ge en statusrapport gällande vad man har gjort dagen innan och vad man ska göra under dagen. 

Utöver de dagliga rapporterna till hela gruppen så skickar gruppen, en gång i veckan, in en veckorapport samt en tidsrapport till examinator samt handledare så att de kan följa hur projektet fortgår. 
Slutligen så har projektgruppen möte med handledaren en gång per vecka där man tar upp vad varje medlem har gjort sedan senaste handledarmötet, om det är något specifikt som behöver tas upp med gruppen och handledaren samt vad alla ska göra fram till nästa möte.
\section{Riskhantering}
Under projektets gång så kan det uppstå händelser som kan påverka slutresultatet på ett negativt sätt. Genom att identifiera dessa risker och sätta ett mått på hur allvarlig var och en  är, så kan en handlingsplan tas fram för att hantera risken.
\subsection{Risker, sannolikhet och inverkan}
För att kunna ta reda på hur allvarlig varje risk är så sätts ett värde från 1 till 4 på både sannolikheten att den inträffar och vilken inverkan den skulle få för projektet. Slutligen så fås en riskfaktor genom att  multiplicera sannolikheten med dess inverkan. Detta värde ger en indikation på hur allvarlig en risk är. Riskerna som identifierades för detta projekt visas i tabell 1.

\textbf{Tabell 1: Risker i projektet}\\
\begin{tabular}{|c c c c c|}
 \hline
 Nr & Beskrivning & Sannolikhet & Inverkan & Riskfaktor\\
 \hline
 1 & Någon blir sjuk & 3 & 2 & 6\\
 2 & Någon hoppar av kursen & 1 & 3 & 3\\
 3 & Storleken av projektet blev större än förväntat & 2 & 3 & 6\\
 4 & Krav från kund ändras sent i projektet & 2 & 3 & 6\\
 5 & Personer i projektgruppen blir oense & 3 & 3 & 9\\
 \hline
\end{tabular}
\subsection{Begränsningar och åtgärdsplan}
Om det inte går att undvika en risk så gäller det att antingen minska sannolikheten att den inträffar och/eller minska dess inverkan. Nedan beskrivs åtgärder för varje risk i tabell 1.
 
\textbf{1. Någon blir sjuk}\\
Åtgärd: Om personen håller på med arbete som andra är beroende av så får man fördela om det arbetet på andra personer så att man kan komma vidare i projektet. Beroende på sjukdom så finns även möjligheten att personen arbetar hemifrån så att projektet inte blir lidande. Projektgruppen kommer också arbeta i olika branches i Git för att det ska ske så lite påverkan som möjligt för varandra om man skulle bli sjuk.

\textbf{2. Någon hoppar av kursen}\\
Åtgärd: Beroende på hur långt in i projektet man har kommit så kan olika åtgärder vidtas. Om det är tidigt eller halvvägs i projektet så får man ta en diskussion med kund och eventuellt justera kravspecifikationen efter antalet personer som man är i projektet. Man får även titta på hur man ska lösa personens roll i teamet, går det att lägga rollen på någon annan person eller kan man fördela upp den på flera personer.
Om man skulle befinna sig nära slutet på projektet så kanske det är möjligt att genomföra de krav som finns kvar och därmed kanske den enda åtgärd som behöver genomföras är att fördela personens roll på övriga personer i projektet.

\textbf{3. Storleken på projektet blev större än förväntat}\\
Åtgärd: Gruppen får arbeta med att utveckla en kravspecifikation där man uppskattar den tid som varje krav kan tänkas ta. 
Om man märker att krav tar längre tid än förväntat så får man omvärdera vad som kan tänkas hinnas med och ta kontakt med kund för att omförhandla de krav som existerar.
\textbf{4. Krav från kund ändras sent i projektet}\\
Åtgärd: Om krav från kund förändras så får man se om man har möjlighet, tid- och teknikmässigt, att tillgodose dessa. Om det inte ryms inom projektets ramar så får man förhandla med kund om att antingen inte ta med det kravet överhuvud taget eller om man kan hitta någon kompromiss, en lösning som inte är lika tid- och teknikskrävande.

\textbf{5. Personer i projektgruppen blir oense}\\
Åtgärd: Konflikter i gruppen ska hanteras så snart som möjligt då det kan ha stor påverkan på projektet och stämningen i gruppen. Om personer i gruppen blir oense så får man hantera konflikten olika beroende på vilken typ av konflikt det är. 
Om om det är en sakkonflikt t.ex. vilket programmeringsspråk man ska använda för att skriva front-end delen i projektet så får man låta samtliga sidor yttra sig och berätta varför de anser att deras alternativ är bäst. Därefter så får gruppen försöka komma till ett gemensamt beslut och om inte det lyckas så har teamledaren det sista ordet.
Om det är eller har utvecklats till en personlig konflikt så får man ta in en medlare, någon person i gruppen som inte är involverad i konflikten som får båda parter att lugna ner sig och lösa diskussionen om det fanns någon grund till varför konflikten uppstod.
\end{document}
