\documentclass[a4paper,10pt]{article}
\usepackage[margin=1.4in]{geometry}
\usepackage[swedish]{babel}
\usepackage[utf8]{inputenc}
\usepackage{titlesec}
\usepackage{titling}
\usepackage{todonotes}



\setlength{\parskip}{1em}
\setlength{\parindent}{0pt}
\titlespacing{\section}{0pt}{\parskip}{-\parskip}
\titlespacing{\subsection}{0pt}{\parskip}{-\parskip}
\titlespacing{\subsubsection}{0pt}{\parskip}{-\parskip}
\titlespacing{\part}{0pt}{\parskip}{-\parskip}


\title{Projektplan}
\author{Grupp 6}


\begin{document}
\begin{titlingpage}
    \maketitle

\end{titlingpage}
\section*{\begin{center}Dokumentationshistorik\end{center}}
\begin{center}
 \begin{tabular}{|c c c c |}
 \hline
 Datum & händelse & iteration & version\\
 \hline
 2018-02-05 & Dokument skapas & 1 &  0.1\\
 \hline
\end{tabular}
\clearpage
\end{center}
\tableofcontents
\clearpage
\section{Projektbeskrivning}
Här ges en översiktlig beskrivning av projektet. Underrubrikerna beskriver varför projektet utförs och vad målet med projektet är, samt under vilken tidsrymd projektet utförs och vilka begränsningar projektet har.
\subsection{Bakgrund}
På civilingenjörsutbildningen i datavetenskap samt mjukvaruteknik så utför man, under vårterminen på det 3:e året, ett kandidatarbete med kurskod TDDD96. Arbetet går ut på att studenter ska övas i att utföra ett mjukvaruprojekt där man arbetar i grupp mot en kund och ska producera en produkt. Gruppen som ligger bakom denna projektplan har kandidatarbetet att bygga ett schemaläggningsstöd för kirurgi till Region Östergötland.

Region Östergötland utförde under hösten 2017 ett projekt, GOLI(a)T-projektet, med syfte att ta fram förslag på en gemensam operationsprocess för de olika verksamheter som ingår i regionen.

För att ytterligare effektivisera arbetet under den framtagna operationsprocessens planeringsfas och minska väntetider efterfrågar Region Östergötland nu ett grafiskt schemaläggningsverktyg som kan användas av operationsplanerare för att koordinera vårdresurser.
 
När ett beslut om operation har fattats måste en tid för operation bokas. Det är nödvändigt att rätt specialkompetens, utrustning och lokaler för operationen finns tillgängliga. Schemaläggningsverktygets syfte kommer därför vara att visa tider där dessa villkor är uppfyllda och då operationen kan bokas in.

Det system som regionen för närvarande använder håller på att tas ur drift, och fram till det att ett nytt system kan upphandlas så behövs en temporär lösning. Den lösningen kan också vara till hjälp vid upphandlingen genom att belysa vilken funktionalitet verksamheterna inom regionen efterfrågar.
\subsection{Projektets begränsningar}
Projektet är begränsat till att uppfylla de krav som specificeras i kravspecifikationen. Budgeten för projektet är begränsad till 400 timmars arbete per projektmedlem.
\subsection{Projektets mål}
Målet med projektet är i första hand att ge gruppens medlemmar erfarenhet med att jobba mot en kund med ett större projekt i form av ett utvecklingsteam på 7 medlemmar. Detta innefattar att varje medlem ska få testa på att ta ansvar och vara drivande inom ett visst område beroende på dess roll i teamet. I målet ingår även att teamet ska arbeta för att skapa största möjliga värde för kunden.

Ett annat mål är att projektgruppen ska skapa en produkt åt kund, den optimala slutstatusen på detta mål är om teamet lyckas skapa en färdig produkt som kund är redo att ta i drift vid projektets slut.
\subsection{Start- och slutdatum}
2018-01-15. 
Datumet då projektet tog sin början då gruppmedlemmarna träffades vid kursintroduktionen.

2018-01-22
Efter att gruppen valt roller samt diskuterat de olika projekten som fanns tillgodo tilldelades de ett projekt, då kunde projektet starta igång på riktigt.

2018-05-23
Slutredovisning av projektet.

2018-05-28 
Slutversion av kandidatarbetet inlämnat till examinator.
\section{Tid- och resursplan}
Gruppen kommer arbeta dels med den tid (budget) som är tilldelad detta projekt men även de resurser som gruppen har i form utav förkunskaper hos de olika medlemmarna. Med hjälp av tiden och kunskapen så kommer ett antal aktiviteter att utföras, aktiviteter som sedan ligger till grund för milstolpar och beslutspunkter.
\subsection{Milstolpar}
Inom projektet så kommer fyra stycken iterationer att göras mot handledare och examinator. Då vi arbetar med Scrum-metoden så kommer vi även att ha sprintar i intervall om två veckor, dessa sammanfaller ibland med de fyra större iterationerna. 
\subsubsection{Iteration 1}
Deadline: 2018-02-19
Dokumentation från förstudien ska lämnas in till handledare och examinator, den dokumentation som ska levereras är:
\begin{itemize}
\item Projektplan
\item Kravspecifikation
\item Kvalitetsplan
\item Statusrapport
\item Systemanatomier
\item Påbörjad arkitekturdokument
\item Påbörjad testplan
\end{itemize}
\subsubsection{Iteration 2}
Deadline: 2018-03-05
Ytterligare dokumentation ska levereras till handledare och examinator:
\begin{itemize}
\item Preliminär rapporthalva utav kandidatarbetet
\item Arkitekturdokument
\item Testplan och testrapport för iteration två
\item Utvärdering av iteration två
\end{itemize}
Sedan ska även de tidigare inlämnade dokumenten skickas in så att handledare och examinator kan se hur dessa har förändrats.
\subsubsection{Iteration 3}
Deadline: 2018-04-23
\subsubsection{Iteration 4}
Deadline: 2018-05-07
Utkast två av kandidatarbetet ska lämnas in.
\subsubsection{Iteration 5}
Deadline: 2018-05-28
Slutversion av kandidatarbetet ska lämnas in till Kristian.
\subsection{Beslutspunkter}
De beslutspunkter som existerar i projektet är främst när vi har sprint-planering inför varje sprint, då bestämmer gruppen vad som ska utföras de kommande två veckorna och hur de ska utföras. Sedan kommer det även finnas mindre beslutspunkter såsom dagliga Scrum-möten där man diskuterar problem och hur de ska lösas. 
Datum för sprint-planering tillika start av en sprint:
\begin{itemize}
\item 2018-02-19
\item 2018-03-05
\item 2018-03-19
\item 2018-04-09
\item 2018-04-23
\item 2018-05-07
\item 2018-05-21
\end{itemize}
\subsection{Delresultat}
Under projektets gång så kommer ett antal delresultat att arbetas fram, dessa är till stor del kopplade med antingen milstolpar och beslutspunkter (sprintar). Ser man till de milstolpar som existerar i projektet så är det där angivet vad det är för resultat som ska ha uppnåtts till angivet datum. Det kommer även att finnas resultat vid slutet av varje sprint, dessa är samma datum som finns under beslutspunkter, eftersom den nya sprinten planeras när den gamla redovisas och utvärderas. Dessa sprint-resultat kommer att utarbetas vid respektive sprint-planering men de kommer utgå ifrån den kravspecifikation som existerar samt de iterationer som finns under rubriken Milstolpar.
\end{document}
