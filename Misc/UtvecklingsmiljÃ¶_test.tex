\documentclass{article}
\usepackage[utf8]{inputenc}
\usepackage{parskip}
\usepackage{hyperref}

\title{Guide till utvecklingsmiljö (förslag)}
\author{Henrik Lindström}

\begin{document}
\maketitle
\section{Webstorm}
\begin{enumerate}
\item Gå in på \url{https://www.jetbrains.com/student/}
\item Klicka på ``Apply now" och fyll i infon (använd din studentmail)
\item Gå in på din mail och bekräfta/aktivera
\item Ladda ner/installera WebStorm (\url{https://www.jetbrains.com/webstorm/})
\end{enumerate}
\section{Node.js}
Ladda ner/installera Node.js (\url{https://nodejs.org/en/download/}).
\section{Angular}
Installera Angluar 1.6.5 enligt nedan (1.6.6 har problem).
\subsection{Installation}
\begin{enumerate}
\item Öppna terminal eller kommando-prompt.
\item Skriv in ``npm install -g @angular/cli@1.6.5"
\end{enumerate}
\subsection{Skapa körkonfiguration i WebStorm}
\begin{enumerate}
\item Gå in i Run-\textgreater Edit Configurations...
\item Klicka på gröna plusset i övre vänstra hörnet
\item Välj ``npm"
\item Sätt ``Name:" till förslagsvis ``Start"
\item Sätt ``Scripts:" till ``start"
\item Klicka ``Apply"
\end{enumerate}
\subsection{Köra projekt}
Nu är det bara att köra projektet med den nya ``Start"-konfigurationen och sedan gå in på \url{http://localhost:4200} i valfri webbläsare.

Ändringar i koden byggs om automatiskt och sidan uppdateras direkt.
\subsection{Bygga projekt}
Att bygga projektet genererar en optimerad version av applikationen i ``/Dist" under projektmappen.
\begin{enumerate}
\item Öppna projektmappen i en terminal, t.ex. genom att trycka ALT+F12 i WebStorm.
\item Skriv ``ng build -{}-prod -{}-base-href ./"
\end{enumerate}
\end{document}