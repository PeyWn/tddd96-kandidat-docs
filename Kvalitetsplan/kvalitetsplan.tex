\documentclass[a4paper,10pt]{article}
\usepackage[margin=1.4in]{geometry}
\usepackage[swedish]{babel}
\usepackage[utf8]{inputenc}
\usepackage{titlesec}
\usepackage{titling}
\usepackage{todonotes}



\setlength{\parskip}{1em}
\setlength{\parindent}{0pt}
\titlespacing{\section}{0pt}{\parskip}{-\parskip}
\titlespacing{\subsection}{0pt}{\parskip}{-\parskip}
\titlespacing{\subsubsection}{0pt}{\parskip}{-\parskip}
\titlespacing{\part}{0pt}{\parskip}{-\parskip}


\title{Kvalitetsplan}
\author{Grupp 6}


\begin{document}
\begin{titlingpage}
    \maketitle
    \begin{abstract}
    \noindent Syftet med denna kvalitetsplan är att beskriva de tekniker, aktiviteter och metoder som kommer att användas för att försäkra att en hög kvalitet hålls i projektet. Detta innebär att slutprodukten ska uppfylla alla krav, levereras i tid och hålla sig inom budgeten. 
    \end{abstract}

\end{titlingpage}
\tableofcontents
\clearpage
\section{Dokumentationshistorik}
\label{sec:Dokumentation}



\begin{center}
 \begin{tabular}{|c c c c |} 
 \hline
 Datum & händelse & iteration & version\\ 
 \hline
 2018-02-05 & Dokument skapas & 1 &  0.1\\
 \hline
\end{tabular}
\end{center}

\section{Inledning}
\vspace{5mm}
\label{sec:Inledning}
\subsection{Syfte}
Syftet med denna kvalitetsplan är att beskriva de tekniker, aktiviteter och metoder som kommer att användas för att försäkra att en hög kvalitet hålls i projektet. Detta innebär att slutprodukten ska uppfylla alla krav, levereras i tid och hålla sig inom budgeten.
\subsection{Omfattning}
Detta dokument redogör och refererar till  de delar i projektdokumentation [refererade dokument] som direkt behandlar hur projektets skall uppnå sina högkvalitativa egenskaper.
\subsection{Refererade dokument}
Projektplan, Kodstandard, Dokumentstandard, Testplan
\subsection{Definitioner, namn och förkortningar}

\begin{itemize}
\vspace{5mm}
    \item JavaScript - Ett scriptspråk för webbutveckling
    \item Angular - Ett ramverk för gränssnitt till webbsidor
    \item WebStorm - En utvecklingsmiljö till JavaScript
    \item Git - Ett versionshanteringsverktyg
    \item GitLab - En webb-baserad versionshanterare som använder git
    \item Scrum - En agil utvecklingsmetodik
    \item Asana - Ett webb-baserat verktyg för projekthantering
\end{itemize}

\section{Översiktlig kvalitetsplan}

Denna sektion ger en översikt på hur kvalitetsarbetet i projektet ska utföras.

\subsection{Organisation}

Projektgruppens kvalitetssamordnare är ansvarig för kvalitetssäkringen av projektet och  programvaran. Kvalitetsarbete ska utföras av samtliga projektmedlemmar men det är kvalitetssamordnaren som leder arbetet och har ansvar att utbilda och inspirera övriga medlemmar.

\subsection{Riskhantering}

För att kunna leverera en produkt som håller den kvalité som utlovats så har risker mot projektet identifierats och en riskbedömning av dessa tagits fram. Se Projektplan [KAP X.X] 
Riskbedömningen omfattar en värdering av sannolikheten av olika risker samt eventuell inverkan dessa skulle ha på projektet. En sammanslagning utav dessa två är en indikator på hur direkta hoten är mot kvalitén i den levererade produkten.
En analytisk förstudie som omfattar dessa risker mot projektet i sin helhet och hur dessa skall undvikas går att läsa i Projektplan [KAP X.X]. 

\subsection{Verktyg}

För att versionshantera kod så kommer GitLab att användas. På google drive så finns en mapp som delas med kund för att dela filer. WebStorm kommer användas som integrerad utvecklingsmiljö och Angular som webbramverk. Asana används som ett projekthanteringssystem.

Varje person ansvarar för att den är förtrogen med programvaror som används. Skulle det finnas osäkerhet så går det att fråga en annan projektmedlem. Vid behov så kan en medlem med kunskap hålla en genomgång för de andra.

\subsection{Standarder, rutiner och konventioner}

Denna sektion beskriver de standarder, rutiner och konventioner som ska följas under projektarbetet.

\subsection{Scrum}
För att få en sådan kvalitétssäkrande arbetsmetod som möjligt så valdes arbetsmetodiken Scrum.

\subsubsection*{Sprintplanering}
Projektet skall varannan vecka hålla en så kallad sprint planering. Planeringen går ut på att kravspecifikationen bryts ner i arbetsmoment som ska genomföras under nästa sprint. Arbetet följer de prioriteringar kund satt upp.

\subsubsection*{Sprintgenomgång}
Efter varje sprint hålls en granskning av sprintens resultat där varje projektmedlem redovisar resultaten från föregående sprint. 

\subsubsection{Sprintåterblick}

Det skall även hållas en s.k. sprintåterblick där gruppen analyserar kvalitén på resultat samt diskuterar hur nästkommande sprint kan genomgå förbättringar.

Scrums iterativa arbetssätt ger projektet möjlighet att utvärdera och korrigera både produkten och processer utefter de önskemål som kunden framfört. Förhinder att uppnå den kvalité som utsatts kan snabbt upptäckas och hanteras. Referera till avsnitt Quality measurement - Kvalitetsmätning.

\subsection{Pappersprototyper}
Vid tidig design av gränssnitt till produkten ska prototyper göras för att kunna få feedback från både gruppen och kunden. Prototyperna ska göras på papper eftersom detta är ett snabbt och effektivt sätt som inte kräver kunskap om något datorverktyg.

\subsection{Dokumentstandard}
För att alla dokument som skrivs i projektet ska ha ett konsekvent utseende och stil ska dokumentet Dokumentstandard skrivas och följas. Till denna ska det också finnas dokumentmallar som alla dokument ska bygga på.

\section{Prestationer, resurser och planering}
För mjukvaruresurser se detta dokument, verktyg. 
Den uppskattade tiden och andra resurser i projektet finns beskrivet under punkten Resurser i projektplanen. 

\subsection{Aktiviteter, utfall och uppgifter}
Denna sektion behandlar aktiviteter, utfall och uppgifter för kvalitetssäkring av produkten och processen. 
\subsubsection*{Kvalitetssäkring av produkt}
Syftet med kvalitetssäkring av produkten är att skapa ett förtroende för kvaliteten på produkten. Detta innefattar all mjukvara samt alla dokument som skrivs under projektet. I denna sektion definieras de aktiviteter och uppgifter som krävs för att uppnå detta utfall.
\subsubsection*{Utvärdering av planering}
För att på ett så entydigt sätt som möjligt klarlägga vad projektet har för standard så har en kravspecifikation utformats tillsammans med kund [se kravspec]. 
Denna standard definierar projektets eftersträvade kvalitet enligt överensstämmelse med kund.
Kravspecifikationen är en grundpelare för den plan som utarbetats för projektet. [se projektplan]. Kraven representerar kundens behov, önskemål och förväntningar. 
\subsubsection*{Evaluate product for acceptability}
Om berörda projektmedlemmar samtycker att ett delsystem är i ett sådant stadie att de kan testas för integration i systemet så skall testfall skrivas för delsystemet. När dessa delsystem godkänts i projektets testprotokoll [se testplan kap X.X] så integreras delsystemet i projektet.
\subsubsection*{Utvärdering av underhåll för produkten}
I projektet inkluderas inte något underhåll av produkten eftersom det är en prototyp som kunden själva får vidareutveckla vid behov. Därför definieras inga aktiviteter eller uppgifter för detta.
\subsubsection{Hur produktens kvalite ska kunna mätas?}




\end{document}
