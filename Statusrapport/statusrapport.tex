\documentclass[a4paper,10pt]{article}
\usepackage[margin=1.4in]{geometry}
\usepackage[swedish]{babel}
\usepackage[utf8]{inputenc}
\usepackage[T1]{fontenc}
\usepackage{titlesec}
\usepackage{titling}
\usepackage{todonotes}

\usepackage{graphicx}
\usepackage{fancyhdr}

\pagestyle{fancy}
\rhead{\includegraphics[width=1cm]{../Templates/Aeon}}
\lhead{}

\setlength{\parskip}{1em}
\setlength{\parindent}{0pt}
\titlespacing{\section}{0pt}{\parskip}{-\parskip}
\titlespacing{\subsection}{0pt}{\parskip}{-\parskip}
\titlespacing{\subsubsection}{0pt}{\parskip}{-\parskip}
\titlespacing{\part}{0pt}{\parskip}{-\parskip}


\title{ Statusrapport iteration 2 }
\author{Grupp 6}


%%%%%%%%%%%%%%%%%%%%%%%%%%%%%%%%%%%%%%%%%%%%%%%%%%%%%%%%%%%%%%%%%%%%%%%%%%%%%%%%
%Syfte med dokumentet är att svara på följande fråga:
%   Har ni nått målen med förstudien? Vad fattas? Ge en kort motivering.
%   En typisk förstudie omfattar följande:
%       1. Omfattning -- Yes
%       2. Avgränsningar -- Yes
%       3. Nulägesanalys -- Nja ingår typlite men har utförts av östergötland
%       4. Intresseanalys -- Nja behövdes ej ty projekt efterfrågades
%       5. Kravspecifikation -- Yes
%       6. Lösningsförslag --
%       7. Lönsamhetsanalys -- Nja ingår typlite men har utförts av östergötland
%       8. Milstolpsplan -- Yes
%   Vilka av dessa punkter kan vi berört hur gick det, har vi missat någon
%   varför.
%%%%%%%%%%%%%%%%%%%%%%%%%%%%%%%%%%%%%%%%%%%%%%%%%%%%%%%%%%%%%%%%%%%%%%%%%%%%%%%%


\begin{document}
\begin{titlingpage}
    \maketitle
    \begin{abstract}
    \noindent Vi har under förstudien i iteration två färdigställt ett antal dokument för att definiera vårt projekt. I dessa dokument så behandlas ett flertal olika frågeställningar såsom bakgrund, syfte och mål med projektet. Det finns även dokument där tidsplanen, kraven och milstolparna för projekt har formulerats. Utöver att sammanställa dokumentation så har gruppen haft flera utbildningar som har berört bland annat arbetsmetodik och teknisk kompetens.
    \end{abstract}
\end{titlingpage}

\section{Dokumentation}
\label{sec:Dokumentation}
De dokument som har sammanställts under iteration två är följande:

\textbf{Projektplan:} Det här dokument innehåller en beskrivning av hur projektet ska genomföras samt varför det ska genomföras.

\textbf{Kravspecifikation:} Det här dokumentet behandlar alla krav som är relevant på produkten som projektet ska resultera i. Dessa krav berör många aspekter bland annat krav på arkitekturen, funktionaliteten och begränsningar.

\textbf{Kvalitetsplan:} Den här planen innehåller en form av \emph{hypotes} på hur arbetet i gruppen ska fungera.

\textbf{Systematanomi:} Denna anatomi ger en grafisk bild av systemet som ska utvecklas i projektet.

\section{Utbildning och forskning}
\label{sec:Utbildning och forskning}
Under iteration 2 har gruppen tagit del av diverse utbildningar för att förbereda inför kommande iterationer.

\textbf{Git:} Gruppen har haft en genomgång av git som versionshanteringsverktyg samt featurebranch-workflow. Syftet med den här genomgången var att ge gruppen en introduktion till git, hur den fungera samt hur gruppen kommer arbeta med det under projektet.

\textbf{Scrum:} Gruppen har haft en genomgång av arbetsmetodiken scrum som används i projektet. Syftet med den här genomgången var att introducera scrum och hur gruppen kommer att använda sig av det under kursens gång.

\textbf{Seminarium och föreläsningsserie i kursen TDDD96:} Gruppen har även tagit del av den officiella utbildningen som erbjuds i kursen TDDD96. Ämnen som här berörts inkluderar:
\begin{itemize}
    \item Hållbar utveckling
    \item Projekt arbetet
    \item Dokument skrivning
    \item Sälja en idé
    \item Dela kunskap om design
\end{itemize}

\textbf{\LaTeX:} Under iterationen tog gruppen ett beslut om att använda LaTeX för att skriva dokument. Som ett resultat av det så har gruppens medlemmar allokerat tid för att sätta sig in LaTeX.

\textbf{Standarder:} Eftersom många av dokumenten som skrevs under iterationen följer standarder så har gruppen satt sig in i ett flertal olika standarder. Syftet med att göra det är att skaffa kunskap till konstruktionen av mallar för dokumenten som beskrevs i sektion \ref{sec:Dokumentation}.


\end{document}
