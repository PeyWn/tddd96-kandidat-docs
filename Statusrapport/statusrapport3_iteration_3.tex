\documentclass[a4paper,10pt]{article}
\usepackage[margin=1.4in]{geometry}
\usepackage[swedish]{babel}
\usepackage[utf8]{inputenc}
\usepackage[T1]{fontenc}
\usepackage{titlesec}
\usepackage{titling}
\usepackage{todonotes}

\usepackage{graphicx}
\usepackage{fancyhdr}

\pagestyle{fancy}
\rhead{\includegraphics[width=1cm]{../Templates/Aeon}}
\lhead{}

\setlength{\parskip}{1em}
\setlength{\parindent}{0pt}
\titlespacing{\section}{0pt}{\parskip}{-\parskip}
\titlespacing{\subsection}{0pt}{\parskip}{-\parskip}
\titlespacing{\subsubsection}{0pt}{\parskip}{-\parskip}
\titlespacing{\part}{0pt}{\parskip}{-\parskip}

\def\ftitle{Statusrapport Iteration 3}
\def\fversion{1.0}

\begin{document}

\begin{titlepage} % Suppresses displaying the page number on the title page and the subsequent page counts as page 1
	\newcommand{\HRule}{\rule{\linewidth}{0.5mm}} % Defines a new command for horizontal lines, change thickness here

	\center % Centre everything on the page

	%------------------------------------------------
	%	Headings
	%------------------------------------------------

	\textsc{\LARGE Linköpings universitet \\ \vspace{0.2em} Institutionen för datavetenskap }\\[2cm]

    \large\today

    \vspace{1cm}


	%------------------------------------------------
	%	Title
	%------------------------------------------------

	\HRule\\[0.4cm]

	{\huge\bfseries Schemaläggningsstöd för  \vspace{.1em} \\ kirurgi  - \ftitle }\\[0.4cm] % Title of your document

	\HRule\\[1cm]

	%------------------------------------------------
	%	Author(s)
	%------------------------------------------------

	\begin{minipage}{0.7\textwidth}
			\large
            \emph{Version: \fversion}
            \vspace{1em}

            \textbf{\\Adam Andersson, Niclas Byrsten, \\Björn Hvass, Hendrik Lindström, \\Martin Persson, Christoffer Sjöbergsson, \\Tor Utterborn}
            

            \vspace{1em}

            Handledare: Jonas Wallgren

            Examinator: Kristian Sandahl
	\end{minipage}
	~

	%------------------------------------------------
	%	Logo
	%------------------------------------------------

	%\vfill\vfill
	%\includegraphics[width=0.2\textwidth]{../Templates/Aeon}\\[1cm] % Include a department/university logo - this will require the graphicx package

	%----------------------------------------------------------------------------------------

	\vfill % Push the date up 1/4 of the remaining page

\end{titlepage}


\clearpage
\begin{abstract}
\noindent Under iteration tre i kandidatprojektet så har gruppen påbörjat utbecklingen av produkten. Gruppen har även opponerat på PUM10 och fått en opponering av PUM7.
Slutligen så har gruppen kompletterat de dokument utefter den feedback som man har fått från PUM7 och handledare.
\end{abstract}
\clearpage

\section{Produkten}
\label{sec:Produkten}
Gruppen har påbörjat utvecklingen av den produkt som ska skapas. Man har delat upp sig i två subgrupper, en för front-end utveckling och en för back-end utveckling.

Gruppen som har arbetat med front-end har skapat mycket av det visuella för produkten. 

Back-end gruppen har skapat den databas som systemet kommer behöva samt det API som kommer behövas för kommunikation mellan front-end och back-end.

\section{Kund}
Vid slutet av iterationen så hade gruppen ett möte med kund för att visa den funktionalitet och det utseende som produkten hade för tillfället. 

\section{Tid}
Under denna iteration så märkte gruppen att det var mycket tid kvar att lägga på projektet för att medlemmarna skulle nå upp till 400 timmar per person. Gruppen har därför infört punkten "Tid" som stående punkt på veckomötet med handledaren. Där tar man upp hur mycket tid som är kvar för gruppen och diskuterar om det finns hinder för att få in de timmar som behövs. 
Det har också lagts till en kolumn i gruppens tidrapport-dokument där man kan se hur många timmar varje person måste lägga i snitt per vecka för att nå upp till 400 timmar.

\section{Måluppfyllnad}
Gruppen har arbetat på bra denna iteration. Målen är inte helt uppfyllda då det var tänkt att utvecklingen av produkten skulle ha kommit längre och att gruppen skulle ha lagt fler timmar. Tack vare att detta uppmärksammades i mitten av iterationen så har åtgärder tagits till så att gruppen har börjat ta ifatt de timmar som saknas och det börjar se bra ut med att få in de timmar som behövs.

\end{document}
