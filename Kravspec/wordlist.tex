\section{Ordlista}
\textbf{ICD-10: }
Statistisk klassifikation med diagnoskoder för att gruppera sjukdomar och dödsorsaker

\textbf{KVÅ-kod: }
En åtgärdskod är en kod som används för statistisk beskrivning av åtgärder i bland annat hälso- och sjukvård.

\textbf{Partigt organ (lateralitet): }
Organ som det finns två av, exempelvis njurar, armar eller ben.

\textbf{Behandlingsnummer: }
Unikt ID för patientens planerade operation och operationstillfälle.

\textbf{Medicinsk brådskandegrad: }
Inom den tid som patienten behöver bli opererad utan att riskera få ytterligare skador av sin sjukdom/skada.

\textbf{Huvudåtgärd: }
Det huvudsakliga syftet med operationen. Till exempel ta bort blindtarmen  (huvudåtgärd) och sätta en urinvägskateter (åtgärd som inte föranleder operation).

\textbf{Föberedelsetid: }
Den tid som patienten befinner sig på operationssalen och där personalen arbetar med förberedande aktiviteter så att kirurgen kan börja operera. Startar när patienten kommer in på operationssalen och slutar när kirurgen påbörjar operationen.

\textbf{Operationstid: }
Från det att kirurgen börjar operera tills operationen är klar och förbandet är lagt på såret.

\textbf{Avvecklingstid: }
Tiden från att operationen är klar tills patienten lämnat operationssalen och är iordningställd för nästa patient.

\textbf{Klinisk specialitet: }
Exempelvis ortopedi, urologi eller neurokirurgi

\textbf{MTÖ-nummer: }
Unikt nummer för en medicinteknisk produkt inom Region Östergötland.

\textbf{Elektiv: }
Planerad vård

\textbf{Öppenvård: }
Vård av patienten som inte kräver inläggning på en vårdavdelning.

\textbf{Slutenvård: }
Vård av patienten som kräver en vårdplats på sjukhus.
