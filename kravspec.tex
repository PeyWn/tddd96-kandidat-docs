
% A template for an IEEE 830-1998 SRS.
% 1/11/2000 jhrg
%
% $Id$

\documentclass{article}

\usepackage{changebar}
\usepackage{acronym}
\usepackage{gloss}
\usepackage{xspace}
\usepackage{epsfig}
\usepackage[swedish]{babel}
\usepackage[utf8]{inputenc}

% Use these on place of epsfig for regular PS figures.
% \usepackage{psfig}
% \psfigurepath{my-figs}

\newcommand{\na}{{\sc N/A}\xspace}
\newcommand{\Cpp}{\rm {\small C}\raise.5ex\hbox{\footnotesize ++}\xspace}
\newcommand{\dap}{\rm {\small DAP}\raise.5ex\hbox{\footnotesize ++}\xspace}
\newcommand{\maewesturl}{http://maewest.gso.uri.edu/\-cgi-bin/\-nph-dsp/\-htn\_sst\_decloud/\-1992/\-i92098065016.htn\_d.Z\xspace}

% Note: to get the gloosary to work, run bibtex in the *.gls.aux file,
% then latex registry, then bibtex *.gls, then latex. Also, make sure to set
% your BST and BIBINPUTS environment variables so that the BST and BIB files
% will be found.
\makegloss

% Change paragraph typesetting; eliminate indenting and add more space between
% paragraphs. 2/15/2000 jhrg
\setlength{\parindent}{0em}     % Amount of indentation
\addtolength{\parskip}{1ex}     % Vertical separation

\begin{document}

\title{Kravspecifikation för schemaläggningsstöd för kirurgi}
\author{Grupp 6}
\date{\today \\ $Revision$ }

\bibliographystyle{plain}

\maketitle
\tableofcontents

%%%%%%%%%%%%%%%%%%%%%%%%%%%%%%%  Introdunction  %%%%%%%%%%%%%%%%%%%%%%%%%%%%

\section{Introduktion}
\emph{[What is this document about, in very general terms.]}

This document conforms to the IEEE~830-1998 \ac{SRS} recommended practice.
Since the recommended practice covers a wide range of possible projects, some
of the information in it is not appropriate for this part of \acs{DODS}.
Where that is the case, or I think it is the case, I have marked the section
N/A.

\textbf{Bold face} type is used to indicate a word or phrase that may be
found in the glossary.

\emph{Emphasized} text contained in square brackets ([]) is used to indicate
an editorial comment about the information that should be provided in a part
of the \ac{SRS}.

\subsection{Syfte}
\emph{[Who is this document for?]}

\subsection{Område}
\emph{[What part of the project does this SRS cover.]}

\subsection{Översikt}

The remainder of this document is organized as follows:
\begin{enumerate}
\item Section~\ref{sec:overall} provides background for the specific
requirements and relates those requirements to the rest of DODS.
\item Section~\ref{sec:specific} lists the specific requirements for the
  Cache.
\item Following Section~\ref{sec:specific} are a list of acronyms and
  abbreviations, a change log, a glossary and references.
\end{enumerate}

%%%%%%%%%%%%%%%%%%%%%%%%%  Overall Description  %%%%%%%%%%%%%%%%%%%%%%%%%%%%%

\section{Övergripande beskrivning}
\label{sec:overall}

\emph{[This section of the SRS should describe the general factors that
  affect the product and its requirements. This section should not state
  specific requirements. Instead, it provides a background for those
  requirements, which should be defined in Section 3.]}

\subsection{Produktperspektiv}
\label{subsec:Produktperspektiv}
\subsubsection{Systemgränssnitt}
\label{subsec:Systemgranssnitt}
\subsubsection{Användargränssnitt}
\label{subsec:Anvandargranssnitt}
\subsubsection{Hårdvarugränssnitt}
\label{subsec:Hardvarugranssnitt}
\subsubsection{Mjukvarugränssnitt}
\label{subsec:Mjukvarugranssnitt}
\subsubsection{Kommunikationsgränssnitt}
\label{subsec:Kommunikationsgranssnitt}
\subsubsection{Minne}
\label{subsec:Minne}
\subsubsection{Opperationer}
\label{subsec:Opperationer}
\subsubsection{Platsanpassningskrav}
\label{subsec:Platsanpassningskrav}
\subsection{Produktfunktioner}
\label{subsec:Produktfunktioner}
\subsection{Andvändarkarateristik}
\label{subsec:Andvandarkarateristik}

\subsection{Begränsningar}
\label{subsec:Begransningar}
\emph{[This section lists any other items that may limit the developer's
  options.]}


\subsubsection{Regelverk}
\label{subsec:Regelverk}
\subsubsection{Hårdvarubegränsningar}
\label{subsec:Hardvarubegransningar}
\subsubsection{Gränssnitt till andra system}
\label{subsec:Granssnitt till andra system}
\subsubsection{Paralella operationer}
\label{subsec:Paralella operationer}
\subsubsection{Kontroll funktioner}
\label{subsec:Kontroll funktioner}
\subsubsection{Styr funktioner}
\label{subsec:Styr funktioner}
\subsubsection{Högnivåspråkkrav}
\label{subsec:Hognivasprakkrav}
\subsubsection{Handskaknings protokoll}
\label{subsec:Handskaknings protokoll}
\subsubsection{Pålitlighetskrav}
\label{subsec:Palitlighetskrav}
\subsubsection{Applikationens kritikalitiet}
\label{subsection:Applikationens kritikalitiet}
\subsubsection{Säkerhetshänsysnstagande}
\label{subsec:Sakerhetshansysnstagande}

\subsection{Antagande och beroende}
\label{subsec:Antagande och beroende}
%%%%%%%%%%%%%%%%%%%%%%%%%%%  Specific Requirements %%%%%%%%%%%%%%%%%%%%%%%%%

\section{Specifika krav}
\label{sec:Specifika krav}
\emph{[This section of the \ac{SRS} lists all of the software requirements to
  a level of detail sufficient to enable designers to design a system to
  satisfy those requirements, and testers to test that the systems satisfies
  those requirements. Each separate requirment should be uniquely numbered.]}

\subsection{Externa Gränssnitt}
\label{subsec:Externa Granssnitt}
\subsubsection{Användargränssnitt}
\label{subsec:EG-Anvandargranssnitt}
\subsubsection{Hårdvarugränssnitt}
\label{subsec:EG-Hardvarugranssnitt}
\subsubsection{Mjukvarugränssnitt}
\label{subsec:EG-Mjukvarugranssnitt}
\subsubsection{Kommunikationsgränssnitt}
\label{subsec:EG-Kommunikationsgranssnitt}
\subsection{Systemfunktioner}
\label{subsec:Systemfunktioner}
\subsection{Prestandakrav}
\label{subsec:Prestandakrav}
\subsection{Designbegränsningar}
\label{subsec:Designbegransningar}
\subsection{Systemattribut}
\label{subsec:Systemattribut}
\subsection{Övriga krav}
\label{subsec:Ovriga krav}

\appendix


\section{Versionshistorik}
\label{section:Versionshistorik}
\begin{verbatim}
$Log: SRS-template.tex,v $
Revision 1.4  2000/07/20 21:35:58  jimg
Minor changes

Revision 1.3  2000/07/17 05:46:19  jimg
Added more boiler plate based on the caching SRS.

\end{verbatim}

\printgloss{dods-glossary}

\raggedright

\bibliography{dods}

\end{document}
