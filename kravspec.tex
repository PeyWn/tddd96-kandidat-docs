
% A template for an IEEE 830-1998 SRS.
% 1/11/2000 jhrg
%
% $Id$

\documentclass{article}

\usepackage{changebar}
\usepackage{acronym}
\usepackage{gloss}
\usepackage{xspace}
\usepackage{epsfig}
\usepackage[swedish]{babel}
\usepackage[utf8]{inputenc}

% Use these on place of epsfig for regular PS figures.
% \usepackage{psfig}
% \psfigurepath{my-figs}

\newcommand{\na}{{\sc N/A}\xspace}
\newcommand{\Cpp}{\rm {\small C}\raise.5ex\hbox{\footnotesize ++}\xspace}
\newcommand{\dap}{\rm {\small DAP}\raise.5ex\hbox{\footnotesize ++}\xspace}
\newcommand{\maewesturl}{http://maewest.gso.uri.edu/\-cgi-bin/\-nph-dsp/\-htn\_sst\_decloud/\-1992/\-i92098065016.htn\_d.Z\xspace}

% Note: to get the gloosary to work, run bibtex in the *.gls.aux file,
% then latex registry, then bibtex *.gls, then latex. Also, make sure to set
% your BST and BIBINPUTS environment variables so that the BST and BIB files
% will be found.
\makegloss

% Change paragraph typesetting; eliminate indenting and add more space between
% paragraphs. 2/15/2000 jhrg
\setlength{\parindent}{0em}     % Amount of indentation
\addtolength{\parskip}{1ex}     % Vertical separation

\begin{document}

\title{Kravspecifikation för schemaläggningsstöd för kirurgi}
\author{Grupp 6}
\date{\today \\ $Revision$ }

\bibliographystyle{plain}

\maketitle
\tableofcontents

%%%%%%%%%%%%%%%%%%%%%%%%%%%%%%%  Introdunction  %%%%%%%%%%%%%%%%%%%%%%%%%%%%

\section{Introduktion}
\emph{[What is this document about, in very general terms.]}

Detta dokument avser beskriva alla de krav som finns för det schemaläggningsstöd
som ska utvecklas i kursen TDDD96. Projektet utvecklas med avdelningen för test
och innovation vid region östergötland.

\emph{Emphasized} text contained in square brackets ([]) is used to indicate
an editorial comment about the information that should be provided in a part
of the \ac{SRS}.

\subsection{Syfte}
\emph{[Who is this document for?]}
Syftet med dokumentet är att ligga till grund för en gemensam kravbild på
projektet mellan projektgruppen och kunden. Kraven ska vara kompletta och
prioritetsordnade för att projektgruppen ska kunna utveckla projektet till den
standard som önskas och med enkelhet kunna göra rätt prioriteringar när val görs.

\subsection{Område}
\emph{[What part of the project does this SRS cover.]}
Kravspecifikationen omfattar alla delar i den mjukvara som ska utvecklas, såväl
server som klient.

\subsection{Översikt}
Strukturen i det här dokument är som följer:

\begin{enumerate}
\item Del~\ref{sec:overall} beskriver det större sammanhanget för de olika
kraven.
\item Del~\ref{sec:Specifika krav} Ger en deltaljerad förklaring till alla krav.
\item Efter del~\ref{sec:Specifika krav} finns en lista av förkortnigar, en lista
med versionshistorik samt en ordlista och referenser.

\end{"enumerate}

%%%%%%%%%%%%%%%%%%%%%%%%%  Overall Description  %%%%%%%%%%%%%%%%%%%%%%%%%%%%%

\section{Övergripande beskrivning}
\label{sec:overall}

\emph{[This section of the SRS should describe the general factors that
  affect the product and its requirements. This section should not state
  specific requirements. Instead, it provides a background for those
  requirements, which should be defined in Section 3.]}

\subsection{Produktperspektiv}
\label{subsec:Produktperspektiv}
Systemet avser agera som en prototyp och förstudie till en del av ett större
stystem som kunden håller på att bygga.
Mjukvaran kommer därför inte att interagera med andra system. Detta innebär att all
data kommer att leva i systemet. Denna data kommer att vara påhittad men likna
den data som i verkligheten skulle kunna finnas.
I största möjliga mån kommer indata att vara statisk, så som vilka resurser som
finns däremot kommer det schema, som produkten ska skapa, att vara dynamiskt.
\subsubsection{Systemgränssnitt}
\label{subsec:Systemgranssnitt}

\subsubsection{Användargränssnitt}
\label{subsec:Anvandargranssnitt}
Systemet ska vara webbaserat och gå att interager med i de webbläsare som
används på slutanvändarens dator (Explorer 11?).
\subsubsection{Hårdvarugränssnitt}
\label{subsec:Hardvarugranssnitt}
SERVER? DATOR FÖR ANV?
\subsubsection{Mjukvarugränssnitt}
\label{subsec:Mjukvarugranssnitt}
Klienten i systemet ska gå att använda i webbläsaren på användarens dator.
Datorerna på sjukhuset där systemet kommer användas administreras av en
IT-avdelning. Det är IT-avdelningen som beslutar om vilken webbläsare som ska
användas detta innebär att systemet bara behöver stödja den webbläsaren fullt
ut.

VILKA ANDRA SYSTEM KOMMER ATT ANVÄDAS?
\subsubsection{Kommunikationsgränssnitt}
\label{subsec:Kommunikationsgranssnitt}
\subsubsection{Minne}
\label{subsec:Minne}
\subsubsection{Operationer}
\label{subsec:Operationer}
\textbf{Logga in i systemet}\\
Det ska gå att logga in i systemet med användarnamn och lösenord. Denna
funktionen ska finnas inte primärt av säkerhetskäl utan mest för att alla
aktiviteter utförda i systemet ska kunna spåras till en användare.
\textbf{Översikt av beslutade Operationer}\\
Systemet ska kunna ge en översikt över alla operationer som är beslutade.
Det ska gå att se information om både operationen och patienten. Det är också
viktigt att se hur akut operationen är. Översikten ska visa den viktigaste
informationen men även göra det möjligt att se all detaljerad information om ett
specifikt beslut.\\
\textbf{Planeringsvy}\\
Det ska vara möjligt att se alla detaljer som finns i planeringsunderlaget.
Det bör även gå att ändra parametrar i sökningen så som tidsåtgång.\\
För att kunna samverka om gemensamma resurser måste det finns metoder att boka
och se tillgänglighet för ressureser som är delade eller helt tillhör en annan
avdelning.\\
\textbf{Ingreppsspecifika förberdelser (hitta och boka tid)}\\

\textbf{Patientspecifika förberdelser}\\
\textbf{Schemaöversikt}\\
Schemaöversikten är det huvudsakliga verktyget för att visualisera lediga tider.
Översikten behöver en mängd olika möjligheter att filtrera och anpassa vilken
data som syns. Detta gäller tillexempel att kunna se alla lediga tider för en
kirurg. Eller se alla lediga tider som passar för en beslutad operation.\\


\subsubsection{Platsanpassningskrav}
\label{subsec:Platsanpassningskrav}
\subsection{Produktfunktioner}
\label{subsec:Produktfunktioner}
Produktens huvudfunktion ska vara att visualisera de lediga tider för en vald
operation. Med en ledig tid avses en tid då alla ressureser som krävs för den
operationen är ledig, så som personal med rätt kunskap samt lokaler och verktyg.
Vid sökning av en lämplig tid är det viktigt att mjukvaran inte hindrar
användaren från att använda sin egen kunskap, det ska därför vara möjligt att
ändra relevanta parametrar för sökningen.

Utöver detta ska det även vara möjligt att se alla beslutatde operationer som
behöver planeras. Det ska också finnas en vy där alla planerade operationer
visas. Båda dessa vyer ska ha möjligheter att filtrera och sortera innehållet
efter olika parametrar.
\subsection{Andvändarkarateristik}
\label{subsec:Andvandarkarateristik}
Produkten som ska utvecklas har på sikt flera olika användare. Då detta
projekt är avsett att vara en prototyp för framtida system så riktas gränssnittet
främst mot en användare som vi kallar plannerarren. Planerraren har god
erfraenhet av olika operationer och komplexiteten som de har. I tidigare system
har planerraren fattat beslut om operationstider genom att sammla information
från flera olika sytem och källor. Användaren har vana i att arbeta i sjukhusets
andra system och kan antas ha grundläggande datorvana.

\subsection{Begränsningar}
\label{subsec:Begransningar}
\emph{[This section lists any other items that may limit the developer's
  options.]}


\subsubsection{Regelverk}
\label{subsec:Regelverk}
Då ingen riktig patientdata kommer finnas i systemet så är det inte nödvändigt
att ta hänsyn till specifika regelverk
\subsubsection{Hårdvarubegränsningar}
\label{subsec:Hardvarubegransningar}
\subsubsection{Gränssnitt till andra system}
\label{subsec:Granssnitt till andra system}
Inga interatkioner med andra system kommer vara aktuella under detta projekt.
\subsubsection{Paralella operationer}
\label{subsec:Paralella operationer}
\subsubsection{Kontrollfunktioner}
\label{subsec:Kontrollfunktioner}
\subsubsection{Styrfunktioner}
\label{subsec:Styrfunktioner}
\subsubsection{Högnivåspråkkrav}
\label{subsec:Hognivasprakkrav}
Systemet ska skrivas i typescript. Klienten ska använd sig av Angular.
\subsubsection{Pålitlighetskrav}
\label{subsec:Palitlighetskrav}
\subsubsection{Applikationens kritikalitiet}
\label{subsec:Applikationens kritikalitiet}
\subsubsection{Säkerhetshänsysnstagande}
\label{subsec:Sakerhetshansysnstagande}

\subsection{Antagande och beroende}
\label{subsec:Antagande och beroende}
%%%%%%%%%%%%%%%%%%%%%%%%%%%  Specific Requirements %%%%%%%%%%%%%%%%%%%%%%%%%

\section{Specifika krav}
\label{sec:Specifika krav}
\emph{[This section of the \ac{SRS} lists all of the software requirements to
  a level of detail sufficient to enable designers to design a system to
  satisfy those requirements, and testers to test that the systems satisfies
  those requirements. Each separate requirment should be uniquely numbered.]}

\subsection{Externa Gränssnitt}
\label{subsec:Externa Granssnitt}
\subsubsection{Hårdvarugränssnitt}
\label{subsec:EG-Hardvarugranssnitt}
\subsubsection{Mjukvarugränssnitt}
\label{subsec:EG-Mjukvarugranssnitt}
\subsubsection{Kommunikationsgränssnitt}
\label{subsec:EG-Kommunikationsgranssnitt}
\subsection{Funktionella krav}
\label{subsec:Funktionella krav}
\subsubsection{Beslutsöversikt}
\textbf{Ska kunna}
\begin{enumerate}
  \item Det ska finnas en sorteringsbar översikt av beslutade operationer.
  \item Det ska finnas en filtreringsfunktion i översikten av beslutade
  operationer.
  \item Det ska gå att se all information i ett beslut genom val av ett
  beslut från översikten av beslut.
  Ingående data:
  Personnummer, Efternamn, Förnamn, Diagnoskod enligt ICD-10, Angiven KVÅ-kod
  som är huvudåtgärd, Operation av parigt organ, angiven sida, Beslutande
  klinik, Planeringsstatus och Medicinsk brådskandegrad.
  \item I översikten med beslutade operationer ska medicinisk brådskandegrad
  framgå i antal dagar, timmar eller minuter det är kvar till behandling utifrån
  beslutsdatum. Senaste tiden som patienten ska bli opererad.
  \item Det ska tydligt framgå om en patient ska hanteras akut (alla åtgärder
  som ska genomföras inom 24 timmar från beslutsdatum) eller elektivt.
  \item Det ska vara tydligt vilka operationer som är färdigplanerade och vilka
  som behöver mer åtgärder.
\end{itemize}
\subsection{Systemfunktioner}
\label{subsec:Systemfunktioner}
\subsection{Prestandakrav}
\label{subsec:Prestandakrav}
\subsection{Designbegränsningar}
\label{subsec:Designbegransningar}
\subsection{Systemattribut}
\label{subsec:Systemattribut}
\subsection{Övriga krav}
\label{subsec:Ovriga krav}

\appendix


\section{Versionshistorik}
\label{section:Versionshistorik}
\begin{verbatim}
$Log: SRS-template.tex,v $
Revision 1.4  2000/07/20 21:35:58  jimg
Minor changes

Revision 1.3  2000/07/17 05:46:19  jimg
Added more boiler plate based on the caching SRS.

\end{verbatim}

\printgloss{dods-glossary}

\raggedright

\bibliography{dods}

\end{document}
