
% A template for an IEEE 830-1998 SRS.
% 1/11/2000 jhrg
%
% $Id$

\documentclass{article}

\usepackage{changebar}
\usepackage{acronym}
\usepackage{gloss}
\usepackage{xspace}
\usepackage{epsfig}
\usepackage[swedish]{babel}
\usepackage[utf8]{inputenc}

% Use these on place of epsfig for regular PS figures.
% \usepackage{psfig}
% \psfigurepath{my-figs}

\newcommand{\na}{{\sc N/A}\xspace}
\newcommand{\Cpp}{\rm {\small C}\raise.5ex\hbox{\footnotesize ++}\xspace}
\newcommand{\dap}{\rm {\small DAP}\raise.5ex\hbox{\footnotesize ++}\xspace}
\newcommand{\maewesturl}{http://maewest.gso.uri.edu/\-cgi-bin/\-nph-dsp/\-htn\_sst\_decloud/\-1992/\-i92098065016.htn\_d.Z\xspace}

% Note: to get the gloosary to work, run bibtex in the *.gls.aux file,
% then latex registry, then bibtex *.gls, then latex. Also, make sure to set
% your BST and BIBINPUTS environment variables so that the BST and BIB files
% will be found.
\makegloss

% Change paragraph typesetting; eliminate indenting and add more space between
% paragraphs. 2/15/2000 jhrg
\setlength{\parindent}{0em}     % Amount of indentation
\addtolength{\parskip}{1ex}     % Vertical separation

\begin{document}

\title{Kravspecifikation för schemaläggningsstöd för kirurgi}
\author{Grupp 6}
\date{\today \\ $Revision$ }

\bibliographystyle{plain}

\maketitle
\tableofcontents

%%%%%%%%%%%%%%%%%%%%%%%%%%%%%%%  Introdunction  %%%%%%%%%%%%%%%%%%%%%%%%%%%%

\section{Introduktion}
\emph{[What is this document about, in very general terms.]}

Detta dokument avser beskriva alla de krav som finns för det schemaläggningsstöd
som ska utvecklas i kursen TDDD96. Projektet utvecklas med avdelningen för test
och innovation vid region östergötland.

\emph{Emphasized} text contained in square brackets ([]) is used to indicate
an editorial comment about the information that should be provided in a part
of the \ac{SRS}.

\subsection{Syfte}
\emph{[Who is this document for?]}
Syftet med dokumentet är att ligga till grund för en gemensam kravbild på
projektet mellan projektgruppen och kunden. Kraven ska vara kompletta och
prioritetsordnade för att projektgruppen ska kunna utveckla projektet till den
standard som önskas och med enkelhet kunna göra rätt prioriteringar när val görs.

\subsection{Område}
\emph{[What part of the project does this SRS cover.]}
Kravspecifikationen omfattar alla delar i den mjukvara som ska utvecklas, såväl
server som klient.

\subsection{Översikt}
Strukturen i det här dokument är som följer:

\begin{enumerate}
\item Sektion~\ref{sec:overall} beskriver det större sammanhanget för de olika
kraven.
\item Sektion~\ref{sec:specific} Ger en deltaljerad förklaring till alla krav.
\item Efter sektion~\ref{sec:specific} finns en lista av förkortnigar, en lista
med versionshistorik samt en ordlista och referenser.

\end{"enumerate}

%%%%%%%%%%%%%%%%%%%%%%%%%  Overall Description  %%%%%%%%%%%%%%%%%%%%%%%%%%%%%

\section{Övergripande beskrivning}
\label{sec:overall}

\emph{[This section of the SRS should describe the general factors that
  affect the product and its requirements. This section should not state
  specific requirements. Instead, it provides a background for those
  requirements, which should be defined in Section 3.]}

\subsection{Produktperspektiv}
\label{subsec:Produktperspektiv}
\subsubsection{Systemgränssnitt}
\label{subsec:Systemgranssnitt}
\subsubsection{Användargränssnitt}
\label{subsec:Anvandargranssnitt}
\subsubsection{Hårdvarugränssnitt}
\label{subsec:Hardvarugranssnitt}
\subsubsection{Mjukvarugränssnitt}
\label{subsec:Mjukvarugranssnitt}
\subsubsection{Kommunikationsgränssnitt}
\label{subsec:Kommunikationsgranssnitt}
\subsubsection{Minne}
\label{subsec:Minne}
\subsubsection{Opperationer}
\label{subsec:Opperationer}
\subsubsection{Platsanpassningskrav}
\label{subsec:Platsanpassningskrav}
\subsection{Produktfunktioner}
\label{subsec:Produktfunktioner}
\subsection{Andvändarkarateristik}
\label{subsec:Andvandarkarateristik}

\subsection{Begränsningar}
\label{subsec:Begransningar}
\emph{[This section lists any other items that may limit the developer's
  options.]}


\subsubsection{Regelverk}
\label{subsec:Regelverk}
\subsubsection{Hårdvarubegränsningar}
\label{subsec:Hardvarubegransningar}
\subsubsection{Gränssnitt till andra system}
\label{subsec:Granssnitt till andra system}
\subsubsection{Paralella operationer}
\label{subsec:Paralella operationer}
\subsubsection{Kontroll funktioner}
\label{subsec:Kontroll funktioner}
\subsubsection{Styr funktioner}
\label{subsec:Styr funktioner}
\subsubsection{Högnivåspråkkrav}
\label{subsec:Hognivasprakkrav}
\subsubsection{Handskaknings protokoll}
\label{subsec:Handskaknings protokoll}
\subsubsection{Pålitlighetskrav}
\label{subsec:Palitlighetskrav}
\subsubsection{Applikationens kritikalitiet}
\label{subsection:Applikationens kritikalitiet}
\subsubsection{Säkerhetshänsysnstagande}
\label{subsec:Sakerhetshansysnstagande}

\subsection{Antagande och beroende}
\label{subsec:Antagande och beroende}
%%%%%%%%%%%%%%%%%%%%%%%%%%%  Specific Requirements %%%%%%%%%%%%%%%%%%%%%%%%%

\section{Specifika krav}
\label{sec:Specifika krav}
\emph{[This section of the \ac{SRS} lists all of the software requirements to
  a level of detail sufficient to enable designers to design a system to
  satisfy those requirements, and testers to test that the systems satisfies
  those requirements. Each separate requirment should be uniquely numbered.]}

\subsection{Externa Gränssnitt}
\label{subsec:Externa Granssnitt}
\subsubsection{Användargränssnitt}
\label{subsec:EG-Anvandargranssnitt}
\subsubsection{Hårdvarugränssnitt}
\label{subsec:EG-Hardvarugranssnitt}
\subsubsection{Mjukvarugränssnitt}
\label{subsec:EG-Mjukvarugranssnitt}
\subsubsection{Kommunikationsgränssnitt}
\label{subsec:EG-Kommunikationsgranssnitt}
\subsection{Systemfunktioner}
\label{subsec:Systemfunktioner}
\subsection{Prestandakrav}
\label{subsec:Prestandakrav}
\subsection{Designbegränsningar}
\label{subsec:Designbegransningar}
\subsection{Systemattribut}
\label{subsec:Systemattribut}
\subsection{Övriga krav}
\label{subsec:Ovriga krav}

\appendix


\section{Versionshistorik}
\label{section:Versionshistorik}
\begin{verbatim}
$Log: SRS-template.tex,v $
Revision 1.4  2000/07/20 21:35:58  jimg
Minor changes

Revision 1.3  2000/07/17 05:46:19  jimg
Added more boiler plate based on the caching SRS.

\end{verbatim}

\printgloss{dods-glossary}

\raggedright

\bibliography{dods}

\end{document}
