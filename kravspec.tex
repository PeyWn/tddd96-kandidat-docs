
% A template for an IEEE 830-1998 SRS.
% 1/11/2000 jhrg
%
% $Id$

\documentclass{article}
\usepackage[margin=1.4in]{geometry}
\usepackage[swedish]{babel}
\usepackage[utf8]{inputenc}
\usepackage{titlesec}
\usepackage{titling}
\usepackage{todonotes}



\setlength{\parskip}{1em}
\setlength{\parindent}{0pt}
\titlespacing{\section}{0pt}{\parskip}{-\parskip}
\titlespacing{\subsection}{0pt}{\parskip}{-\parskip}
\titlespacing{\subsubsection}{0pt}{\parskip}{-\parskip}
\titlespacing{\part}{0pt}{\parskip}{-\parskip}


\begin{document}

\title{Kravspecifikation för schemaläggningsstöd för kirurgi}
\author{Grupp 6}
\date{\today \\ $Revision$ 0.1 }

\begin{titlingpage}
    \maketitle
    \section*{
    \begin{center}
    Dokumentationshistorik
    \end{center}}
    \begin{center}
     \begin{tabular}{|c c c c |}
     \hline
     Datum & händelse & iteration & version\\
     \hline
     2018-02-16 & Dokument skapas & 1 &  0.1\\
     \hline
    \end{tabular}
    \end{center}
\end{titlingpage}
\tableofcontents
\clearpage


%%%%%%%%%%%%%%%%%%%%%%%%%%%%%%%  Introdunction  %%%%%%%%%%%%%%%%%%%%%%%%%%%%

\section{Introduktion}
\emph{[What is this document about, in very general terms.]}

Detta dokument avser beskriva alla de krav som finns för det schemaläggningsstöd
som ska utvecklas i kursen TDDD96. Projektet utvecklas med avdelningen för test
och innovation vid region östergötland.

\emph{Emphasized} text contained in square brackets ([]) is used to indicate
an editorial comment about the information that should be provided in a part
of the \ac{SRS}.

\subsection{Syfte}
\emph{[Who is this document for?]}
Syftet med dokumentet är att ligga till grund för en gemensam kravbild på
projektet mellan projektgruppen och kunden. Kraven ska vara kompletta och
prioritetsordnade för att projektgruppen ska kunna utveckla projektet till den
standard som önskas och med enkelhet kunna göra rätt prioriteringar när val görs.

\subsection{Område}
\emph{[What part of the project does this SRS cover.]}
Kravspecifikationen omfattar alla delar i den mjukvara som ska utvecklas, såväl
server som klient.

\subsection{Översikt}
Strukturen i det här dokument är som följer:

\begin{enumerate}
\item Del~\ref{sec:overall} beskriver det större sammanhanget för de olika
kraven.
\item Del~\ref{sec:Specifika krav} Ger en detaljerad förklaring till alla krav.
\item Efter del~\ref{sec:Specifika krav} finns en lista av förkortningar, en lista
med versionshistorik samt en ordlista och referenser.

\end{"enumerate}

%%%%%%%%%%%%%%%%%%%%%%%%%  Overall Description  %%%%%%%%%%%%%%%%%%%%%%%%%%%%%

\section{Övergripande beskrivning}
\label{sec:overall}

\emph{[This section of the SRS should describe the general factors that
  affect the product and its requirements. This section should not state
  specific requirements. Instead, it provides a background for those
  requirements, which should be defined in Section 3.]}

\subsection{Produktperspektiv}
\label{subsec:Produktperspektiv}
Systemet avser agera som en prototyp och förstudie till en del av ett större
stystem som kunden håller på att bygga.
Mjukvaran kommer därför inte att interagera med andra system. Detta innebär att all
data kommer att leva i systemet. Denna data kommer att vara påhittad men likna
den data som i verkligheten skulle kunna finnas.
I största möjliga mån kommer indata att vara statisk, så som vilka resurser som
finns däremot kommer det schema, som produkten ska skapa, att vara dynamiskt.
\subsubsection{Systemgränssnitt}
\label{subsec:Systemgranssnitt}

\subsubsection{Användargränssnitt}
\label{subsec:Anvandargranssnitt}
Systemet ska vara webbaserat och gå att integrera med i de webbläsare som
används på slutanvändarens dator (Explorer 11?).
\subsubsection{Hårdvarugränssnitt}
\label{subsec:Hardvarugranssnitt}
SERVER? DATOR FÖR ANV?
\subsubsection{Mjukvarugränssnitt}
\label{subsec:Mjukvarugranssnitt}
Klienten i systemet ska gå att använda i webbläsaren på användarens dator.
Datorerna på sjukhuset där systemet kommer användas administreras av en
IT-avdelning. Det är IT-avdelningen som beslutar om vilken webbläsare som ska
användas detta innebär att systemet bara behöver stödja den webbläsaren fullt
ut.

VILKA ANDRA SYSTEM KOMMER ATT ANVÄDAS?
\subsubsection{Kommunikationsgränssnitt}
\label{subsec:Kommunikationsgranssnitt}
\subsubsection{Minne}
\label{subsec:Minne}
\subsubsection{Operationer}
\label{subsec:Operationer}
\textbf{Logga in i systemet}\\
Det ska gå att logga in i systemet med användarnamn och lösenord. Denna
funktionen ska finnas inte primärt av säkerhetsskäl utan mest för att alla
aktiviteter utförda i systemet ska kunna spåras till en användare.\\
\textbf{Översikt av beslutade Operationer}\\
Systemet ska kunna ge en översikt över alla operationer som är beslutade.
Det ska gå att se information om både operationen och patienten. Det är också
viktigt att se hur akut operationen är. Översikten ska visa den viktigaste
informationen men även göra det möjligt att se all detaljerad information om ett specifikt beslut.\\
\textbf{Planeringsvy}\\
Det ska vara möjligt att se alla detaljer som finns i planeringsunderlaget.
Det bör även gå att ändra parametrar i sökningen så som tidsåtgång.\\
För att kunna samverka om gemensamma resurser måste det finns metoder att boka
och se tillgänglighet för resurser som är delade eller helt tillhör en annan
avdelning.\\
\textbf{Ingreppsspecifika förberedelser (hitta och boka tid)}\\

\textbf{Patientspecifika förberedelser}\\
\textbf{Schemaöversikt}\\
Schemaöversikten är det huvudsakliga verktyget för att visualisera lediga tider.
Översikten behöver en mängd olika möjligheter att filtrera och anpassa vilken
data som syns. Detta gäller till exempel att kunna se alla lediga tider för en
kirurg. Eller se alla lediga tider som passar för en beslutad operation.\\


\subsubsection{Platsanpassningskrav}
\label{subsec:Platsanpassningskrav}
\subsection{Produktfunktioner}
\label{subsec:Produktfunktioner}
Produktens huvudfunktion ska vara att visualisera de lediga tider för en vald
operation. Med en ledig tid avses en tid då alla resurser som krävs för den
operationen är ledig, så som personal med rätt kunskap samt lokaler och verktyg.
Vid sökning av en lämplig tid är det viktigt att mjukvaran inte hindrar
användaren från att använda sin egen kunskap, det ska därför vara möjligt att
ändra relevanta parametrar för sökningen.

Utöver detta ska det även vara möjligt att se alla beslutade operationer som
behöver planeras. Det ska också finnas en vy där alla planerade operationer
visas. Båda dessa vyer ska ha möjligheter att filtrera och sortera innehållet
efter olika parametrar.
\subsection{Andvändarkarateristik}
\label{subsec:Andvandarkarateristik}
Produkten som ska utvecklas har på sikt flera olika användare. Då detta
projekt är avsett att vara en prototyp för framtida system så riktas gränssnittet
främst mot en användare som vi kallar planeraren. Planeraren har god
erfarenhet av olika operationer och komplexiteten som de har. I tidigare system
har planeraren fattat beslut om operationstider genom att samla information
från flera olika system och källor. Användaren har vana i att arbeta i sjukhusets
andra system och kan antas ha grundläggande datorvana.

\subsection{Begränsningar}
\label{subsec:Begransningar}
\emph{[This section lists any other items that may limit the developer's
  options.]}


\subsubsection{Regelverk}
\label{subsec:Regelverk}
Då ingen riktig patientdata kommer finnas i systemet så är det inte nödvändigt
att ta hänsyn till specifika regelverk
\subsubsection{Hårdvarubegränsningar}
\label{subsec:Hardvarubegransningar}
\subsubsection{Gränssnitt till andra system}
\label{subsec:Granssnitt till andra system}
Inga interatkioner med andra system kommer vara aktuella under detta projekt.
\subsubsection{Paralella operationer}
\label{subsec:Paralella operationer}
\subsubsection{Kontrollfunktioner}
\label{subsec:Kontrollfunktioner}
\subsubsection{Styrfunktioner}
\label{subsec:Styrfunktioner}
\subsubsection{Högnivåspråkkrav}
\label{subsec:Hognivasprakkrav}
Systemet ska skrivas i typescript. Klienten ska använd sig av Angular.
\subsubsection{Pålitlighetskrav}
\label{subsec:Palitlighetskrav}
\subsubsection{Applikationens kritikalitiet}
\label{subsec:Applikationens kritikalitiet}
\subsubsection{Säkerhetshänsysnstagande}
\label{subsec:Sakerhetshansysnstagande}

\subsection{Antagande och beroende}
\label{subsec:Antagande och beroende}
%%%%%%%%%%%%%%%%%%%%%%%%%%%  Specific Requirements %%%%%%%%%%%%%%%%%%%%%%%%%

\section{Specifika krav}
\label{sec:Specifika krav}
\emph{[This section of the \ac{SRS} lists all of the software requirements to
  a level of detail sufficient to enable designers to design a system to
  satisfy those requirements, and testers to test that the systems satisfies
  those requirements. Each separate requirment should be uniquely numbered.]}

\subsection{Externa Gränssnitt}
\label{subsec:Externa Granssnitt}
\subsubsection{Hårdvarugränssnitt}
\label{subsec:EG-Hardvarugranssnitt}
\subsubsection{Mjukvarugränssnitt}
\label{subsec:EG-Mjukvarugranssnitt}
\subsubsection{Kommunikationsgränssnitt}
\label{subsec:EG-Kommunikationsgranssnitt}
\subsection{Funktionella krav}
\label{subsec:Funktionella krav}
\subsubsection{Generella krav}
\label{subsec:Generella krav}
\textbf{Ska krav: }
\begin{enumerate}
    \item Det ska krävas användarnamn och lösenord för att komma in i systemet.
\end{enumerate}

\subsubsection{Beslutsöversikt}
\textbf{Ska krav: }
\begin{enumerate}
  \item Det ska finnas en sorteringsbar översikt av beslutade operationer.
  \item Det ska finnas en filtreringsfunktion i översikten av beslutade
  operationer.
  \item Det ska gå att se all information i ett beslut genom val av ett
  beslut från översikten av beslut.
  Ingående data:
  Personnummer, Efternamn, Förnamn, Diagnoskod enligt ICD-10, Angiven KVÅ-kod
  som är huvudåtgärd, Operation av parigt organ, angiven sida, Beslutande
  klinik, Planeringsstatus och Medicinsk brådskandegrad.
  \item I översikten med beslutade operationer ska medicinisk brådskandegrad
  framgå i antal dagar, timmar eller minuter det är kvar till behandling utifrån
  beslutsdatum. Senaste tiden som patienten ska bli opererad.
  \item Det ska tydligt framgå om en patient ska hanteras akut (alla åtgärder
  som ska genomföras inom 24 timmar från beslutsdatum) eller elektivt.
  \item Det ska vara tydligt vilka operationer som är färdigplanerade och vilka
  som behöver mer åtgärder.
\end{enumerate}
\subsubsection{Planeringsvy}
\textbf{Ska krav: }
\begin{enumerate}
\item Det ska tydligt framgå i planeringsvyn vilka tider som är inom gränsen
för den medicinska brådskandegraden.
\item I planeringsvyn ska det framgå vilka dagar och tider som är lediga för
att boka in aktuell operation.
\item En ledig operationstid ska visas som ledig när tillgänglig kompetens,
material, utrustning och lokaler (operationssal, postoperativ vårdplats och
vårdplats på avdelning) är tillgängliga.
\item När man bokar operationstid ska aktuella resurser låsas till bokningen.
Detta gäller inte förbrukningsmaterial.
\item Det ska vara möjligt att boka om eller boka av (stryka) en redan bokad
operationstid.
\item För bokade operationer i planeringsvyn ska det framgå huvudåtgärd,
patient och operatör.
\item I planeringsvyn ska det finnas tre olika markeringar: Bokad operation,
ledig operationssalstid där samtliga resurser finns tillgängliga, ledig
operationssalstid där resurser saknas.
\item Det ska i planeringsvyn framgå vilka tider (för aktuellt ingrepp) där
kompetens, material eller utrustning saknas.
Exempel: Om det finns en operationssalstid ledig, men förutbestämd utrustning
är upptagen.
\item Sammantagen operationstid ska visualiseras.
\end{enumerate}
\textbf{Bör krav: }
\begin{enumerate}
\item Det ska gå att välja en eller flera kompetenser som ska vara med vid
genomförandet av operationen.
\item Det ska gå att se vilka undersökningar och prover som ska genomföras
specifikt för patienten innan operation. Exempel: Utredningar eller prover
beställda av anestesiolog.
\item Det ska tydligt framgå vilka förberedelser som ska göras specifikt för
ingreppet, när de ska göras och var. T.ex. blodprover ska tas på avdelningen
när patienten kommer till sjukhuset.
\item Det ska tydligt framgå i vilken ordning förberedelserna ska genomföras.
\item Det ska vara möjligt att ange om patienten ska hanteras inom öppenvård
(dagkirurgi) eller slutenvård. Det ska vara möjligt att ange om patienten ska
hanteras inom öppenvård (dagkirurgi) eller slutenvård.
\item Det ska vara möjligt att i ett planeringsunderlag inkludera andra, för
patienten, beslutade kirurgiska åtgärder inom egen klinik eller annan klinik
inom RÖ.
\item Det ska tydligt i ett planeringsunderlag framgå vilka KVÅ-koder som ingår
i samverkan.
\item Det ska tydligt i ett planeringsunderlag framgå vilka ICD10-koder som
ingår i samverkan.
\item Den mest brådskande medicinska brådskandegraden ska gälla vid samverkan.
\item Tidsåtgången för varje huvudåtgärd ska visualiseras.
\item Vid samverkan ska operationstiden för de två beslutade åtgärderna slås
ihop, förberedelsetid och avvecklingstid ska baseras på den åtgärd som har
längst tid.
\item Finns det andra enheter (ej operationsavdelning eller -klinik) som ska
medverka under operationen så ska det gå att ange vilka det är.
Exempel: Röntgen, patologi.
\item Det ska tydligt i ett planeringsunderlag framgå vilka kliniker som är
involverade i aktuellt genomförande av operationen.
\item Det ska tydligt framgå vad som ska utföras av varje enskild deltagande
klinik.
\end{enumerate}
\textbf{Om tid finns krav:}
\begin{enumerate}
  \item Det ska vara lätt att se närmast lediga operationstid utifrån aktuella
  åtgärder och lediga resurser.
  \item Det ska vara möjligt att registrera patientspecifika implantat och
  beställningsdatum som ska användas till planerad operation.
  \item Om patienten har en planerad inläggning ska det vara möjligt att ange
  tid, datum och plats.
  \item Det ska gå att ange hur patienten kommer till aktuell
  operationsavdelning. Alternativ: Direkt hemifrån, via mottagning, avdelning.
\end{enumerate}
\subsubsection{Schemavy}
\textbf{Ska krav: }
\begin{enumerate}
\item Det ska finnas en kalendervy som innehåller alla planerade operationer.
\end{enumerate}
\textbf{Om tid finns krav:}
\begin{enumerate}
\item Det ska finnas en filterfunktion i kalendervyn för salar, kompetenser, utrustning m.m. T ex kunna se lediga operationstider för en enskild sal.
\end{enumerate}
\subsection{Systemfunktioner}
\label{subsec:Systemfunktioner}
\subsection{Prestandakrav}
\label{subsec:Prestandakrav}
\subsection{Designbegränsningar}
\label{subsec:Designbegransningar}
\subsection{Systemattribut}
\label{subsec:Systemattribut}
\subsection{Övriga krav}
\label{subsec:Ovriga krav}

\section{Ordlista}
\textbf{ICD-10: }
Statistisk klassifikation med diagnoskoder för att gruppera sjukdomar och dödsorsaker

\textbf{KVÅ-kod: }
En åtgärdskod är en kod som används för statistisk beskrivning av åtgärder i bland annat hälso- och sjukvård.

\textbf{Partigt organ (lateralitet): }
Organ som det finns två av, exempelvis njurar, armar eller ben.

\textbf{Behandlingsnummer: }
Unikt ID för patientens planerade operation och operationstillfälle.

\textbf{Medicinsk brådskandegrad: }
Inom den tid som patienten behöver bli opererad utan att riskera få ytterligare skador av sin sjukdom/skada.

\textbf{Huvudåtgärd: }
Det huvudsakliga syftet med operationen. Till exempel ta bort blindtarmen  (huvudåtgärd) och sätta en urinvägskateter (åtgärd som inte föranleder operation).

\textbf{Föberedelsetid: }
Den tid som patienten befinner sig på operationssalen och där personalen arbetar med förberedande aktiviteter så att kirurgen kan börja operera. Startar när patienten kommer in på operationssalen och slutar när kirurgen påbörjar operationen.

\textbf{Operationstid: }
Från det att kirurgen börjar operera tills operationen är klar och förbandet är lagt på såret.

\textbf{Avvecklingstid: }
Tiden från att operationen är klar tills patienten lämnat operationssalen och är iordningställd för nästa patient.

\textbf{Klinisk specialitet: }
Exempelvis ortopedi, urologi eller neurokirurgi

\textbf{MTÖ-nummer: }
Unikt nummer för en medicinteknisk produkt inom Region Östergötland.

\textbf{Elektiv: }
Planerad vård

\textbf{Öppenvård: }
Vård av patienten som inte kräver inläggning på en vårdavdelning.

\textbf{Slutenvård: }
Vård av patienten som kräver en vårdplats på sjukhus.


\appendix

\section{Ingående data till systemet}
\label{sec:Ingaende data till systemet}

\textbf{Beslut}
\begin{itemize}
    \item Personnummer
    \item Efternamn, Förnamn
    \item Diagnoskod enligt ICD-10
    \item Åtgärdskod enligt KVÅ
    \item Angiven KVÅ-kod som är huvudåtgärd
    \item Operation av parigt organ (lateralitet), angiven sida
    \item Datum/Tid för beslut om kirurgisk åtgärd
    \item Namn på beslutsfattare inkl. titel
    \item Beslutande klinik
    \item Medicinsk brådskandegrad
    \item Behandlingsnummer
    \item Akut eller elektiv operation
\end{itemize}

\textbf{Operationstider}
\begin{itemize}
    \item Huvudåtgärder (KVÅ-koder) med operationstider. Förberedelsetid, operationstid och avvecklingstid.
\end{itemize}

\textbf{Kompetenser}
\begin{itemize}
    \item Titel
    \item Efternamn, Förnamn
    \item Kliniktillhörighet
    \item Klinisk specialitet
    \item Kompetens (KVÅ-kod, huvudåtgärd)
    \item Schema
\end{itemize}

\textbf{Lokal}
\begin{itemize}
    \item Namn på lokal
    \item Typ av lokal (operationssal, preoperativ sal, postoperativ, vårdavdelning, intensivvårdsavdelning)
    \item Tillgänglig salstid
    \item Möjliga operationer att genomföra i lokalen (KVÅ-kod, huvudåtgärd)
    \item Ingreppsbeskrivning (KVÅ-kod, huvudåtgärd):
\end{itemize}


\texrbf{Material}
\begin{itemize}
    \item Artikelnamn
    \item Artikelnummer
    \item Typ av material
    \item Reserverade tider för flergångsmaterial (tidsåtgång: operation + sterilisering)
\end{itemize}

\textbf{Utrustning}
\begin{itemize}
    \item Namn
    \item MTÖ-nummer
    \item Typ av utrustning
    \item Reserverade tider
\end{itemize}


\printgloss{dods-glossary}

\raggedright

\bibliography{dods}

\end{document}
