\chapter{Intervjuer om val av ramverk}\label{appendix:martin}

\section{Intervjufråga 1}
\textbf{Har du några tidigare erfarenheter av att utveckla webbapplikationer?}
\begin{enumerate}
\item Nej.
\item Ja.
\item Nej.
\item Nej.
\item Ja.
\end{enumerate}

\section{Intervjufråga 2}
\textbf{Var det från början svårt att sätta sig in i hur Angular fungerar?}
\begin{enumerate}
\item Ja, det var en brant inlärningskurva. 
\item Jag har bara hoppat in i kod och lärt mig medan jag kodat. Mindre grejer kunde utföras med övrig programmering, Det funkade okej.
\item Ja, det tog ett tag. Jag hade ingen erfarenhet med HTML/css, mycket att lära sig. Det var en annorlunda struktur.
\item Ja, allt funkade men man fattar inte varför, mycket magi. Svårt att felsöka då man inte fattar. Lätt att använda om man följer guiden. Svårt att hitta på egna lösningar, behövde googla mycket.
\item Ja. Mycket hände bakom kulisserna. Svårt i början, när man lärt sig strukturen var det enklare.
\end{enumerate}

\section{Intervjufråga 3}
\textbf{Vad har fungerat bra med Angular under projektets gång?}
\begin{enumerate}
\item Sättet att strukturera koden i komponenter och hur de kommunicerar. Passar ihop bra med hur applikationen grafiskt ser ut vilket gör det lätt att dela in i komponenter.
\item Vi har fått advancerade komponent snabbt, och haft lätt att fixa moduler. Snabbt att sätta upp.
\item Komponentstrukturen har fungerat bra.
\item Komponentuppdelning  är bra, särskilt för teamarbete. Modularitet är bra. Mycket färdiga funktioner som funkar på direkten.
\item Enkelt att återanvända komponenter. Bra med förslag från webstorm, hade varit jobbigt annars.
\end{enumerate}

\section{Intervjufråga 4}
\textbf{Vad har fungerat dåligt med Angular under projektets gång?}
\begin{enumerate}
\item Det är svårt att göra rätt när man inte är van vid webbutveckling. Man mIssar förståelse för hur Angular är tänkt att användas. Det är svårt att komma in i tänket.
\item Det är svårt att veta vad som ska ingå i moduler och hur stora modulerna bör vara.
\item Filstrukturen var svår att navigera genom.
\item Filstrukturen är en röra. Hierarkin bli väldigt djup och rörig. Oklart hur Angular tänkt att man ska lösa saker när det finns fler alternativ.
\item När man inte är kunnig är koden svår att läsa eftersom mycket sker bakom kulisserna. Efter ett tag är det inte så svårt. Brant inlärningskurva.
\end{enumerate}

\section{Intervjufråga 5}
\textbf{Vilken funktionalitet är den du haft mest nytta av?}
\begin{enumerate}
\item Strukturen med komponenter, kommunikationen i allmänhet. Det specifika htmlen för angular är kraftfull och lätt att förstå.
\item Ärvda moduler.
\item Komponenterna.
\item Dynamisk HTML var kraftfullt. Man slipper även bygga om klienten när man skrivit ny kod.
\item Komponentstrukturen. Modulariteten.
\end{enumerate}

\section{Intervjufråga 6}
\textbf{Vilket är ditt samlade omdöme om hur Angular fungerat under projektet?}
\begin{enumerate}
\item Ett mer strukturerat förberedande moment hade behövts. Extremt effektiv verktyg när man är insatt, men jag har inte mycket att jämföra med. Jag hade stor nytta av tidigare programmeringserfarenheter.
\item Svårt att sätta sig in i, men har gett bra resultat.
\item Svårt att sätta sig in i, men effektivt när man väl gjort det.
\item Har fungerat bra på det stora hela. Mycket gratis som bara funkar. Modularitet är bra.
\item Svårt att sätta sig in i, men bra när man bemästrat det är det en väldig tillgång.
\end{enumerate}

\section{Intervjufråga 7}
\textbf{Hade det lönat sig att utföra en förstudie om val av ramverk?}
\begin{enumerate}
\item Det kunde ha lönat sig att leta efter ett enklare verktyg. Förstudie kunde ha varit värt tiden.
\item Vi hade inte tjänat på att undersöka fler alternativ.
\item Det funkade bra att ta kundens förslag rakt av.
\item Jag tyckte att det gick bra att komma igång och har inget emot valprocessen vi använde.
\item Vi sparade tid genom att välja kundens alternativ.
\end{enumerate}