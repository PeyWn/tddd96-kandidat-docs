\chapter{Intervjuer om TypeScript och JavaScript}\label{appendix:henrik}

\section{Intervjufrågor}
Följande tabell presenterar de frågor som ställdes under intervjuundersökningen om TypeScript och JavaScript.
\begin{table}[H]
\centering
\caption{Intervjufrågor}
\label{henrik_fragor}
\begin{tabular}{|c|l|}
\hline
Nr & \multicolumn{1}{c|}{Fråga}                                                                                                                     \\ \hline
1  & \begin{tabular}[c]{@{}l@{}}Hade du någon erfarenhet med JavaScript eller TypeScript\\ innan projektet?\end{tabular}                           \\ \hline
2  & \begin{tabular}[c]{@{}l@{}}Fick du använda TypeScript eller JavaScript under utvecklingen\\ i projektet?\end{tabular}                          \\ \hline
3  & \begin{tabular}[c]{@{}l@{}}Hjälpte typningen i TypeScript dig att hitta fel under skrivning\\ eller byggning av webbklienten?\end{tabular}     \\ \hline
4  & \begin{tabular}[c]{@{}l@{}}Tror du att fel hade gått odetekterade utan typkontroll, alltså om\\ vi skrivit i JavaScript istället?\end{tabular} \\ \hline
5  & Hjälpte klasser och typer dig att förstå kod skriven av någon annan?                                                                           \\ \hline
6  & \begin{tabular}[c]{@{}l@{}}Påverkade klasser och typning din produktivitet och\\ utvecklingsfart under projektet?\end{tabular}                 \\ \hline
7  & \begin{tabular}[c]{@{}l@{}}Hjälpte klasser och typning dig att underhålla och ändra befintlig\\ kod?\end{tabular}                              \\ \hline
8  & \begin{tabular}[c]{@{}l@{}}Har du upplevt några övriga för- eller nackdelar med TypeScript\\ jämfört med JavaScript?\end{tabular}              \\ \hline
9  & \begin{tabular}[c]{@{}l@{}}Har du några övergripande tankar om hur användandet av\\ TypeScript fungerade i projektet?\end{tabular}             \\ \hline
\end{tabular}
\end{table}

\section{Svar på intervjuer}\label{henrik_svar}
I denna sektion sammanfattas de svar till frågorna i tabell \ref{henrik_fragor} som erhölls från intervjun.

\subsection{Gruppmedlem 1}
\begin{enumerate}
\item Ingen alls med något av dem.
\item Mycket TypeScript och en del JavaScript.
\item Nej, har koll på egen kod och även andras genom bra dialog med dem. Kan dock ske kommunikationsfel.
\item Troligen att några mindre fel inte upptäckts direkt.
\item Det ökar läsbarhet för koden, speciellt att veta returvärden och vilka medlemmar som finns i klasser.
\item Det extra skrivandet kompenseras av de problem som kringgås tack vare statisk typning och klasser.
\item Nej, inte märkbart.
\item TypeScript har bidragit till att det blivit enklare att utveckla men språket är lite rörigt då den även tillåter JavaScript.
\item Har fungerat bra, modernt språk och känner igen syntax från innan.
\end{enumerate}

\subsection{Gruppmedlem 2}
\begin{enumerate}
\item Ingen alls.
\item Mest TypeScript och lite JavaScript.
\item Har hänt ett par gånger.
\item Ja, det har hänt vid felstavningar.
\item Hjälper att veta typen, men bättre variabelnamn hade kunnat ersätta.
\item Lite, men inte nämnvärt.
\item Ingen speciell skillnad.
\item Inget speciellt.
\item Det har fungerat bra, men har inte så mycket att jämföra med.
\end{enumerate}

\subsection{Gruppmedlem 3}
\begin{enumerate}
\item Endast lite JavaScript.
\item Har använt båda ungefär lika mycket.
\item Inte något som märkts. Något bättre felmeddelanden.
\item Lite färre fel, men enhetstester hade kunnat fånga upp dem.
\item Nja, något enklare i funktioner, men kontrakt och kommentarer kan ersätta.
\item Mer att skriva men får veta om fel tidigare, tror det jämnar ut sig.
\item Lättare att refakorisera, om man glömmer att ändra  någonstans får man felmeddelanden.
\item Stör sig på att typerna kommer efter funktioner och variabler. Saknas typer som heltal.
\item Ingen emot att TypeScript användes, tror inte det hade gjort någon skillnad att skriva i JavaScript istället.
\end{enumerate}

\subsection{Gruppmedlem 4}
\begin{enumerate}
\item Ingen alls.
\item Har använt mycket TypeScript med delar i JavaScript.
\item Ja, ett par gånger. Till exempel att skicka fel variabel till funktion.
\item Tror det lätt hänt att skicka fel typ av variabel till funktioner.
\item Ja, hjälper att veta vad funktioner tar för argument och returnerar. Hade kunnat ersättas med namngivning, men det är jobbigt.
\item Lite sämre produktivitet. Är jobbigt att skapa klasser till små objekt.
\item Om man ändrar på ett ställe men glömmer på ett annat får man ett fel vilket är bra.
\item Inget speciellt.
\item Tror att vi tjänade mycket på TypeScript, speciellt då ramverket som användes till klienten byggde på det.
\end{enumerate}

\subsection{Gruppmedlem 5}
\begin{enumerate}
\item Lite JavaScript, visste inte ens vad TypeScript var.
\item Främst TypeScript men har jobbat lite med JavaScript-bibliotek.
\item Ja, IDE:n har klagat många gånger om typfel under skrivning.
\item Ja, blir enkelt små fel.
\item Ja, bra med autokomplettering och hjälper att veta om t.ex. ett personnummer är av typen nummer eller sträng.
\item Har inte påverkat produktiviteten märkbart.
\item Inte speciellt.
\item En del syntax är svår att förstå, känns som att mycket händer bakom kullisserna.
\item Tror det var bra och underlättade då klasser kunde återanvändas.
\end{enumerate}

 

