\chapter{Intervjuer om teamledarens roll kombinerad med Scrum-metodik}\label{appendix:adam}

\section{Intervjufråga 1}
\textbf{Vilken roll har du i projektet?}
\begin{enumerate}
\item Teamledare
\item Teamledare
\item Konfigurationsansvarig
\end{enumerate}

\section{Intervjufråga 2}
\textbf{Hur kommer det sig att ni valde att arbeta med Scrum?}
\begin{enumerate}
\item Det kändes som en självklar metod do vi ville ha en agil utvecklingsmetod, kändes som en vettig bas och grund. Vi har även gjort lite förändringar i metoden.
\item Vi ville prova på en agil utvecklingsmetod och ha kontinuerliga uppdateringar om statusen på projektet. Scrum kändes bra då man har lite koll på det sedan innan. Det blev en ganska anpassad version av det.
\item Jag tror att det var ett förslag från kunden plus att vi ville jobba iterativt, Scrum var den metodik vi kände till bäst.
\end{enumerate}

\section{Intervjufråga 3}
\textbf{Vilka delar av Scrum har ni tillämpat?}
\begin{enumerate}
\item Vi har inte daily Scrum. Våra sprinter ligger på en vecka och vi har applicerart review, retrospective och planning.
\item Allting är lite informellt, vi försöker ha daily Scrum, kanban-board fungerar inte jättebra i vår grupp. Vi har kört vårat eget race och inte hållit en så formel standard på retrospective, review och planning. Vi har helt enkelt inte kört de formella Scrum-reglerna.
\item Vi använder Kanban-bräde. Vi kör sprintar på två veckor och sprint planning, inte så mycket retrospective.
\end{enumerate}

\section{Intervjufråga 4}
\textbf{Vem är Scrum-master?}
\begin{enumerate}
\item Utvecklingsledaren
\item Teamledaren
\item Teamledaren
\end{enumerate}

\section{Intervjufråga 5}
\textbf{Varför valde ni personen i fråga 4 som Scrum-master?}
\begin{enumerate}
\item Kändes ganska självklart vem som skulle ha rollen. Utvecklingsledaren frågade lite vad man skulle göra som Scrum-master och då kom vi fram till att det var han som skulle ha den rollen.
\item Teamledaren (jag) tog initiativ och läste på om Scrum och för de flesta kändes det naturligt att teamledaren tog den rollen. 
\item Det kändes rimligt, han styr upp massa saker.
\end{enumerate}

\section{Intervjufråga 6}
\textbf{Hur har det gått med att upprätthålla processerna i Scrum?}
\begin{enumerate}
\item Dailies är svårt att få naturliga. Tentaperioder och lediga dagar har ställt till det lite med sprintplaneringen.
\item Det har varit det svåraste, att upprätthålla allting, alla ses inte så ofta så daily Scrum är svår att upprätthålla. Det saknas erfarenhet att jobba med Scrum så det är svårt att få rutin på det.
\item Det har fungerat bra, har skjutit lite på tiderna ibland.
\end{enumerate}

\section{Intervjufråga 7}
\textbf{Hur används rollen som teamledare i erat projekt?}
\begin{enumerate}
\item Personen har mest haft ansvar för att hålla koll på vad som behöver göras och diskuteras, övergripande koll. Andra uppgifter har inkluderat att bokaa handledarmöten, skicka in inlämningar och strukturera upp alla andra.
\item Kallar till och håller i möten. Personen har fått ta på sig mer ledarroll än teamledarroll, ser till att deadlines hålls och håller kontakt uppåt.
\item Kommunikation med kund och bokning av möten med dem. Har bokat lokaler, listat upp vad vi ska göra och håller i handledarmöten. Personen har även blivit ganska mycket dokumentansvarig och sett till att dokument har blivit skrivna.
\end{enumerate}

\section{Intervjufråga 8}
\textbf{Hur fungerar det med att ha teamledare i Scrum-metodik?}
\begin{enumerate}
\item Man måste ändå ha någon runtomkring, det är mycket annat vi gör utöver Scrum. Hade vi enbart kört med Scrum kanske inte teamledaren hade behövts så mycket.
\item Det blir lite att teamledare har hamnat som Scrum-master, har koll på dokumentskrivning och har applicerat Scrum på utvecklingen.
\item Det har fungerat bra, alla har fått vara med och bestämma men teamledaren har dragit lite i det och styrt upp.
\end{enumerate}

\section{Intervjufråga 9}
\textbf{Har både rollen som teamledare och Scrum-metodik gynnats av varandra?}
\begin{enumerate}
\item Tycker det är ganska skönt att det är uppdelat då vi har många andra mål som ska utföras medans Scrum-master kan planera och hålla koll på backloggen. Skönt att Scrum-master slipper fokusera på saker runtomkring.
\item Hade varit skönt om det var mer uppdelat. Ganska körigt att vara både teamledare och Scrum-master, men vid uppdelning av dessa hade man behövt fler personer som tar initiativ.
\item Ja, jag tror att som teamledare har man stöd av det till exempel när man pratar med kund. Det är lättare att hålla koll på planeringen.
\end{enumerate}

\section{Intervjufråga 10}
\textbf{Finns det något man kan ändra på, i teamledar-rollen eller Scrum-metodiken, för att arbetet skulle kunna förbättras?}
\begin{enumerate}
\item Som kombinerad teamledare och Scrum-master måste man försöka låta dem sköta så mycket som möjligt, det blir lätt att man lägger sig i lite för mycket.
\item Det hade gynnats om fler hade läst på om Scrum och var insatta i vad det innebär för att arbetet skulle gå smidigt. Det hade behövts en mer ingående utbildning i Scrum.
\item Teamledaren har kanske behövt ta lite för mycket ansvar som egentligen inte tillhör det ansvarsområdet. Det hade varit bra med en tydligare uppdelning av ansvaret. Det hade också varit bra om vi hade haft dagliga möten och uppföljning.
\end{enumerate}

\section{Intervjufråga 11}
\textbf{Har du något övrigt att tillägga?}
\begin{enumerate}
\item Nej
\item Nej
\item Nej
\end{enumerate}