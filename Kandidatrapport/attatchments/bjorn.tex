\chapter{Intervjuer för versionshantering för ett mindre mjukvaruutvecklingsprojekt.}






\vspace{3em}
\begin{center}
    Christoffer Sjöbergsson | Mjukvaruteknik Linköpings universitet
\end{center}

\begin{enumerate}
    \item Kan du kort beskriva ett mindre mjukvaruutvecklingsprojekt som du har medverkat i? Vad var syftet med projektet?


    Al projekt som är en del av utbildningen vars syfte var att utveckla en bot som spelar start carft II. Projektgruppen bestod av sex personer och majoriteten av koden skrevs i programspråket C++.


    \item Vilket versionshanteringsprogram användes under projektet?

    Git med GitLab.

    \item Hade ni möjligheten att välja?

    Ja

    \item Följdfråga om föregående fråga sant: Hur kommer det sig att ni valde just det ni gjorde?

    Alla studenter som gruppen bestod av hade genom Linköpings universitet ett konto på Gitlab, gruppen ansåg också att Git var det enklaste och bästa verktyget för versionshantering.

    \item Följdfråga om föregående fråga falskt: Hur påverkade det projekt?

    \item I projekt hade ni en arbetsmetodik för versionshantering, om så var fallet hur fungerade den?

    Tanken var från början att använda arbetssättet ``feature bransch'' som kan översättas till ``funktionalitets förgrening''. Ingen funktionalitet från Gitlab användes under projektet. Med det var inte något som bestämdes utan mer utav en informell överenskommelse.

    \item Använde sig gruppen av metodiken så som det var tänkte i teorin?

    Eftersom det inte fanns någon hårt definierad teoretisk metodik utan en informell överenskommelse om hur Git skulle användas kan men inte svara på om gruppen använde metodiken eller inte.

    \item Passade metodiken storleken på gruppen?

    Ja, eftersom metodiken arbetades fram dynamiskt efter gruppens uppbyggnad.

    \item Använde gruppen något speciellt ramverk eller arbetssätt för att hantera arbetsprocessen och planera arbetet, t.ex. Veckomöten, Scrum?

    Gruppen arbetade mycket tillsammans och hade en flytande ``backlog'', som är en lista med arbetsuppgifter. Listan definierades på plats i realtid.

    \item Följdfråga: Vävde ni samman arbetssättet med versionshanteringen och arbetsmetodiken på något sätt?

    Då arbetet var mycket dynamiskt användes grenar i Git för att dela upp koden för parallellt arbete.

    \item Skulle du säga att arbetsmetodiken för versionshanteringen som ni använde var effektiv? Varför?

    På grund av att det inte fanns någon tydligt definierade metodik så uppstod en del problem så som långlivade grenar. Dessa fick stora konflikter med huvudgrenen. Annars så var inte metodiken särskilt besvärlig och var på så sätt lätt att använda. Git användes mer som ett sätt att spara ändringar.

    \item Hur tycker du att programmet i sig fungerade i projektet?

    Bra

    \item Var det svårt att lära sig versionshanteringsprogrammet? Fanns det programvara som gjorde versionshanteringen blev lättare att hantera?

    Gruppen anpassade sig ganska snabbt och lärde sig allt mer komplicerade funktioner under tidens gång. Somliga använde inbyggt stöd i editorn.
\end{enumerate}






\vspace{3em}
\begin{center}
    Tor Utterborn | Datateknik Linköpings universitet
\end{center}
\begin{enumerate}

  \item Kan du kort beskriva ett mindre mjukvaruutvecklingsprojekt som du har medverkat i? Vad var syftet med projektet?
  Arbetar på Ericsson med att utveckla mjukvara för test, 4år deltidstjänst.
  Arbetet kretsade kring utvecklingen av mikrodatorer som kan stresstesta mobilmaster.

  \item Vilket versionshanteringsprogram användes under projektet?
  SVN - Apache Subversion.

  \item Hade ni möjligheten att välja?
  Nej, standard på arbetsplatsen.

  \item Följdfråga om föregående fråga sant: Hur kommer det sig att ni valde just det ni gjorde?
  \item Följdfråga om föregående fråga falskt: Hur påverkade det projekt?
  Då det var nytt när för Tor när han började var det svårt att komma in i tänket.

  \item I projekt hade ni en arbetsmetodik för versionshantering, om så var fallet hur fungerade den?
  SVN användes mer som ett sätt att spara ändringar, alla ändringar testades sedan automatiskt på natten.

  \item Använde sig gruppen av metodiken så som det var tänkte i teorin?

  Många egna praktiska lösningar användes då det uppstod problem.

  \item Passade metodiken storleken på gruppen?

  Utmärkt, då den var så pass liten.

  \item Använde gruppen något speciellt ramverk eller arbetssätt för att hantera arbetsprocessen och planera arbetet, t.ex. Veckomöten, Scrum?

  En lista på arbetsuppgifter som innehöll brådskandegrad.

  \item Följdfråga: Vävde ni samman arbetssättet med versionshanteringen och arbetsmetodiken på något sätt?

  Inte mer än att meddelande som skrevs för ändringar i SVN var inspirerade av uppgiftslistan.

  \item Skulle du säga att arbetsmetodiken för versionshanteringen som ni använde var effektiv? Varför?

  När man väl kan det så flyter allt på bra.

  \item Hur tycker du att programmet i sig fungerade i projektet?

  Samma som föregående fråga.

  \item Var det svårt att lära sig versionshanteringsprogrammet? Fanns det programvara som gjorde versionshanteringen blev lättare att hantera?

  Svårt att komma in i och svårt att våga då man inte visste hur det fungerade.

\end{enumerate}





\vspace{3em}
\begin{center}
    Jesper Jonsson | Datateknik Linköpings universitet
\end{center}
\begin{enumerate}

  \item Kan du kort beskriva ett mindre mjukvaruutvecklingsprojekt som du har medverkat i? Vad var syftet med projektet?

  Forte studentförening på Linköpings universitet - arbetar med fortes bokningssystem som kontinuerligt utvecklas och underhålls. Det är fem aktiva utvecklare av systemet.

  \item Vilket versionshanteringsprogram användes under projektet?

  Git - Github

  \item Hade ni möjligheten att välja?

  Ja

  \item Följdfråga om föregående fråga sant: Hur kommer det sig att ni valde just det ni gjorde?

  Alla var vana med Git samt att Github har många praktiska verktyg som underlättar arbetet.

  \item Följdfråga om föregående fråga falskt: Hur påverkade det projekt?

  \item I projekt hade ni en arbetsmetodik för versionshantering, om så var fallet hur fungerade den?

  Projekt som finns på Github användes, där inkluderas en kamband liknande metodik. Det kan beskrivas som tavlor likt verktyget Trello som används för att flytta uppgifter och visualisera vart de befinner sig i produktionen. Olika uppgiftslistor med kort används också. Funktionalitets förgrening används till hundra procent, utöver det så granskades alla ändringar av minst två personer innan de integreras med huvud grenen. Circleci ramverket används för kontinuerlig testning.

  \item Använde sig gruppen av metodiken så som det var tänkte i teorin?

  Inte alltid då redaktionella ändringar ibland hoppar över metodiken eftersom det är onödigt.

  \item Passade metodiken storleken på gruppen?
  Optimerat för storleken av gruppen.

  \item Använde gruppen något speciellt ramverk eller arbetssätt för att hantera arbetsprocessen och planera arbetet, t.ex. Veckomöten, Scrum?

  Inte egentligen då det är mer av ett sidoprojekt.

  \item Följdfråga: Vävde ni samman arbetssättet med versionshanteringen och arbetsmetodiken på något sätt?

  Projekt funktioner i Github används mycket.

  \item Skulle du säga att arbetsmetodiken för versionshanteringen som ni använde var effektiv? Varför?

  Skönt att ha allt på samma ställe och integrationen med GitHub.

  \item Hur tycker du att programmet i sig fungerade i projektet?

  Ja.

  \item Var det svårt att lära sig versionshanteringsprogrammet? Fanns det programvara som gjorde versionshanteringen blev lättare att hantera?

  Inte direkt då alla var vana med Git från tidigare erfarenheter. Vissa använder det inbyggda stödet för Git i vissa text redigerings program.

\end{enumerate}






\vspace{3em}
\begin{center}
    Jakob Norell | Datateknik Linköpings universitet
\end{center}
\begin{enumerate}

  \item Kan du kort beskriva ett mindre mjukvaruutvecklingsprojekt som du har medverkat i? Vad var syftet med projektet?

  Konstruktion med mikrodatorer en kurs på Linköpings universitet.

  \item Vilket versionshanteringsprogram användes under projektet?

  Git och Github.

  \item Hade ni möjligheten att välja?

  Ja.

  \item Följdfråga om föregående fråga sant: Hur kommer det sig att ni valde just det ni gjorde?

  Några medlemmar i gruppen var redan bekanta med Git och alla var överens om att vi skulle versionshantera på något sätt. Alla var också positivt inställda till att lära sig Git.

  \item Följdfråga om föregående fråga falskt: Hur påverkade det projekt?

  \item I projekt hade ni en arbetsmetodik för versionshantering, om så var fallet hur fungerade den?

  Gruppen gjorde en gren för varje delmodul i projektet. Och en huvudgren som gruppen endast skickade färdig funktionalitet till och testade. Alla ändringars meddelande skulle vara korta och beskrivande. Dom skulle vara utformade efter en förutbestämd mall.

  \item Använde sig gruppen av metodiken så som det var tänkte i teorin?

  Ja på sätt och vis, det var vissa detaljer som kunde har fungerat bättre, till exempel så laddades flera olika ändringar upp med samma meddelande vilket var lite förvirrande. På grund av oerfarenhet som gjordes misstag som ledde till att gruppen var tvungna att kringgå metodiken.

  \item Passade metodiken storleken på gruppen?

  Då det var ett litet projekt så kändes det som att det enkla tillvägagångssätt.

  \item Använde gruppen något speciellt ramverk eller arbetssätt för att hantera arbetsprocessen och planera arbetet, t.ex. Veckomöten, Scrum?

  Gruppen använde sig av projekt modellen LIPS med veckomöten.

  \item Följdfråga: Vävde ni samman arbetssättet med versionshanteringen och arbetsmetodiken på något sätt?

  Inte mer än delade namn på vissa uppgifter.

  \item Skulle du säga att arbetsmetodiken för versionshanteringen som ni använde var effektiv? Varför?

  Ja, mer positivt än negativt. Var en extrem fördel att kunna jobba parallellt utan att vara ett hinder för varandra.

  \item Hur tycker du att programmet i sig fungerade i projektet?

  Det fungerade bra.

  \item Var det svårt att lära sig versionshanteringsprogrammet? Fanns det programvara som gjorde versionshanteringen blev lättare att hantera?

  Det finns en viss inlärningskurva med det var inte ett problem för våran grupp, det fanns personer som kunde och lärde upp de andra. Någon i gruppen använde text hanterare så som atom som har integrerat stöd för Git.

\end{enumerate}
