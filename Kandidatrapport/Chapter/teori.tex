\chapter{Teori}
Detta kapitel beskriver teorin bakom de olika metoder och verktyg som har använts.

\section{Scrum} \label{scrum}
Scrum är en agil utvecklingsmetod som består av tre komponenter, produktägaren, scrummästaren och utvecklingsteamet. 
Produktägaren är den person som kommer med en beställning till utvecklingsteamet. 
Scrummästaren arbetar för att scrum-metodiken ska upprätthållas och ser, tillsammans med produktägaren, till att underhålla backloggen så att utvecklingsteamet förstår vad som ska göras och så att arbetet kan fortgå. 
Utvecklingsteamet är själva kärnan i scrum-metodiken, det är teamet som bestämmer vilka uppgifter som ska utföras under en sprint och hur de ska lösas. 
För en utförligare beskrivning av scrum-metodiken, se avsnitt \ref{adam_scrum}.

\section{GitLab}
GitLab är ett versionshanteringssystem på internet, som versionshanterar med Git, där de som är medlemmar i projektet kan lägga till och ändra i filer. 
Git fungerar på det sättet att varje medlem i ett projekt arbetar på en kopia lokalt för att sedan skicka upp de ändringar man gjort till den gemensamma mappen online. 
GitLab har också tidigare versioner sparade så om någon skulle göra uppdateringar av en programvara som inte fungerar så kan man enkelt gå tillbaka till en tidigare version av programmet.

I Git kan man skapa grenar i ett projekt, detta innebär att man arbetar på en kopia av projektet som sedan skickas upp och måste godkännas av någon annan medlem i projektet för att kunna slås ihop med den ursprungliga projektmappen. 
Det finns också möjlighet att ärendehantera med hjälp av Git. 
Då skapar en person ett ärende om att den t.ex. har hittat en bugg eller att den anser att en ny funktionalitet ska implementeras. 
Sedan så finns dessa ärenden sparade i projektet så att medlemmar kan se och hantera dem. \cite{gitlab}

\section{Webstorm}
WebStorm är en IDE som främst används för web, JavaScript och TypeScript utveckling. WebStorm innehåller flera smidiga funktioner såsom kodkomplettering, API-dokumentation och inbyggt stöd för versionshantering. \cite{webstorm}

\section{Node.js}

\section{Angular}

\section{MySQL}
MySQL är ett databashanteringssystem för att skapa och behandla data. För att hålla reda på information förvarar MySQL data i tabeller. En tabell innehåller data som är relaterad med varandra och sedan kan man koppla samman dessa tabeller med varandra.
\cite{mysql}

\section{Sequelize}
Sequelize är ett ORM-system som används för att konvertera data från databaser så att du kan hämta och manipulera data utan att använda SQL.
\cite{sequelize}
