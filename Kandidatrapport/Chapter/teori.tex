\chapter{Teori}
Detta kapitel beskriver teorin bakom de olika metoder och verktyg som har använts.

\section{Scrum}
Scrum är en agil utvecklingsmetod som består av tre organ, produktägaren, scrummästaren och utvecklingsteamet. Produktägaren är den person som kommer med en beställning till utvecklingsteamet. Scrummästaren arbetar för att scrum-metodiken ska upprätthållas och ser till, tillsammans med produktägaren, att underhålla backloggen så att utvecklingsteamet förstår vad som ska göras och så att arbetet kan fortgå. Utvecklingsteamet är själva kärnan i scrum-metodiken, det är teamet som bestämmer vilka uppgifter som ska utföras under en sprint och hur de ska lösas. För en utförligare beskrivning av scrum-metodiken, se avsnitt \ref{adam_scrum}.