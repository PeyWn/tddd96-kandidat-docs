\chapter{Teori}

\section{Scrum}
Scrum är en arbetsmetodik som främst används för mjukvaruutveckling. Metodiken appliceras på mindre team (3-9 personer) och går ut på att teamet arbetar i tidsbestämda sprintar där det existerar ett antal uppgifter som ska genomföras under respektive sprint.
I scrum så existerar ett fåtal specifika roller som används för att beskriva de inblandade i projektet. De roller som existerar är:

\begin{itemize}
	\item \textbf{Produktägare}
	
	Produktägaren är den person som representerar beställaren, vanligtvis aktieägarna i större bolag, och är den röst mot utvecklingsteamet gällande de krav som finns och vilka som har högst prioritet. En viktig egenskap för en produktägare är empati, då man arbetar mot aktieägare som ofta har olika bakgrund och värderingar som man måste förhålla sig till. Produktägaren måste också vara bra på att samarbeta med utvecklarna så att arbetet med produkten går bra och att krav uppfylls.
	
	\item \textbf{Scrummästare}
	
	Scrummästaren skiljer sig från den klassiska teamledaren eller projektledaren. Dess roll är att se till att det inte existerar några hinder för utvecklingsteamet så att de kan uppnå projektmål och leverera en produkt. En annan viktig uppgift för scrummästaren är att se till att de processer som man valt att applicera i scrum-metoden utförs och efterföljs. Samarbete med produktägaren finns också för att underhålla backloggen så att utvecklingsteamet förstår vad som ska göras och så att arbetet med produkten kan fortgå.
	
	\item \textbf{Utvecklingsteam}
	
	Utvecklingsteamet är själva projektgruppen och de ansvarar för att göra framsteg i arbetet och utföra de uppgifter som existerar i respektive sprint. Trots att det kan finnas olika typer av roller så kallas alla i teamet för utvecklare, för att inte skapa förvirring så kallas utvecklingsteamet för leveransteam och personerna för teammedlemmar.	
\end{itemize}

Utöver de roller som existerar i scrum så använder man sig också utav fyra olika möten som inträffar med olika frekvens. Det viktiga med dessa möten är att de är tidsbestämda och den maximala tid som ska läggas på dessa beror i flera fall på hur lång en sprint i projektet är, detta för att man inte ska lägga alltför lång tid på möten. De möten som existerar är:

\begin{itemize}
	\item \textbf{Sprintplanering (maximalt 4 timmar för en tvåveckors-sprint}
	
	Sprintplaneringen är uppdelad i två delar, den första delen går ut på att produktägaren, scrummästaren och utvecklingsteamet väljer de uppgifter från backloggen som man tror kommer hinnas med under den kommande sprinten. När detta är gjort så kommer planeringens andra del, utvecklingsteamet diskuterar och kommer mer specifikt fram till vilka arbetsuppgifter som behöver göras för att uppgifterna från backloggen ska kunna utföras och uppfyllas. Denna andra iteration över uppgifterna kan leda till att man delar upp vissa uppgifter eller att vissa uppgifter läggs tillbaka i backloggen då man inte tror att dessa kommer hinnas med under kommande sprint.
	
	\item \textbf{Daglig scrum (maximalt 15 minuter)}
	
	Varje dag så hålls ett dagligt scrum möte för att uppdatera utvecklingsteamet om vilka framsteg och hinder som finns i sprinten så att teamet kan öka sina chanser att nå målen med nuvarande sprint. Dessa möten är tidsbestämda till 15 minuter och de kan utföras stående i en ring, så att man inte fokuserar på annat och så att man håller sig inom de 15 minuterna. Man kan utgå ifrån tre frågor under en daglig scrum:
	
	\begin{itemize}
		\item Vad har jag gjort sedan igår?
		\item Vad ska jag åstadkomma till imorgon?
		\item Vad hindrar mig?
	\end{itemize}
	
	Om det skulle komma fram att det finns några hinder i gruppen så ska scrummästaren notera detta och man bör sätta en person på att lösa problemet så att sprinten kan fortgå utan hinder.
	
	\item \textbf{Sprintgenomgång (maximalt 2 timmar för en tvåveckors-sprint)}
	
	När en sprint är avslutad så hålls en sprintgenomgång, där man tittar på vilka arbetsuppgifter utvecklingsteamet har, och inte har, hunnit med att slutföra. Man håller också en demo för beställaren där man visar vilka framsteg som har gjorts under sprinten. Efter genomgång av utförda och icke utförda arbetsuppgifter samt demo så diskuterar utvecklingsteamet och beställaren vad teamet ska arbeta med härnäst.
	 
Det som är viktigt att tänka på under en sprintgenomgång är att man inte ska visa upp arbetsuppgifter som inte är färdiga t.ex. ska man inte visa upp funktionalitet i ett program om funktionaliteten inte är helt färdig.

\item \textbf{Sprintåterblick (maximalt 90 minuter för en tvåveckors-sprint)}

När en sprintgenomgång har utförts så är det dags för en sprintåterblick. Sprintåterblick innebär att man utvärderar den sprint som har varit genom att utgå från två frågor:

\begin{itemize}
	\item Vad gick bra under sprinten?
	\item Vad kan förbättras till nästa sprint?
\end{itemize}

Från dessa två frågor identifierar man förbättringar som kan göras, man väljer ut några av dessa punkter och arbetar med dem i nästa sprint.

\end{itemize}

\section{GitLab}
GitLab är ett versionshanteringssystem på internet där de som är medlemmar i projektet kan lägga till och ändra i filer. Git fungerar på det sättet att varje medlem i ett projekt arbetar på en kopia lokalt för att sedan skicka upp de ändringar man gjort till den gemensamma mappen online. GitLab har också tidigare versioner sparade så om någon skulle göra uppdateringar av en programvara som inte fungerar så kan man enkelt gå tillbaka till en tidigare version av programmet.

Det går också att skapa grenar i ett projekt, detta innebär att man arbetar på en kopia av projektet som sedan skickas upp och måste godkännas för att slås ihop med den ursprungliga projektmappen.  
https://about.gitlab.com/2015/05/18/simple-words-for-a-gitlab-newbie/

\section{Webstorm}

\section{Node.js}

\section{Angular}

\section{MySQL}
MySQL är ett databashanteringssystem för att skapa och behandla data. För att hålla reda på information förvarar MySQL data i tabeller. En tabell innehåller data som är relaterad med varandra och sedan kan man koppla samman dessa tabeller med varandra.
https://www.w3schools.com/php/php_mysql_intro.asp 

\section{Sequelize}
Sequelize är ett ORM-system som används för att konvertera data från databaser så att du kan hämta och manipulera data utan att använda SQL.
http://docs.sequelizejs.com/