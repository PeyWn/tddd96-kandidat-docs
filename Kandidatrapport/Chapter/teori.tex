\chapter{Teori}
Detta kapitel beskriver teorin bakom de olika metoder och verktyg som har använts.

\section{Scrum} \label{scrum}
Scrum är en agil utvecklingsmetod som består av tre komponenter, produktägaren, scrummästaren och utvecklingsteamet.
Produktägaren är den person som kommer med en beställning till utvecklingsteamet.
Scrummästaren arbetar för att scrum-metodiken ska upprätthållas och ser, tillsammans med produktägaren, till att underhålla backloggen så att utvecklingsteamet förstår vad som ska göras och så att arbetet kan fortgå.
Utvecklingsteamet är själva kärnan i scrum-metodiken, det är teamet som bestämmer vilka uppgifter som ska utföras under en sprint och hur de ska lösas.
För en utförligare beskrivning av scrum-metodiken, se avsnitt \ref{adam_scrum}.

\section{GitLab}
GitLab är ett webbhotell som versionshanterar med versionshanteringsprogrammet Git, där de som är medlemmar i projektet kan lägga till och ändra i filer.
Med Git så arbetar varje medlem i ett projekt på en lokal kopia av data som finns på gitlabservern för att sedan skicka upp de ändringar man gjort till gitlabservern. Git fungerar på så sätt att alla ändringar sparas. Det betyder att om någon skulle göra en ändring som har oönskad effekt kan man enkelt gå tillbaka till en tidigare version av den data som finns på gitlabservern.

Git har stöd för något som kallas grenar, en gren kan beskrivas som en kopia av all projektets data, vanligtvis av den data som finns i projektets huvudgren. När en ny gren har skapats så sker alla ändringar som görs på den nya grenen. Det finns flera fördelar med att arbeta i grenar. För det första så möjliggör det parallellt och strukturerat arbete då man kan namnge en gren så namnet motsvarar ändringarnas syfte. Till exempel så kan det vara namnet på någon ny funktionalitet. Vidare så gör det inget för om en gren inte fungerar i sitt nuvarande tillstånd, det gör att man inte förstör för sina medarbetar när man testar nya saker. För det andra så kan man genom gitlab skapa sammanslagningsförfrågningar för att väva samman två grenar. Fördelen med det är att det på ett väldigt enkelt sätt går att se vilka ändringar som har gjorts på den nya grenen och vad som kommer att uppdateras.

Det finns också möjlighet att ärendehantera med hjälp av Git.
Då skapar en person ett ärende om att den t.ex. har hittat en bugg eller att den anser att en ny funktionalitet ska implementeras.
Sedan finns dessa ärenden sparade på projektet sida på gitlab så att medlemmar kan se och hantera dem. \cite{gitlab}

\section{Webstorm}
WebStorm är en IDE som främst används för web, JavaScript och TypeScript utveckling. WebStorm innehåller flera smidiga funktioner såsom kodkomplettering, API-dokumentation och inbyggt stöd för versionshantering. \cite{webstorm}

\section{Node.js}

\section{Angular}

\section{MySQL}
MySQL är ett databashanteringssystem för att skapa och behandla data. För att hålla reda på information förvarar MySQL data i tabeller. En tabell innehåller data som är relaterad med varandra och sedan kan man koppla samman dessa tabeller med varandra.
\cite{mysql}

\section{Sequelize}
Sequelize är ett ORM-system som används för att konvertera data från databaser så att du kan hämta och manipulera data utan att använda SQL.
\cite{sequelize}
