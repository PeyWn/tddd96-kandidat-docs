\chapter{Resultat}
I denna del kommmer resultatet av projektet beskrivas. Det innebär dels
resultatet den mjukvara som har utvecklats, dels vilka erfarenheter som
teamet har samlat på sig under projektets gång.
\section{Systembeskrivning}
Systemet som skapats består av två separata delar, en back-end och en front-end.

\subsection{Front-end}
Projektets front-end utvecklades i ramverket Angular. Front-end använder sig därför av den komponentbaserade arkitektur som starkt förordas av ramverket. Komponenter i Angular är indelade i delar som i grova drag motsvarar delarna Model, View och Control i designmönstret MVC. Av denna anledning har applikationen följt detta designmönster på komponentnivån.    

Det grafiska innehållet på projektets front-end var uppdelat i komponenter som uppdaterades eller byttes ut till andra komponenter när användaren interagerade med systemet. De komponenter som fanns överst i applikationens komponenthierarki var tomma behållare vars enda syfte var att separera applikationens olika beståndsdelar från varandra. Inuti dessa placerades sedan komponenter med faktiska funkioner, som knappsatser eller sökrutor.

Applikationen bestod av flera olika vyer, där vissa av vyerna skulle visas upp i samma behållare men vid olika tillfällen, beroende på vilken vy användaren för stunden är intresserad av. För att byta mellan vyerna användes Angulars tjänster. Tjänsterna var inte direkt kopplade till någon specifik komponent och kunde därför användas för att kommunicera mellan komponenter i olika delar av komponenthierarkin.

All den data om patienter, salar och utrustning som användaren var intresserad av fanns på en server i projektets back-end. Därför var det nödvändigt med kommunikation mellan de båda delarna. På projektets front-end sköttes kommunikationen med AJAX-anrop. Dessa anrop skedde i tjänster för att de skulle vara tillgängliga för alla kompononenter som behövde tillgång till data av olika sorter.    

\subsection{LoFi-Prototyper}
Under projektets andra iteration utvecklades tre olika LoFi-prototyper. Dessa
LoFi-prototyper designades för att visa på olika designalternativ på
användargränssnittet. Ett exempel på detta är informationen som var med i listan
över beslutade operationer. I en av prototyperna identifierades patienten som hörde ihop med beslutet med namn, den andra med personnummer. I den tredje LoFi-prototypen
fanns inget som identifierade patienten utan enbart information om vilken
typ av operation det var. På liknande sätt utvärderades olika sätt att
visualisera lediga tider i schemat samt olika sätt att anpassa sökparametrarna för en sökning efter lediga tider. När prototyperna visades för kunden kunde alternativen sållas bort och en tydligare bild av behoven framträdde.
Till exempel fanns det endast personnummer och inte namn på patienterna, vilket var något som önskades.

Den första iterationen av LoFi-prototyper användes sedan för att ta fram en ny
LoFi-prototyp med enbart mindre designalternativ som visades för tre olika
operationsplanerare.
\subsection{Systemanatomi}
\begin{figure}
\includegraphics[width=\textwidth,height=.4\textheight]{Figures/Systemanatomi.png}\\
\caption{Systemanatomi}
\label{fig:Systemanatomi}
\end{figure}

I första iterationen togs det fram en systemanatomi som användes under projektets gång som ett hjälpmedel för strukturera upp arbetet under utvecklingen. Den gav en översiktlig bild av vilken funktionalitet produkten skulle innehålla och hur de olika delarna samverkade med varandra. Utöver detta presenterades den i ett tidigt skede för kunden i syfte att säkerställa att projektgruppens syn på systemet stämde överens med kundens. 

\subsection{Värde för kund}
Den produkt som har skapats är ett lättöverskådligt schemaläggningssystem för operationer. Det följer ***Skriv in typiskt bra krav som är godkända av kund och som visar på funktionalitet och värde för kunden***

\section{Separation av front-end och back-end}

\section{Integration i redan existerande system}

\section{Gemensamma erfarenheter}
I början av projektet fick vi snabbt större erfarenhet av olika områden
tack vare de olika presentationer som de olika deltagarna i projektet gav.
Vi fick även en ökad förståelse för hur delar av sjukvården fungerar och att ett bra
it-system kan göra verklig skillnad.

\section{Översikt över individuella bidrag}
I denna delen presenteras deltagarnas individuella bidrag översiktligt.

\todo{Lägg till era rubriker och en kort synopsis här}
\subsection{Adam}
En studie i hur teamledarens roll går att applicera tillsammans med scrum-metodik.
\subsection{Björn}
Hur kan versionshantering användas effektivt för ett mindre mjukvaruutvecklings projekt.
\subsection{Christoffer}
Betydelsen av att samla krav från en varierad grupp aktörer
\subsection{Henrik}
För/nackdelar med TypeScript jämfört med JavaScript
\subsection{Martin}
Angular som webbutvecklingsplattform
\subsection{Niclas}
Prototypuveckling i kandidatprojekt 
\subsection{Tor}
Kvalitetsförsäkrande metoder i ett småskaligt mjukvaruprojekt
