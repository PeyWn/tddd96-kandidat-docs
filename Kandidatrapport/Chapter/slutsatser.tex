\chapter{Slutsatser}
I detta avsnitt ska de frågeställningar som ställts i början av rapporten besvaras.

\textbf{Hur kan systemet som utvecklats implementeras så att värde skapas för kunden?}

Systemet som utvecklats kan bytas ut mot det nuvarande systemet för att få en tydligare grafisk vy över de operationer som är planerade. Det utvecklade systemet ger också ett enkelt sätt att boka in nya operationer på datum och tider där all kompetens, utrustning och material finns tillgängligt.

\textbf{Hur kan man separera front-end och back-end på ett bra sätt så att delarna inte är beroende av varandra?}
För att separera på back-end och front-end så att de kan utvecklas parallellt kan man först definiera ett tydligt API för all kommunikation som kommer att ske mellan dem. När API:et är färdigt kan detta implementeras på back-end i projektet. Efter detta, eller innan om API:et är väldefinierat, kan en modul skapas på front-end som definerar all funktionalitet som krävs för att använda detta API. Nu kan de två delarna fortsätta utvecklas separat så länge API:et inte ändras, och när detta händer är det endast komunikationsmodulen på front-end som behöver uppdateras.

\textbf{Vilka erfarenheter kan dokumenteras från projektet som kan vara intressanta för framtida projekt?}

Under projektets gång så har gruppen fått flera värdefulla erfarenheter. Den största erfarenheten är hur det är att arbeta i projekt med en större grupp människor och hur det är att arbeta med en agil utvecklingsmetod. Att arbeta mot en kund har också varit en bra erfarenhet då detta inneburit extra moment, som kundmöten och demonstrationer av programvaran, till skillnad från ett vanligt skolprojekt.

\textbf{Vilket stöd kan man få genom att skapa och följa upp en systemanatomi?}
Skapandet av en systemanatomi ger upphov till diskussion om vilka huvudfunktioner som ska finnas och hur systemet i helhet ska hänga ihop. Anatomin hjälper till att ge en gemensam syn på systemet både inom projektgruppen och mellan gruppen och kund.   

\textbf{Hur väl kan produkten som utvecklas integreras och arbeta med redan existerande system?}