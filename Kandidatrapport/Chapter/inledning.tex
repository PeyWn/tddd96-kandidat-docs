\chapter{Inledning}
Detta kapitel beskriver syftet med produkten som utvecklats, vilka frågeställningar som ska besvaras i rapporten samt vilka avgränsningar projektet har haft.

\section{Motivering}
På en operationsavdelning utförs olika ingrepp som kräver många olika sorters kompetenser, utrustning och material. 
För att organisera detta arbete sitter
idag flera planerare med ansvar för olika salar och schemalägger olika operationer som ska genomföras. 

Varje planerare ska se till att alla resurser finns tillgängliga. Operationen behöver lokaler att utföras i med behörig personal och rätt verktyg. Dessutom ska operationer med hög brådskandegrad få prioritet. All denna information är
mycket för en person att bevaka. Om det uppstår fel kan det i sin tur leda till att en operation inte kan utföras för att alla behov inte är tillgodosedda vid operationstillfället,
eller att operationssalar står tomma.

För att förenkla processen för planerare att schemalägga operationer ska ett schemaläggningsstöd utvecklas. All information som rör olika operationer ska finnas
tillgänglig så att man enkelt ska kunna få fram tillgängliga operationstider för det ingrepp som ska utföras. 

Denna programvara kommer att förenkla  vårdpersonalens arbete då man inte behöver hålla lika många saker i huvudet. Planerare kommer att kunna se och fylla de tillgängliga operationstiderna som finns.
Detta kommer även att hjälpa dem som väntar på operation. Väntan på att opereras kommer att minska om sjukhus kan optimera den tid som finns tillgänglig. Därmed kan sjukhuset operera fler patienter än vad som görs idag.

\section{Syfte}\label{sec:syfte}
Syftet med projektet var att utveckla ett schemaläggningsstöd för kirurgi för användning inom Region Östergötland. Programmet är tänkt att underlätta och optimera schemaläggning av operationer. Den slutgiltiga produkten tillsammans med demonstrationsmöten och dokumentation kommer att användas av Region Östergötland som en form av förstudie till ett större projekt som pågår på universitets sjukhuset, GOLI(a)T. \cite{goliat} Rapportens syfte är att beskriva tillvägagångssättet för utveckling av produkten, beskrivning av slutprodukten och en utvärdering av hur projektet har gått och hur resultatet blev.

\section{Frågeställningar}
För att ge rapporten ett fokus så har ett antal frågeställningar formulerats som behandlas i rapporten:
\begin{enumerate}
	\item Hur kan systemet som utvecklas implementeras så att värde skapas för kunden?
	\item Hur kan man separera front-end och back-end på ett bra sätt så att delarna inte är beroende av varandra?
	\item Vilka erfarenheter kan dokumenteras från projektet som kan vara intressanta för framtida projekt?
	\item Vilket stöd kan man få genom att skapa och följa upp en systemanatomi?
\end{enumerate}

\section{Avgränsningar}
Denna rapport är avgränsad till den produkt som arbetats fram samt de individuella bidragen. I övrigt så har kunden gett oss tämligen fria händer vid utvecklingen av produkten. Den begränsning vi fick är att produkten ska gå att köra på Internet Explorer 11.
I övrigt så har projektarbetet en tidsbudget på 2800 timmar, det vill säga 400 timmar per person. Dessa timmar är fördelade på olika områden som dokumentskrivning, utveckling och individuellt bidrag.

\section{Definitioner}

\begin{itemize}

\item GOLI(a)T - Gemensam Operationsprocess, Ledning och IT-stöd. Projekt inom Region Östergötland där man siktar på att standardisera och förenklar operationsprocesser.

\item ORM - Object-relational mapping. Objektorienterat system som konverterar databastabeller till klasser, tabellrader till objekt och celler till objekt-attribut.

\item LoFi-prototyper - LoFi betyder ungefär ``low fidelity'' och kan översättas till ``låg precision''. LoFi-prototypen är i det här fallet en ritad pappersprototyp av det grafiska gränssnittet.

\item Front-end - Med front-end menas den del av systemet som körs i användarens webbläsare. Front-end har i denna rapport använts synonymt med klient.

\item Back-end - Med back-end menas den del av systemet som inte körs i användarens webbläsare. Back-end har i denna rapport använts synonymt med server. 

\end{itemize}

\newpage
