\chapter{Inledning}

\section{Motivering}
På en operationsavdelning utförs olika ingrepp som kräver många olika sorters kompetenser, utrustning och material. För att organisera detta arbete sitter idag en person och schemalägger de olika operationer som ska genomföras. Personen ska se till att resurser finns tillgängliga, att operationen har lokaler att utföras i och att den operation med högst brådskandegrad får högst prioritet. All denna information är mycket för en person att hålla koll på och det kan leda till att en operation inte kan utföras för att alla behov inte är tillgodosedda vid operationstillfället eller att operationssalar står tomma.
För att förenkla processen att schemalägga operationer ska ett schemaläggningsstöd tas fram där all information som rör operationen ska finnas tillgänglig så att man enkelt ska kunna få fram tillgängliga operationstider för det ingrepp som ska utföras. Denna programvara kommer att förenkla vårdpersonalens arbete då man inte behöver hålla lika många saker i huvudet och man kommer att kunna fylla upp de tillgängliga operationstider som finns. Detta kommer även att hjälpa dem som väntar på operation då väntan på att opereras kommer att minska om sjukhus kan optimera den tid som finns tillgänglig och därmed kunna operera fler patienter än vad som görs idag.

\section{Syfte}
Syftet med projektet var att utveckla ett schemaläggningsstöd för kirurgi inom Region Östergötland för att underlätta och optimera schemaläggning av operationer.
Rapportens syfte är att beskriva tillvägagångssättet för utveckling av produkten, beskrivning av slutprodukten och en utvärdering av hur projektet har fortgått och hur resultatet blev.

\section{Frågeställning}
För att ge rapporten ett fokus så har ett antal frågeställningar formulerats som behandlas i rapporten:
\begin{enumerate}
	\item Hur kan systemet som utvecklas implementeras så att värde skapas för kunden?
	\item Hur kan man separera front-end och back-end på ett bra sätt så att delarna inte är beroende av varandra?
	\item Vilka erfarenheter kan dokumenteras från projektet som kan vara intressanta för framtida projekt?
	\item Vilket stöd kan man få genom att skapa och följa upp en systemanatomi?
	\item Hur väl kan produkten som utvecklas integreras och arbeta med redan existerande system?
\end{enumerate}

\section{Avgränsningar}
Denna rapport är avgränsad till den produkt som arbetats fram samt de individuella bidragen. I övrigt så har kunden gett oss tämligen fria händer vid utvecklingen av produkten. Den begränsning vi fick är att produkten ska gå att köra på Internet Explorer 11.
I övrigt så har projektarbetet en tidsbudget på 2800 timmar, det vill säga 400 timmar per person. Dessa timmar är fördelade på olika områden som dokumentskrivning, utveckling och individuellt bidrag.

\newpage
