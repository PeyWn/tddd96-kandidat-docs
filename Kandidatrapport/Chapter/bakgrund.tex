\chapter{Bakgrund}
I detta kapitel beskrivs bakgrunden till kundens beslut att beställa projektet.
Utöver det så beskrivs också övergripande de erfarenheter som teamet hade när de gick
in i projektet.

\section{Processen idag}
Planering av operationer på ett sjukhus sker i en mycket komplex miljö. Det
finns väldigt många resurser som måste koordineras för att allt ska fungera som
det ska.
I dagsläget finns nödvändig information om detta i olika system och ibland finns
det inte i något system alls utan bara i huvudet på de olika anställda. Detta
skapar en situation där det är svårt att överblicka information om de olika
resurser som finns och deras tillgänglighet vid olika tider.

Det är mer än bara en operationssal och en kirurg som ska finnas tillgängligt
för att en operation ska kunna genomföras. Det krävs ofta att ett antal prover
genomförs på patienten innan operationen genomförs därefter ska patienten
förberedas inför operationen. Vilka förberedelser som ska göras beror både på
vilken patient det är och vilken operation det är som ska utföras. Sedan är det
själva operationen som ska genomföras. I vissa fall krävs det flera olika
kompetenser på plats i operationssalen och då måste dessa kunna bokas i förväg.
Det är även olika uppsättningar verktyg och specialutrustning för olika
operationer.
Patienten måste givetvis också ha en plats att återhämta sig på efter
operationen, en så kallad postoperativ vårdplats.

\section{Projektet}
För att lösa detta problem behövs ett system som kan visualisera alla resurser
och hitta lediga tider för olika typer av operationer baserat på tillgången av
dessa resurser.

Det är ett stort nytt system under utveckling där schemaläggningsstöd är en del
i ett större sammanhang. Projektet som vi beskriver i denna rapport avser vara
en prototyp för att testa vilka funktioner som är viktiga och även hur dessa
kan implementeras i ett användargränssnitt som är lättarbetat och effektivt.
Det är därför ett primärt fokus på funktioner och gränssnitt över robusthet och
säkerhet.

\section{Gemensamma erfarenheter}
Teamets erfarenheter vid ingången i projektet består mest i tidigare projekt
under utbildningen.
