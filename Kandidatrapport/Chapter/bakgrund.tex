\chapter{Bakgrund} \label{cha:bakgrund}
I detta kapitel beskrivs bakgrunden till kundens beslut att beställa projektet.
Utöver det så beskrivs också övergripande de erfarenheter som teamet hade när de gick
in i projektet.

\section{Processen idag}
Planering av operationer på ett sjukhus sker i en mycket komplex miljö. Det
finns väldigt många resurser som måste koordineras för att allt ska fungera på rätt sätt.
I dagsläget finns nödvändig information om detta i olika system, ibland finns
det inte i något system alls utan informationen finns bara \enquote{i huvudet} på de anställda. Detta
skapar en situation där det är svårt att få en överblick över den information som finns om de olika
resurser och deras tillgänglighet vid olika tider.

Det är mer än bara en operationssal och en kirurg som ska finnas tillgängliga
för att en operation ska kunna genomföras. Det krävs ofta att ett antal prover har
genomförts på patienten före operationen. Därefter ska patienten
förberedas inför operationen. Vilka förberedelser som ska göras beror både på
vilken patient det är och vilken operation det är som ska utföras.

Sedan är det
själva operationen som ska genomföras. I vissa fall krävs det flera olika
kompetenser på plats i operationssalen och då måste dessa kunna bokas i förväg.
Det krävs även olika uppsättningar av verktyg och specialutrustning för olika
operationer.
Patienten måste givetvis också ha en plats att återhämta sig på efter
operationen, en så kallad postoperativ vårdplats.

\section{Projektet}
För att lösa detta problem behövs ett system som kan visualisera alla resurser
och hitta lediga tider för olika typer av operationer baserat på tillgången av
dessa resurser.

Ett stort och nytt system är under utveckling där schemaläggningsstöd är en del
i ett större sammanhang. Projektet som beskrivs i denna rapport är avsett att vara
en prototyp för att testa vilka funktioner som är viktiga och även hur dessa
kan implementeras i ett användargränssnitt som är lättarbetat och effektivt.
Det är därför ett primärt fokus på funktioner och gränssnitt över robusthet och
säkerhet.

\section{Gemensamma erfarenheter}
Teamet hade vid projektets start tidigare erfarenheter i diverse programmeringskurser i Java, C++ och Python.
Samtliga medlemmar har även läst en kurs i programmutvecklingsmetodik för att lära sig teori om hur man arbetar i mjukvaruprojekt.
Medlemmarna har även fått praktiskt erfarenhet av att arbeta i projekt, D-studenterna genom kursen konstruktion med mikrodatorer och medlemmarna från U-programmet genom ett AI-projekt.
