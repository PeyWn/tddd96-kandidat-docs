\chapter{Diskussion}
Detta kapitel tar upp diskussioner om resultatet av projektet och den metod som använts.
\section{Resultat}
Saker som kan tas upp:
\begin{itemize}
\item Fanns alternativa implementationssätt?
\item Vad återstår för att kunden skall få ut fullt värde av produkten?
\item Lyckades ni förbättra/fortsätta något från tidigare projekt
\item Viktigaste lärdomar inför framtiden.
\end{itemize}

\section{Metod}
\emph{Vilka konsekvenser fick de valda metoderna för resultaten?} \\
\emph{Fanns det alternativ?}\\
\emph{Källkritik}\\

\textbf{Projektorganisation}

Rollerna i projektet var förutbestämda av kursen och kunde alltså inte förhandlas eller struktureras på ett annat sätt. Det var bra att rollerna var tydligt definierade och att de kom med vissa bestämda ansvarsområden. Det underlättade under projektets gång då det var tydligt vem som hade ansvar för att olika uppgifter blev utförda i tid.

Ett problem som upplevdes var dock att rollerna var strukturellt tyngre än projektet i sig. Det kändes med andra ord klumpigt och tungrott att använda sig av den rätta arbetsfördelningen baserat på de olika rollerna. Detta berodde förmodligen delvis på att alla utöver sina roller också behövde agera programmerare, designers, och andra roller som inte fanns specificerade från början.  

De möten som gruppen hade regelbundet var väldigt användbara det gällde alla olika mötestyper men på olika sätt. De korta daily-scrum-mötena var avgörande för att förmedla gruppens kortsiktiga framsteg, planering samt erfarenheter och problem. Vid vissa tillfällen gick lite för lång tid mellan två möten och då märktes det direkt i gruppens produktivitet.

Handledarmötena hade olika mycket nytta genom projektets gång. Generellt går det att säga att de var mer användbara precis innan och efter de olika inlämmningarna i projektet. Detta då de gav gruppen bra möjlighet att reda ut frågor inför inlämmningarna, samt få bra feedback efter dessa. 

Mellan inlämmningarna så var dessa möten ofta ganska korta och behandlade mest samma saker som kommit upp i de dagliga mötena. De fungerade då mer som ett tillfälle att rapportera status till handledaren.

Kundmöten som delar av gruppen hade med kunden och potentiella användare var mycket givande för projektets kravframtagning. Det som kom som lite av en överaskning var att kunden var väldigt bra på att få med flera olika kompetenser på mötena men de från gruppen som var där saknade iblan erfarheten att på ett effektivt sätt utnytja den existerande kompetensen. 

Trots detta så gjorde den stora variationen av de deltagandes kompetenser att det framkom många olika idéer och förslag som snabbt drev design-processen framåt. 

Gruppen hade relativt stora svårigheter att balansera dokumentationsarbetet med design och programmeringsarbetet. Framförallt lades det mycket tid på dockumentation i början utan att programmera parallelt och detta gjorde att programmeringen behövde ske på en kortare tid än vad som hade varit optimalt. 


\textbf{Förstudie}
\textbf{utvecklingsmetodik}
\textbf{Design}
\textbf{Utveckling}
\textbf{Utvärdering}
\textbf{källor}


\section{Arbetet i ett vidare sammanhang}
Inför valet i höst förväntar sig politiker att vårdfrågan kommer bli en av de största, en undersökning gjord av Novus 2017 visar att 67\% av de tillfrågade tyckte att vårdfrågan var en av de viktigaste.[källa]

På Universitetssjukhuset i Linköping jobbar över 5000 personer. Det är en stor organisation som är fördelade över en mängd olika kliniker. Sjukhus är i sin natur en väldigt rörlig miljö, där förutsättningar kan förändras mycket snabbt. När dessa förändringar sker gäller det att agera med en gång. Ett fel (som en dubbelbokning av en operationssal) kan få förödande konsekvenser. Sjukhus väljer därför ofta att vara mer konservativa än framåtskridande när det gäller att arbeta med nya system.

Idag så jobbar sjukhusen med något tekniskt sett äldre metoder för att t.ex. schemalägga tider för operationer. I dagsläget sitter mycket av kunskapen om hur man skapar ett effektivt schema i huvudet på skicklig personal. Men för scheman där specialistkunskap, operationssalar, verktyg narkos m.m måste koordineras kliniker emellan så kan det bli näst intill omöjligt för en mänsklig schemaläggare att hålla ordning på allt.

\begin{itemize}
\item Samhälleliga aspekter
\item Miljöaspekter
\item Etiska aspekter
\end{itemize}