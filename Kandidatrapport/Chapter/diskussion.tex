\chapter{Diskussion}
Detta kapitel tar upp diskussioner om resultatet av projektet och den metod som använts.
\section{Resultat}
Saker som kan tas upp:
\begin{itemize}
\item Fanns alternativa implementationssätt?
\item Vad återstår för att kunden skall få ut fullt värde av produkten?
\item Lyckades ni förbättra/fortsätta något från tidigare projekt
\item Viktigaste lärdomar inför framtiden.
\end{itemize}

\section{Metod}
Saker som kan tas upp:
\begin{itemize}
\item Vilka konsekvenser fick de valda metoderna för resultaten?
\item Fanns det alternativ?
\item Källkritik
\end{itemize}

\section{Arbetet i ett vidare sammanhang}
Inför valet i höst förväntar sig politiker att vårdfrågan kommer bli en av de största, en undersökning gjord av Novus 2017 visar att 67\% av de tillfrågade tyckte att vårdfrågan var en av de viktigaste.[källa]

På Universitetssjukhuset i Linköping jobbar över 5000 personer. Det är en stor organisation som är fördelade över en mängd olika kliniker. Sjukhus är i sin natur en väldigt rörlig miljö, där förutsättningar kan förändras mycket snabbt. När dessa förändringar sker gäller det att agera med en gång. Ett fel (som en dubbelbokning av en operationssal) kan få förödande konsekvenser. Sjukhus väljer därför ofta att vara mer konservativa än framåtskridande när det gäller att arbeta med nya system.

Idag så jobbar sjukhusen med något tekniskt sett äldre metoder för att t.ex. schemalägga tider för operationer. I dagsläget sitter mycket av kunskapen om hur man skapar ett effektivt schema i huvudet på skicklig personal. Men för scheman där specialistkunskap, operationssalar, verktyg narkos m.m måste koordineras kliniker emellan så kan det bli näst intill omöjligt för en mänsklig schemaläggare att hålla ordning på allt.

\begin{itemize}
\item Samhälleliga aspekter
\item Miljöaspekter
\item Etiska aspekter
\end{itemize}