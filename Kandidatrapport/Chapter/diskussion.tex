\chapter{Diskussion}
Detta kapitel tar upp diskussioner om resultatet av projektet och den metod som använts.
\section{Resultat}
Saker som kan tas upp:
\begin{itemize}
\item Fanns alternativa implementationssätt?
\item Vad återstår för att kunden skall få ut fullt värde av produkten?
\item Lyckades ni förbättra/fortsätta något från tidigare projekt
\item Viktigaste lärdomar inför framtiden.
\end{itemize}

\subsection{Val av Webbutvecklingsramverk}

Under designfasens inledande skede hölls ett möte med kunden där projektgruppen fick veta att kunden i tidigare webbutvecklingsprojekt använt sig av ramverket Angular. Att en stor organisation med gott rykte som Region Östergötland använt Angular tidigare ingav förtroende för Angular hos projektgruppen. Många av gruppens medlemmar saknade tidigare erfarenheter av webbutveckling. För att kompensera för detta såg gruppen möjligheter att dra nytta av kundens tidigare erfarenheter. Av dessa anledningar ansågs det passande att välja Angular som ramverk. Någon utförlig förstudie om vilket ramverk som bäst skulle passa gruppens behov utfördes inte.

Eftersom kunden inte ställt några uttryckliga krav på valet av ramverk hade andra val varit möjliga. Det finns ingen brist på webbutvecklingsramverk. Projektet hade säkert varit genomförbart även om något annat alternativ än Angular valts. Att utföra åtminstone en översiktlig utvärdering av Angular och en eller två konkurrerande webbutvecklingsramverk innan ett slutgiltigt beslut fattats kunde ha resulterat i färre problem under utvecklingsfasen.

Projektgruppen var vid projektets slutskede överlag nöjda med valet av Angular. De flesta ansåg att det varit svårt att sätta sig in i hur Angular fungerar och bör användas. När de väl lärt sig detta så ansåg samtliga att den komponentstruktur Angular tillfört varit till gagn för projektet. Den hade medfört modularitet mellan olika komponenter, vilket bland annat underlättat parallellt arbete på front-end. 

Att Angular har en ganska fix struktur var också en fördel för många av de mindre erfarna projektmedlemmarna. De hade oavsett val behövt sätta sig in i hur webbutveckling måste utföras. Att få riktlinjer från ramverket var alltså inte till någon nackdel för gruppen, även om det kanske irriterat mer erfarna webbutvecklare.

Sammanfattningsvis så var valet lyckat. Eftersom ingen förstudie angående val hade utförts så kunde det möjligtvis ha visat sig att Angular varit olämpligt för ett projekt med Aeons storlek. Att det inte blev så hade mer med tur att göra och mindre med gruppens förberedelser. Gruppen utnyttjade heller inte kunden kompetens av ramverket, vilket varit ett av skälen till valet. Hade det gjorts så hade inlärningsprocessen med all sannolikhet underlättats.  





\section{Metod}
\emph{Vilka konsekvenser fick de valda metoderna för resultaten?} \\
\emph{Fanns det alternativ?}\\
\emph{Källkritik}\\

\subsection{Projektorganisation}
Rollerna i projektet var förutbestämda av kursen och kunde alltså inte förhandlas eller struktureras på ett annat sätt. Det var bra att rollerna var tydligt definierade och att de kom med vissa bestämda ansvarsområden. Det underlättade under projektets gång då det var tydligt vem som hade ansvar för att olika uppgifter blev utförda i tid.

Ett problem som upplevdes var dock att rollerna var strukturellt tyngre än projektet i sig. Det kändes med andra ord klumpigt och tungrott att använda sig av den rätta arbetsfördelningen baserat på de olika rollerna. Detta berodde förmodligen delvis på att alla utöver sina roller också behövde agera programmerare, designers, och andra roller som inte fanns specificerade från början. 

En annan förklaring skulle kunna vara att projektets storlek inte passade överens med organisationen. Den skulle förmodligen ha fungerat bättre i ett större projekt med fler deltagare.

De möten som gruppen hade regelbundet var väldigt användbara, det gällde alla olika mötestyper men på olika sätt. De korta daily-scrum-mötena var avgörande för att förmedla gruppens kortsiktiga framsteg, planering samt erfarenheter och problem. Vid vissa tillfällen gick lite för lång tid mellan två av dessa möten och då märktes det direkt i gruppens produktivitet. För att uppnå maximal produktitvitet var det alltså nödvändigt att dessa möten hölls så ofta som möjligt.

Handledarmötena var till olika mycket nytta genom projektets gång. Generellt går det att säga att de var mer användbara precis innan och efter de olika inlämmningarna i projektet. Detta då de gav gruppen bra möjlighet att reda ut frågor inför inlämmningarna, samt få bra feedback efter inlämmningarna. 
Mellan inlämmningarna så var dessa möten ofta ganska korta och behandlade mest samma saker som kommit upp i de dagliga mötena. De fungerade då mer som ett tillfälle att rapportera status till handledaren.

Möten som delar av gruppen hade med kunden och potentiella användare var mycket givande för projektets kravframtagning. Det som kom som lite av en överaskning var att kunden var väldigt bra på att få med flera personer med olika kompetenser på mötena. De från gruppen som var där saknade ibland erfarenheten att på ett effektivt sätt utnyttja den existerande kompetensen. 
Trots detta så gjorde den stora variationen av de deltagandes kompetenser att det framkom många olika idéer och förslag som snabbt drev design-processen framåt. 

Gruppen hade relativt stora svårigheter att balansera dokumentationsarbetet med design och programmeringsarbetet. Framförallt lades det mycket tid på dokumentation i början utan att programmera parallelt. Detta gjorde att programmeringen behövde ske på en kortare tid än vad som hade varit optimalt. Den primära anledningen till detta var förmodligen att gruppen saknade erfarenhet av projekt med denna nivå av dokumentationskrav. Det system som fanns för att sköta dokumentation fungerade oftast mycket bra och framförallt så var det väldigt bra att versionshantera slutrapporten i git då detta öppnade upp möjligheten att jobba med grenar i git på samma sätt som gruppen gjorde under utvecklingsarbetet med programmet.

\subsection{Utvecklingsmetodik}
Utvecklingsmetodiken Scrum var gruppen inne på ganska tidigt som ett alternativ men bestämde sig inte omgående. Det framkom dock efter ett av de tidigare mötena med kunden att de hade god kunskap och erfarenhet av metoden och därför valde gruppen då att ta till sig Scrum. De regelbunda dagliga mötena har skapat en bra informationsspridning av status, kunskap och planering. De längre och mer utförliga mötena för demo, utvärdering och planering har bidragit med bra möjligheter att styra projektet framåt samtidigt som erfarenheter av vad som fungerat bra och mindre bra snabbt har kunnat fångas upp och bidragit med lärdomar till gruppen.

\subsection{Förstudie}
Förstudien lyckades lägga en mycket bra grund för projektet. Denna fas var även den fas där gruppen fick mest tips och idéer från kursmaterialet. Det som kunde gjorts bättre från gruppen var att snabbare bryta sig ur den strikt teoretiska delen av förstudien och börja omsätta designskisser och krav i kod tidigare för att få ett visst försprång in i utvecklingsfasen. Det fanns ett flertal beslut gällande arkitektur och teknikval som var klara så tidigt att dessa skulle kunnat konfigureras tidigare. Den bild som uppstod av produkten under förstudien är i stort sett den bild som programmet i slutet avbildar. Dock har det under utvecklingen framkommit fler krav som tyvärr inte hinner implementeras men som skulle tillföra stora delar av önskad funtionallitet som inte lyckades identifieras tillräckligt tidigt i processen.

\subsection{Design}
De största dragen i hur designen skulle se ut, hur flödet i gränssnittet skulle fungera och hur data skulle visualiseras togs fram genom enkla LoFi-prototyper som har beskrivits i denna raport. Dessa var mycket enkla att skapa och förändra efter feedback. Gruppen valde medvetet att inte prata om olika designalternativ innan alla hade skissat sina egna prototyper. Detta gjordes för att inga idéer skulle försvinna utan att alla skulle få en chans att tänka själva. Prototyperna visades i ett par omgångar för kunden. Detta gick mycket bra. Senare under projektet visades även prototypen och sedemera även den implemeterade designen för några operationsplanerare. Då dök det upp ytterligare designidéer som vi tyvärr inte hann implementera. Det hade därför varit bra att kunna komma åt den potentiella slutanvändaren tidigare i processen.


\subsection{Utveckling}
Den delen av gruppen som utvecklade front-end famlade lite i mörker då ingen av dem hade tidigare erfarenhet av Typescript eller Angular. Av denna anledning samarbetade de mycket, framförallt i början. Detta visade sig mycket effektivt och gruppen visade snabbt upp bra resultat i arbetet. Att jobba så mycket gemensamt skapade också en kreativ och problemlösande miljö där problem och funderingar i hur gränssittet skulle designas eller se ut snabbt kunde redas ut och lösas med hög kvalitet. Ju längre arbetet gick och ju större del av skeletet av applikationen som fanns på plats ju mer självständigt jobbade de som utvecklade denna del. Det gjorde att arbetet gick ännu mer effektivt.

Att börja med att skapa ett EER-diagram för databasen innan back-end-delen började utvecklas var mycket användbart. Den delgrupp som satt och utvecklade back-end-delen återgick ofta till detta för att verifiera att kod var skriven på rätt sätt och att den följde den ursprungliga tanken. Utöver EER-diagramet skrevs också ett dokument som definerade alla anropsvägar till det API som skapades. Detta fyllde ett liknande syfte som EER-diagrammet och var även det ett flitigt använt referensdokument. Utöver att utvecklarna hade nytta av dessa dokument så kommer de även vara nyttiga för dem som kommer använda koden och fortsätta utveckla projektet i framtiden.

En av anledningarna att det gick så bra att skapa hållbar referensdokumentation var att två av de som utvecklade back-end-delen hade gått en kurs i databashanteringssystem terminen innan detta projekt.

Back-end-delen i projektet är i det stora hela helt självständig och skulle i teorin kunna kopplas till vilken front-end som helst. Detta var ett önskemål från kunden men det har också underlättat utvecklingsarbetet avsevärt då gruppen lättare har kunnat dela upp sig för att jobba på olika delar av projektet.

Över all utveckling i projektet har Gitlab verkligen hjälpt till att hålla en hög kodkvalitet. Detta genom att alla har jobbat på nya funktioner i nya grenar i Git. När koden har blivit klar för att implementeras i det stora projektet har det varit mycket enkelt med verktygen i Gitlab att låta någon annan i projektet granska koden innan den har lagts in i den gemensamma kodbasen.

\subsection{Utvärdering}
Som tidigare nämnts så har gruppen jobbat efter Scrum-metodiken och den har väl definerade processer för att utvärdera erfarenheter efter varje sprint i projektet. Detta har gjort att gruppen regelbundet har utvärderat hur arbetet faktiskt har gått. På det sättet har arbetet kunnat förbättras genom att rätta till det som inte fungerat och hålla kvar vid det som fungerat. Det som gjorde detta extra tydligt var att välja ut några få saker som skulle förbättras i nästa sprint och sedan utvärdera hur det gick. Utan denna process hade det varit avsevärt mycket svårare att utveckla och förbättra arbetet under tiden som projektet pågick.

\subsection{Källor}


\section{Arbetet i ett vidare sammanhang}
Inför valet i höst förväntar sig politiker att vårdfrågan kommer bli en av de största, en undersökning gjord av Novus 2017 visar att 67\% av de tillfrågade tyckte att vårdfrågan var en av de viktigaste.[källa]

På Universitetssjukhuset i Linköping jobbar över 5000 personer. Det är en stor organisation som är fördelad över en mängd olika kliniker. Sjukhus är i sin natur en väldigt rörlig miljö, där förutsättningar kan förändras mycket snabbt. När dessa förändringar sker gäller det att agera med en gång. Ett fel (som en dubbelbokning av en operationssal) kan få förödande konsekvenser. Sjukhus väljer därför ofta att vara mer konservativa än framåtskridande när det gäller att arbeta med nya system.

Idag så jobbar sjukhusen med något tekniskt sett äldre metoder för att t.ex. schemalägga tider för operationer. I dagsläget sitter mycket av kunskapen om hur man skapar ett effektivt schema i huvudet på skicklig personal. Men för scheman där specialistkunskap, operationssalar, verktyg narkos m.m måste koordineras kliniker emellan så kan det bli näst intill omöjligt för en mänsklig schemaläggare att hålla ordning på allt.

\begin{itemize}
\item Samhälleliga aspekter
\item Miljöaspekter
\item Etiska aspekter
\end{itemize}