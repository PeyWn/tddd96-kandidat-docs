\chapter{Prototyputveckling i ett kandidatprojekt av Niclas Byrsten}\label{appendix:prototyp}

\section{Inledning}
Innan ett mjukvaruprojekt kommer in i själva utvecklingsfasen så är det viktigt att både kunden och projektmedlemmarna har samma bild av hur slutproduktens grafiska gränssnitt ska vara utfoemat. Det finns många olika metoder och verktyg som kan vara till hjälp där en av dessa är prototypning.  

\section{Syfte}
Syftet med den här rapporten är att beskriva hur arbetsgången med prototyper utfördes av projektgruppen under och vilka slutsatser som kan dras utifrån det.  

\section{Frågeställningar}
De frågeställningar som rapporten ska besvara är:
\begin{enumerate}
    \item Vilka fördelar och nackdelar finns det med prototypning?
    \item Hur effektivt är prototypning vid ett mindre projekt?
\end{enumerate}

\section{Bakgrund}
Ett inte allt för ovanligt problem som finns då ett gränssnitt ska tas fram är att det kan uppstå problem i kommunikationen mellan projektgruppen som ska utveckla systemet och övriga aktörer med intresse i projektet såsom beställare och slutanvändare. En prototyp används vanligen för att dela ideer om designen men även för att överbrygga detta gap som kan finnas mellan tidigare nämnda grupper \cite{caseStudy}. Den kan ses som ett hjälpmedel eller verktyg för att säkerhetsställa att det system som ska tas fram stämmer överens med de krav och förväntningar som kan finns på systemet. På ett tydligt sätt så går det att hur projektgruppen föreställer sig att att slutprodukten ska se ut och det går i ett tidigt stadie ge både åsikter och synpunkter. På samma gång så kan det även hjälpa projektgruppen att få sig en tydlig bild av hur den grafiska designen kommer att bli utformad. 

\section{Avgränsningar}
Denna rapport kommer endast belysa erfarenheter och slutsatser som kan dras med utgångspunkt från det här projektet.

\section{Teori}
Nedan så presenteras den teori som denna rapport bygger på.

\section{Prototyp}
En prototyp är en arbetsmodell vars syfte är utveckla och testa design ideer. Två typer av prototyper som är vanligt förekommande är low-fidelity och high-fidelity. I den efterföljande texten så  kommer dessa att benämnas som lo-fi prototyp respektive hi-fi prototyp. Uttrycket fidelity beskriver hur prototypen skiljer sig ifrån och hur väl den representerar den slutgiltiga produkten \cite{prototypeChoise}.Den kan sägas vara en simulering av slutprodukten.     

\subsection{lo-fi prototyp}
En lo-fi prototyp skiljer sig ifrån den slutgiltiga produkten på punkter som detaljrikedom, interaktion och utseendemässigt. De kan vara gjorda av papper och de största fördelarna är att de är snabba att skapa, enkla att vidareutveckla  och de kan flytta fokus till själva interaktionsdesignen istället för detaljer och visuell styling \cite{prototypeChoise}.

\subsection{hi-fi prototyp}
En hi-fi prototyp är mer lik slut produkten.  I jämförelse med en lo-fi prototyp så erbjuder en hi-fi prototyp mer realistiska interaktioner, fler detaljer, är mera lik slutprodukten och förmedlar bättre vilka designmöjligheter som kan erbjudas. De är dock mer kostsamma och tar längre tid att konstruera \cite{prototypeChoise}.

\section{Val av Prototyp} 
Under projektet har gruppen tagit fram ett flertal pappersprototyper där vissa  har demonstrerats för kunden. I ett tidigt skede så hade kunden framfört önskemål om att projektgruppen skulle använda sig av pappersprototyper. Gruppens medlemmar hade tidigare läst en kurs i interaktiva system där de bland annat gjorde en pappersprototyp som sedan vidareutvecklades till en hi-fi prototyp med hjälp av digitala mjukvaruverktyg. Men ingen hade erfarenhet av att gå ifrån pappersprototyp till färdig produkt. Att använda sig av pappersprototyper tidigt innan någon kod har skrivits är både billigare och mer tidsbesparande än att göra ändringar senare \cite{paper}. Under detta projekt så fördes även diskussion om gruppen skulle göra en hi-fi prototyp med något digital verktyg innan själva utvecklandet av produkten tog sin början. Men det var inget som gjordes då det upplevdes räcka med pappersprototyper och att det inte var värt den tidsinvestering som krävts.      

\section{Förberedelse}
Efter ett av de första mötena med kunden, där kravspecifikationen togs fram, så hade gruppen ett möte om hur den grafiska designen för produkten  skulle se ut. Då ingen i projektgruppen hade någon riktigt klar bild av hur den skulle vara utformad så kom gruppen överens om att varje person skulle skissa på en egen prototyp av hur det grafiska designen skulle kunna se ut. Tanken med detta var att kunna få fram så många olika förslag som möjligt då. Varje förslag visades för upp för resten av gruppen som kom med frågor och gav sina tankar om den. Efter alla designförslag hade presenterats så hölls en diskussion om var och en av föslagen, där det tillslut valdes ut två stycken förslag som förbättrades och skulle visa på olika alternativ på hur gruppen tänkte sig att den slutgiltiga versionen skulle kunna se ut. Ett nytt möte bokades in med kunden för att få återkoppling. Utöver dessa två så togs det även med en tredje prototyp som hade gjorts efter diskussion tillfället som inte var helt fullständig men som kunde visa olika alternativ på enskilda komponenter i gränssnittet.

\section{Första mötet}
På mötet så närvarade fem stycken personer från kundens sida och fyra stycken från projektgruppens sida. Inledningsvis så introducerade gruppen prototyperna och berättade hur de föreställde sig att den skulle se ut.  De två första prototyperna testades genom att en av dessa fem fick agera testperson med uppgift att boka in en operation samtidigt som testpersonen berättade vad den såg och hur. Under mötet så fördes även en fortlöpande diskussion om vad som var bra och vad som borde förbättras. När tillfälle fanns så gav även de övriga personerna som var närvarande sina åsikter och synpunkter. Många av de frågetecken som fanns innan reddes ut och det blev mer tydligt vad kunden hade för förväntningar. 

\section{Andra mötet}
Efter det  så samlades gruppen för att diskutera vad som hade framkommit under föregående möte och hur nästa prototyp skulle byggas. Några av dessa punkter  presenteras nedan:
\begin{enumerate}
	\item Det ska finnas en bekräftelseruta med en total sammanfattning över alla detaljer i bokningen innan man 	 			  	  bekräftar bokningen.
 	\item Det ska stå namn på patienten i listan över beslut, eller åtminstone initialer.
 	\item Det är ett intressant koncept att visa detaljer om en bokning i form av en tidslinje
 	\item Under översikten över lediga tider kan någon form av spårsystem vara intressant att utveckla.
\end{enumerate}
Två personer blev ansvariga för konstruktionen som visades upp för de andra innan den ansågs redo. Denna gång skulle den presenteras för tre personer som i skrivande stund arbetar som schemaläggare på tre olika avdelningar. Det sågs som positivt att testa på fler slutanvändare då de kan ha olika syn på vad som borde finnas med och hur det ska se ut. Från projektgruppen närvarade tre personer. Inledningsvis så presenterades prototypen och hur gruppens tankar om hur den skulle fungera. Efter det så framförde schemaläggarna sina tankar och åsikter. Precis som under första mötet hölls en kontinuerlig diskussion.
Återigen så hölls ett nytt gruppmöte om det som framkommit och hur prototypen skulle förbättras. Gruppen ansåg att det fanns en tillräckligt bra grund för att börja implementera designen av det grafiska gränssnittet.      
  
\section{Metod}
Den här rapporten bygger på författarens egna upplevlser och diskutioner som har skett inom gruppen. Vidare så gjordes en intervju med gruppens medlemmar för att få en samlad bild av gruppens erfarenheter och synpunkter. Nedan listas de frågor som ställdes:
\begin{enumerate}
	\item Vad tycker du har gått bra med att använda sig av prototyper?
 	\item Är det något du tycker har fungerat mindra bra?
 	\item Vad det värt den tid det tog?
 	\item Hur har du haft användning av prototypen i projektet?
 	\item Har det hjälpt dig att få en bättre uppfattning hur det grafiska ska se ut?
 	\item Skulle du använda dig av prototyper i ett liknande projekt i framtiden? 
\end{enumerate}
 
\section{Resultat}
Utifrån de intervjuerna så erhölls några intressanta svar. Bland annat så skulle alla medlemmar använda sig av prototyper i ett liknande projekt i framtiden och det var värt den tid det tog. Alla utom en tycker att de har haft nytta av den i projektet. De fördelar som nämndes var exempelvis att prototypen var snabb och enkel att göra, många olika ideer fångades upp i gruppen. Bland det negativa finns saker som att den inte var så interaktiv och att det kan vara svårt att visa på vad som är tryckbart. Utöver det så upplevde gruppmedlemmarna att prototypen varit till hjälp vid implementatione av grafisk komponenter. Svaren på alla frågor återfinns i appendix J.  
  
\section{Diskussion}
Under projekt testades fyra pappersprototyper mot tilltänkta slutanvändare. Under dessa tillfällen så användes prototypen för att tydliggöra gruppens bild av den grafiska designen. Den återkoppling som erhölls blev klarare och möjliggjorde en bättre kommunikation. I ett tidigt skede kunde gruppen åtgärda designfel som annars kan ha missats eller upptäckts först när implementationen hade börjat. De oklarheter som fanns i början blev tyligare. Under projektes gång så gick det alltid att blicka tillbaka på prototypen.
Överlag så verkar alla gruppmedlemmar vara nöjda prototypningen. Vissa saker hade kunnat gjorts annorlunda.  

Tidsåtgången för konstruktionen kan upplevas negativt.

\section{Slutsats}

Källor

1.A Case Study of How Interface Sketches, Scenarios and Computer Prototypes Structure Stakeholder Meetings Maria Johanson and Mattias Arvola Hämtad 4 maj 2018  

2. Walker, M., Takayama, L., and Landay, J. A. High-fidelity or low-fidelity, paper or computer? Choosing attributes when testing web prototypes. Hämtad 4 maj 2018  

3. Nielsen Jakob 2003 
ahttps://www.nngroup.com/articles/paper-prototyping 