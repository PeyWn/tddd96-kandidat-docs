\chapter{Prototyping i ett kandidatprojekt - Niclas}
\section{Inledning}
För att kunna säkerhetsställa att det system som ska tas fram stämmer överens med de krav och förväntningar kunden har på systemet så kan en prototyp tas fram. Med hjälp av prototyper så kan beställaren på ett tydligt sätt se hur projektgruppen föreställer sig slutprodukten och även ge sina synpunkter på den. På samma gång så hjälper det även utvecklarna att få sig en tydlig bild av hur den grafiska designen kommer att vara utformad. 

\section{Syfte}
Syfte med den här rapporten är att beskriva hur prototyping användes av projektgruppen under utvecklingsfasen och vilka slutsatser som kan dras utifrån det.   

\section{Frågeställning}
De frågeställningar som rapporten ska besvara är:
\begin{enumerate}
	\item 1. Vilka fördelar och nackdelar finns det av att använda sig av prototyping?
	\item 2. Hur effektivt är det att använda sig av prototyping vid ett mindre projekt?
\end{enumerate}

\section{Definitioner}
Under projektet har gruppen tagit fram ett flertal prototyper som har demonstrerats för kunden. Redan i ett tidigt skede klargjorde kunden sina önskemål om att pappersprototyper skulle användas. Under iteration 1 så tog varje person fram en pappersprototyp och efter en diskussion i gruppen så valdes två ut som sedan presenteras för kunden.  Under iteration 2 så togs en ny prototyp fram som baserades på den tidigare återkopplingen från kunden och presenterades denna gång för tre personer på olika avdelningar som gav sina synpunkter och kommentarer på det framtagna.  

\section{Avgränsningar}
\section{Bakgrund}
\section{Teori}
\section{Metod}
Jag tänkte samla information från Liu:s biblioteks hemsida men även från internet. 
\section{Resultat}
\section{Diskussion}
\section{Slutsats}