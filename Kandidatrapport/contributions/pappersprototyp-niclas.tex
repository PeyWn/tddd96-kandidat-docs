\chapter{Prototyputveckling i ett kandidatprojekt - Niclas}
\section{Inledning}
Innan ett mjukvaruprojekt kommer in i själva utvecklingsfasen så är det viktigt att både kunden och projektmedlemmarna har samma bild av hur slutproduktens grafiska gränssnitt ska se ut. 

\section{Syfte}
Syfte med den här rapporten är att beskriva hur prototyping användes av projektgruppen under utvecklingsfasen och vilka slutsatser som kan dras utifrån det.   

\section{Frågeställning}
De frågeställningar som rapporten ska besvara är:
\begin{enumerate}
	\item Vilka fördelar och nackdelar finns det av att använda sig av prototyping?
	\item Hur effektivt är det att använda sig av prototyping vid ett mindre projekt?
\end{enumerate}

\section{Bakgrund}
Ett inte allt för ovanligt problem som finns då ett gränssnitt ska tas fram är att det kan uppstå problem i kommunikationen mellan projektgruppen som ska utveckla systemet och övriga aktörer med intresse i projektet såsom beställare och slutanvändare. En prototyp används vanligen för att dela ideer om designen men även för att överbrygga detta gap som kan finnas mellan tidigare nämnda grupper[1]. Den kan ses som ett hjälpmedel eller verktyg för att säkerhetsställa att det system som ska tas fram stämmer överens med de krav och förväntningar som kan finns på systemet. På ett tydligt sätt så går det att hur projektgruppen föreställer sig att att slutprodukten ska se ut och det går i ett tidigt stadie ge både åsikter och synpunkter. På samma gång så kan det även hjälpa projektgruppen att få sig en tydlig bild av hur den grafiska designen kommer att bli utformad. Något som kan vara svårt är att veta är vilken typ av prototyp som lämpar sig bäst för ett specifikt fall.

\section{Prototyp} 
Två typer av prototyper som är förekommande är low-fidelity och high-fidelity. I den efterföljande texten så  kommer dessa att benämnas som lo-fi prototyp respektive hi-fi prototyp. Uttrycket fidelity beskriver hur prototypen skiljer sig ifrån och hur väl den representerar den slutgiltiga produkten.[2]  

\subsection{lo-fi prototyp} 
En lo-fi prototyp skiljer sig ifrån den slutgiltiga produkten på punkter som detaljrikedom, interaktion och utseendemässigt. De kan vara gjorda av papper och de största fördelarna är att de är snabba att skapa, enkla att vidareutveckla  och de kan flytta fokus till själva interaktionsdesignen istället för detaljer och visuell styling. [2]

\subsection{hi-fi prototyp}
I jämförelse med en lo-fi prototyp så erbjuder en hi-fi prototyp mer realistiska interaktioner, fler detaljer, är mera lik slutprokten och förmedlar bättre vilka designmöjligheter som kan erbjudas. [2] 

\section{Utförande} 
Efter ett av de första mötena med kunden, där kravspecifikationen togs fram, så hade gruppen ett möte om hur det grafiska gränssnittet skulle se ut. Då ingen i projektgruppen hade någon riktigt klar bild av hur den skulle utformas så kom gruppen överens om att varje person skulle skissa på en egen prototyp av hur det grafiska gränssnittet skulle kunna se ut. Tanken med detta var att kunna få fram så många olika förslag som möjligt. Varje förslag visades för upp för resten av gruppen som kom med frågor och gav sina tankar om den. Efter alla designförslag hade presenterats så hölls en diskussion om var och en av föslagen, där det tillslut valdes ut två stycken förslag som förbättrades och skulle visa på olika alternativ på hur gruppen tänkte sig att den slutgiltiga versionen skulle kunna se ut. Ett nytt möte bokades in med kunden för att få återkoppling. Utöver dessa två så togs det även med en tredje prototyp som hade gjorts efter diskussion tillfället som inte var helt fullständig. Anledningen till det var för att visa upp lite olika alternativ på enskilda komponenter i gränssnittet. På mötet så närvarade fem stycken personer från kundens sida. Inledningsvis så introducerade gruppen prototyperna och berättade hur de föreställde sig att den skulle se ut.  De två första prototyperna testades genom att en av dessa fem fick agera testperson med uppgift att boka in en operation samtidigt som testpersonen berättade vad den såg och hur den . Under mötet så fördes även en fortlöpande diskussion om vad som var bra och vad som borde förbättras. När tillfälle fanns så gav även de övriga personerna som var närvarande sina åsikter och synpunkter. Många av de frågetecken som fanns innan reddes ut och det blev mer tydligt vad kunden hade för förväntningar.  

Utifrån det som kom fram ifrån detta mötet så skapades en ny prototyp som skulle testas. Denna gång var det för tre personer som i skrivande stund arbetar som schemaläggare på olika avdelningar och då det främst är de som kommer använda sig av  på så vis kan ses som representativa framtida användare av det slutgiltiga systemet.    


\section{Avgränsningar}


\section{Definitioner}
  

\section{Teori}
\section{Metod}
Jag tänkte samla information från Liu:s biblioteks hemsida men även från internet. 
\section{Resultat}
\section{Diskussion}
\section{Slutsats}