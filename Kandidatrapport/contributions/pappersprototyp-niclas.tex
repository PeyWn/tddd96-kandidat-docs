\section{Pappersprototypens betydelse vid utveckling av programvara - Niclas}
För att kunna säkerhetsställa att det system som ska tas fram stämmer överens med de krav och förväntningar kunden har på systemet så kan en pappersprototyp tas fram. I en tidig del av projektet så kan beställaren på ett tydligt sätt se hur projektgruppen föreställer sig slutprodukten och även ge sina synpunkter på den. På samma gång så hjälper det även utvecklarna att få sig en tydlig bild av hur den grafiska designen kommer att vara utformad.
