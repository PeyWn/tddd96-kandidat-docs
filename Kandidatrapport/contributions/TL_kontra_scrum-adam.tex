\chapter{Teamledarens roll kombinerad med Scrum-metodik av Adam Andersson}

\section{Inledning}
Scrum-metodik är ett välkänt arbetssätt som ofta används inom mjukvaruutveckling. I en grupp som använder sig av Scrum-metodik finns det ingen specifik ledare, om man bortser från Scrummästaren som har ansvaret för att se till att Scrum-metodiken används, då alla är utvecklare och gruppen tillsammans bestämmer vad som ska göras och hur problem ska lösas.
I detta projekt har alla gruppmedlemar en projektroll där en person innehar rollen som teamledare, för projektroller i detta projekt se avsnitt \ref{roller}. Denna studie ska utreda hur rollen som teamledare fungerar i kombination med Scrum-metodik.

\subsection{Syfte}
Rollen som teamledare innebär att coacha projektgruppen och se till att de mål som finns för projektet uppfylls. Enligt Scrum-metodiken är det gruppen som tillsammans ska bestämma över de beslut som tas och hur problem ska lösas. Syftet med denna studie är att utreda om det går att kombinera rollen som teamledare med Scrum-metodiken.

\subsection{Frågeställningar}
Utefter det syfte som denna studie har ska följande frågor försöka besvaras:

\begin{enumerate}
	\item Hur ser teamledarrollen ut?
	\item Hur fungerar Scrum-metodik?
	\item Går det att kombinera teamledarrollen i en Scrum-metodik och fortfarande bibehålla syftet med de båda? 
\end{enumerate}

\section{Bakgrund}
I början av projektet skulle gruppen tilldela samtliga medlemmar en roll, däri fanns rollen teamledare som författaren av denna studie tog på sig. När projektet hade fortgått i några veckor togs det upp vilken metodik gruppen skulle använda för projektet. Kunden föreslog att vi kunde använda oss av Scrum. Teamledaren läste på om denna metodik och höll en redovisning för gruppen. Efter redovisningen diskuterades om detta verkade vara en bra metod. Då Scrum är en välkänd agil utvecklingsmetodik med enkla och strukturerade riktlinjer tog gruppen beslutet att använda sig av denna. När gruppen började använda sig av metodiken började jag fundera över hur detta skulle påverka rollen som teamledare då teamledaren är en typ av ledarroll men enligt Scrum är det gruppen som tar beslut om vad som ska göras och hur problem ska lösas.

\section{Teori}
Detta avsnitt går igenom den teori som finns till grund för denna studie.

\subsection{Scrum} \label{adam_scrum}
Scrum är en arbetsmetodik som främst används för mjukvaruutveckling. Metodiken appliceras på mindre team (3-9 personer) och går ut på att teamet arbetar i tidsbestämda sprintar där det existerar ett antal uppgifter som ska genomföras under respektive sprint.
I Scrum existerar ett fåtal specifika roller som används för att beskriva de inblandade i projektet. De roller som existerar är:

\begin{itemize}
	\item \textbf{Produktägare}
	
	Produktägaren är den person som representerar beställaren, vanligtvis aktieägarna i större bolag, och förmedlar till utvecklingsteamet vilka krav som finns och vilka av dessa som har högst prioritet. En viktig egenskap för en produktägare är empati, du arbetar med aktieägare som ofta har olika bakgrund och värderingar. Produktägaren måste också vara bra på att samarbeta med utvecklarna så att arbetet med produkten går bra och att krav uppfylls.
	
	\item \textbf{Scrummästare}
	
	Scrummästaren skiljer sig från den klassiska teamledaren eller projektledaren. Dess roll är att se till att det inte existerar några hinder för utvecklingsteamet så att de kan uppnå projektmål och leverera en produkt. En annan viktig uppgift för Scrummästaren är att se till att de processer som valts att applicera i Scrum-metoden utförs och efterföljs. Samarbete med produktägaren finns också för att underhålla backloggen så att utvecklingsteamet förstår vad som ska göras och så att arbetet med produkten kan fortgå.
	
	\item \textbf{Utvecklingsteam}
	
	Utvecklingsteamet är själva projektgruppen och den ansvarar för att göra framsteg i arbetet och utföra de uppgifter som existerar i respektive sprint. Trots att det kan finnas olika typer av roller så kallas alla i teamet för utvecklare. Då benämningen utvecklare ibland kan antas referera till enbart programmerare så har vissa organisationer valt att kalla utvecklingsteamet för leveransteam och personerna i teamet för teammedlemmar.	
\end{itemize}

Utöver de roller som existerar i Scrum använder man sig också av fyra olika mötestyper som inträffar med olika frekvens. Det viktiga med dessa möten är att de är tidsbestämda och den maximala tid som ska läggas på dessa beror i flera fall på hur lång en sprint i projektet är, detta för att man inte ska lägga alltför lång tid på möten. De mötestyper som existerar är:

\begin{itemize}
	\item \textbf{Sprintplanering (maximalt 4 timmar för en tvåveckors-sprint)}
	
	Sprintplaneringen är uppdelad i två delar, den första delen går ut på att produktägaren, Scrummästaren och utvecklingsteamet väljer de uppgifter från backloggen som man tror kommer hinnas med under den kommande sprinten. När detta är gjort kommer planeringens andra del, utvecklingsteamet diskuterar och kommer mer specifikt fram till vilka arbetsuppgifter som behöver göras för att uppgifterna från backloggen ska kunna utföras och uppfyllas. Den andra iterationen över uppgifterna kan leda till att man delar upp vissa uppgifter eller att vissa uppgifter läggs tillbaka i backloggen då man inte tror att dessa kommer hinnas med under kommande sprint.
	
	\item \textbf{Daglig Scrum (maximalt 15 minuter)}
	
	Varje dag hålls ett dagligt Scrum-möte för att uppdatera utvecklingsteamet om vilka framsteg och hinder som finns i sprinten så att teamet kan öka sina chanser att nå målen med nuvarande sprint. Dessa möten är tidsbestämda till 15 minuter och de kan utföras stående i en ring, så att man inte fokuserar på annat och så att man håller sig inom de 15 minuterna. Man kan utgå ifrån tre frågor under en daglig Scrum:
	
	\begin{itemize}
		\item Vad har jag gjort sedan igår?
		\item Vad ska jag åstadkomma till imorgon?
		\item Vad hindrar mig?
	\end{itemize}
	
	Om det skulle komma fram att det finns några hinder i gruppen ska Scrummästaren notera detta och man bör sätta en person på att lösa problemet så att sprinten kan fortgå utan hinder.
	
	\item \textbf{Sprintgenomgång (maximalt 2 timmar för en tvåveckors-sprint)}
	
	När en sprint är avslutad hålls en sprintgenomgång, där man tittar på vilka arbetsuppgifter utvecklingsteamet har, och inte har, hunnit med att slutföra. Man håller också en demo för beställaren där man visar vilka framsteg som har gjorts under sprinten. Efter genomgång av utförda och icke utförda arbetsuppgifter samt demo diskuterar utvecklingsteamet och beställaren vad teamet ska arbeta med härnäst.
	 
Det som är viktigt att tänka på under en sprintgenomgång är att man inte ska visa upp arbetsuppgifter som inte är färdiga, t.ex. ska man inte visa upp funktionalitet i ett program om funktionaliteten inte är helt färdig.

\item \textbf{Sprintåterblick (maximalt 90 minuter för en tvåveckors-sprint)}

När en sprintgenomgång har utförts är det dags för en sprintåterblick. Sprintåterblick innebär att man utvärderar den sprint som har varit genom att utgå från två frågor:

\begin{itemize}
	\item Vad gick bra under sprinten?
	\item Vad kan förbättras till nästa sprint?
\end{itemize}

Från dessa två frågor identifierar man förbättringar som kan göras. Man väljer ut några av dessa punkter och arbetar med dem i nästa sprint.

\end{itemize}

För att Scrum ska fungera så effektivt som möjligt behöver man en grupp som är geografiskt närvarande så att man får känsla av tillhörighet och har möjlighet att närvara vid dagliga Scrum-möten. En annan viktig punkt är att medlemmarna i gruppen måste vara ärliga mot varandra, om man har några hinder i sitt arbete är det viktigt att detta kommer upp på det dagliga Scrum-mötet. Detta så att Scrummästaren eller utvecklingsteamet kan röja undan hindret så att arbetet kan fortsätta.\cite{scrum}

\subsection{Teamledarrollen}
Teamledaren är den person som ska coacha teamet mot de mål som är uppsatta. Teamledaren är lyhörd och ser till att alla känner sig delaktiga i processen. Teamledaren är inte en projektledare och kan därför ta ett steg tillbaka i ledarrollen för att få medlemmar i gruppen att ta plats och utvecklas.\cite{teamledare} Trots att teamledaren kan ta ett steg tillbaka är personen ändå närvarande i gruppen och ser till att ha täta avstämningar för att ha koll på vad som händer och för att skapa samhöriget inom gruppen.\cite{teamguide}

\section{Metod}
För att genomföra denna studie har tre olika metoder använts. Informationssökning genom att leta efter vetenskapliga artiklar som analyserar Scrum-metoden, egna erfarenheter genom att analysera hur arbetet i projektgruppen har fortgått och intervjuer av medlemmar i andra projektgrupper som också har använt sig av Scrum-metoden.

\subsection{Informationssökning}
Det finns en artikel som handlar om att förstå och arbeta med delat ledarskap i agil utveckling.

I den studien har man följt ett företag som nyligen har börjat använda sig av scrum-metoden för att se hur det fungerar i praktiken att dela upp ledarskap bland alla inblandade parter. Det man kunde observera var att medlemmarna i det projekt som observerades inte kunde lägga så mycket tid på projektet som man planerat, detta för att de var involverade i andra projekt som var beroende av dem. Ett annat problem man kunde se var att det saknades kompetens eller vilja att ta ledarrollen, en roll som ska roteras i Scrum till den som har mest kompetens. \cite{sharedleader}

\subsection{Egna erfarenheter}
Under projektet som genomförts hade jag rollen som både teamledare och Scrummästare. Det gav både erfarenhet i att leda en grupp i ett projekt och att hålla koll på de olika processer som används inom Scrum.

Då en teamledare inte ska agera projektledare och eftersom det är väsentligt i Scrum-metodik att det är utvecklingsteamet som bestämmer valde jag att angripa en mer passiv ledarroll. Jag bokade lokaler till gruppen, kallade till möten, höll övergripande koll på mål och deadlines och ledde diverse möten gruppen hade som t.ex. daglig Scrum, sprint planering och handledarmöten. Men när det kom till beslut och planering arbetade jag för att gruppen skulle föra diskussion och komma fram till vad som skulle göras. När personer med rätt kompetens behövdes lät jag dem kliva fram och leda, detta för att få fram bäst resultat och för att följa Scrum.

\subsection{Intervjuer}
Intervjuer har utförts på ett antal studenter som läser kursen TDDD96. För att få in ett bredare spektrum av erfarenheter om hur det är att arbeta med Scrum och en teamledare har intervjuerna utförts på personer från olika projektgrupper.

Intervjuerna gick till på det sättet att jag presenterade den studie jag arbetade med och sedan ställde jag ett antal frågor. Till att börja med ställde jag frågor om vilken roll de hade i projektet, sedan följde frågor om varför de valt Scrum, om personen var Scrum-master och hur det fungerat med att upprätthålla dessa processer. Slutligen så ställde jag frågor om teamledarrollen, hur den tillämpats i projektet och hur det har gått att kombinera den med Scrum-metodik. De frågor som ställdes till till de som intervjuades var:

\begin{enumerate}

\item Vilken roll har du i projektet?

\item Hur kommer det sig att ni valde att arbeta med Scrum?

\item Vilka delar av Scrum har ni tillämpat?

\item Vilken roll i er projektgrupp är Scrummästare?

\item Varför valde ni just den rollen som Scrummästare?

\item Hur har det gått med att upprätthålla Scrum-processerna?

\item Vilka uppgifter har teamledaren i er projektgrupp?

\item Hur har kombinationen av teamledare och Scrum-metodik fungerat?

\item Har rollen som teamledare och Scrum-metodik gynnats av varandra?

\item Finns det något man kan ändra på i teamledarrollen eller i Scrum-metodik för att arbetet skulle kunna förbättras?

\item Har du något övrigt att tillägga?

\end{enumerate}

Eftersom de personer som intervjuades alla var medlemmar i olika projektgrupper fick man insyn i olika tillämpningar av Scrum vilket var väldigt givande.

\section{Resultat}
Detta avsnitt redovisar resultatet från de olika metoder som använts för att samla in fakta.

\subsection{Resultat - Informationssökning}
Den artikel som har hittats gav en beskrivning av utmaningen att implementera Scrum-metodik i ett företag som inte har arbetat med den metoden tidigare. 

Det man kan ta med sig därifrån är att man måste arbeta med metoden för att få in rutinen att hela gruppen ska bestämma. 
I projektet som observerades märkte man också att det inte fanns något större intresse att ta ansvar och ledarroll.
Tanken var att ledarrollen ska skiftas dit kompetensen finns, vilket innebär att man byter ledare beroende på vad det är som ska göras i projektet. 
Slutligen så märkte man också att om man höll sprintarna kortare så tappade man inte resurser och medlemmarna var mer närvarande i det projektet som observerades.

\subsection{Resultat - Egna erfarenheter}
Egna erfarenheter gav en bra förstahandsvy av hur en grupp som inte arbetat med Scrum tidigare tillämpar metoden. 
I början av projektet fanns det inte jättemycket struktur i gruppen, medlemmar tog på sig uppgifter som ansågs behövde utföras. 
När gruppen närmade sig inlämning 1 uppmärksammades det att mycket tid hade lagts på uppgifter som inte var jätteviktiga och gruppen fick därmed ändra i sin planering.
Detta förbättrades när gruppen började använda sig av Scrum. 
Det började hållas sprint planeringar, utvärderingar, tillbakablickar och Scrum-möten. Det optimala är att hålla Scrum-möte varje dag, men det var inte applicerbart i detta projekt då gruppen hade olika scheman att anpassa sig efter. 

Efter inlämning 1 så höll gruppen sin första sprint-tillbakablick där gruppen diskuterade vad som gått bra och mindre bra. 
Under denna tillbakablick var den gemensamma åsikten att Scrum hade fungerat väldigt bra och att medlemmarna nu hade mer koll på läget. 
Som teamledare och Scrummästare tyckte jag att det fungerade bra, det kändes naturligt att jag höll i Scrum-processerna. 

Under kommande sprintar var känslorna gällande Scrum liknande det vid den första tillbakablicken. 
Medlemmarna i gruppen ansåg att det var en bra metod att använda sig av. Under vissa sprintar som sammanföll med tentaperioder kände gruppen av att insynen i vad som skedde i projektet minskade.
Medlemmarna träffades inte lika ofta som vanligt och det ledde till att kommunikationen minskade när man inte höll Scrum-möten. Som teamledare arbetade jag för att få in ett par av dessa möten under tentaperioderna. 
Det viktigaste då var inte att medlemmar skulle ha presterat någonting utan mer för att hålla uppe kommunikationen och att alla skulle vara uppdaterade ifall någon hade arbetat med projektet.

Som Scrummästare och teamledare arbetade jag mycket för att hålla igång diskussioner på diverse Scrum-processer, det är trots allt gruppen som ska diskutera och bestämma vad som ska åstadkommas. 
Detta var svårast i början av projektet men när gruppen närmade sig slutet märktes det hur gruppen arbetade och diskuterade väldigt mycket. 
Detta berodde troligtvis på att gruppen blev slumpad ihop och efter att ha arbetat tillsammans i några månader så känner man varandra bättre än i början. 
Ett annat skäl till ökade diskussioner och sammarbete bör också ha berott på att gruppen hade vant sig vid att använda Scrum och hade bra koll på de processer som metodiken innefattar.

\subsection{Resultat - Intervjuer}
De intervjuer som jag höll var väldigt intressanta då de som intervjuades har haft en annan erfarenhet av Scrum-metodik än mig eftersom att de arbetat i andra projektgrupper. 

De intervjuades projektgrupper hade arbetat lite annorlunda. En grupp hade utvecklingsledaren som Scrummästare vilket kan anses vara bra då utvecklingsledaren har koll på vad som ska göras under utvecklingsfasen.

Gällande processerna i sig så tyckte flera av de intervjuade att det är svårt att upprätthålla dagliga Scrum-möten och ibland sprintplanering. Detta för att medlemmar i projektgrupperna har olika schema och det har krockat med tentaperioder eller lediga dagar.

När man kombinerat teamledarrollen med Scrum så ansåg de intervjuade att när en person var både teamledare och Scrummästare så kunde det bli lite för hög belastning på den personen. 
Den intervjuade vars grupp hade utvecklingsledaren som scrummästare tyckte det hade gått bra och att det var skönt att det var uppdelat.

Om man tittar på förbättringspunkter tyckte de intervjuade att en tydligare uppdelning och mer utbildning i Scrum hade varit nyttigt. Detta för att det skulle gå mer smidigt och att medlemmar i gruppen skulle ha koll på hur processerna i Scrum går till.

\section{Diskussion}
I det här avsnittet så diskuteras metoden och resultatet i studien.

\subsection{Metod}
I denna studie så användes tre olika sätt att införskaffa information, artiklar, egna erfarenheter och intervjuer. 
Det fanns inte mycket artiklar där utvärdering av Scrum och ledarskap har genomförts. 
Den artikeln som använts i denna studie var dock väldigt bra då det gav inblick i hur det fungerade för ett företag som inte arbetat med Scrum tidigare, vilket relaterar till vår situation i början av projektet. 
Det hade varit bra att hitta en artikel där ett projekt hade arbetat med Scrum och resultatet hade fallit väl ut så man kunde jämföra med denna artikel. 
Vad var anledningarna till att de lyckades? 
Hade projektgruppen i den behandlade artikeln kunnat tillämpa något av det för att få ett lyckat resultat?
Det hade också varit bra om en artikel om teamledarrollen hade tagits fram. 

Intervjuer och egna erfarenheter ger ganska liknande svar då teamledarrollen i respektive projekt är väldigt lik varandra.
Egna erfarenheter var helt klart den bästa källan för att förstå hur Scrum fungerar. 
Det går alltid att läsa sig till hur Scrum ska fungera, men det är svårt att se de problem som kan uppstå och hur man ska tackla dem utan att själv ha arbetat med Scrum. 
Det som hade kunnat göras bättre var att reflektera mer över frågeställningarna i denna studie under projektets gång. 

Slutligen så användes intervjuer som metod att samla information. 
Det var ett bra sätt att få en annan syn på Scrum, speciellt eftersom intervjuerna genomfördes i slutet på de intervjuades projekt. 
Två av de tre personer som intervjuades hade agerat teamledare eller Scrummästare eller båda rollerna samtidigt. 
Detta gav bra feedback då de hade färska erfarenheter och hade fått möta och tackla de utmaningar som fanns. 
En annan approach på intervjuer hade kunnat vara att intervjua varje medlem i en projektgrupp för att se om alla såg på Scrum och teamledaren på samma sätt. 
Det finns en chans att det hade kunnat ge olika svar och därmed en spännande grupp att analysera då de inte är eniga. 
Men detta frångår de frågeställningar som denna studie har så jag är nöjd med att ha intervjuat nästan enbart teamledare och Scrummästare.

\subsection{Resultat}
Något som var intressant, och ganska förståeligt, i resultaten från de olika metoderna var att alla tyckte att gruppen som arbetar med Scrum bör vara väl insatt i metodiken för att processerna ska fungera bättre.
Det som hade kunnat utredas bättre för att få ett bättre resultat var teamledarrollen. 
Scrum är ganska omfattande och går att skriva mycket om medans det är svårt att få ut information om teamledarrollen. 
Av de som intervjuats och av egna erfarenheter så följer teamledarrollen ganska väl den beskrivning som samtliga fick under en föreläsning. 
I övrigt tycker jag att resultatet från den information som har hämtats in har varit bra och jag anser mig ha svar på de frågeställningar som är listade i början av detta bidrag.

\section{Slutsatser}
Detta avsnitt är tänkt att svara på de frågeställningar som den individuella studien har. De två första frågorna:

\textbf{Hur ser teamledarrollen ut?}\\
\textbf{Hur fungerar Scrum-metodik?}

anser jag redan vara besvarade i teori-avsnittet så därför fokuserar jag detta avsnitt på den sista frågan.

\textbf{Går det att kombinera teamledarrollen i en Scrum-metodik och fortfarande bibehålla syftet med de båda?}\\
I den studie som har genomförts så har jag tittat på en undersökning som andra personer har gjort, egna erfarenheter samt intervjuer av personer. 
Artikeln visade att det behövs någon form av ledarskap för att gruppen ska kunna fungera, deras fokus var att ledarskapet ska skifta under projektet till den person som för tillfället har bäst kompetens inom området. Men jag tror att projektet skulle behöva en fast ledare för att styra upp gruppen och leda den mot rätt mål.
Mina egna erfarenheter tyder på att det är bra med en ledarroll utöver den som är Scrummästare. 
Det kändes inte helt optimalt att Scrummästar-rollen innehades av teamledaren. 
Det vore bättre om utvecklingsledaren eller någon annan lämplig roll innehade rollen som Scrummästare. 
Intervjuerna gav olika syner på hur man kan tillämpa Scrum-metodik och teamledarrollen. 
Det som togs upp i samtliga intervjuer var att teamledarrollen lätt får för mycket att göra om den innehar både teamledarrollen och rollen som Scrummästare.
Min slutsats och svar på denna fråga är alltså att teamledarrollen går att kombinera med Scrum-metodik. 
En sådan roll behövs för att sköta administrativa saker som inte har med Scrum att göra. 
I min projektgrupps situation har det t.ex. handlat om att ha kontakt med handledare och examinator, boka salar, hålla koll på vad som behöver göras och hålla koll på deadlines. 
Det som går att ta till sig är att det vore bra om någon annan än teamledarrollen tar sig an rollen som Scrummästare. 
Detta för att fördela ansvaret mer i gruppen så att man inte får en kombinerad roll som börjar gå mer åt en projektledare. 

Jag anser att denna studie har genomförts på ett bra sätt och besvarade de frågeställningar som sattes upp. Jag är nöjd med resultet som studien kom fram till.
