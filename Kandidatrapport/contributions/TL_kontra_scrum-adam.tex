\chapter{Teamledarens roll kombinerad med Scrum-metodik av Adam Andersson}

\section{Inledning}
Scrum-metodik är ett välkänt arbetssätt som ofta används inom mjukvaruutveckling. I en grupp som använder sig av scrum-metodik finns det ingen specifik ledare, om man bortser från scrummästaren som har ansvaret för att se till att scrum-metodik används, då alla är utvecklare och gruppen tillsammans bestämmer vad som ska göras och hur problem ska lösas.
I detta projektet har även alla gruppmedlemar en projektroll varav en person innehar rollen som teamledare. Denna studie ska utreda hur rollen som teamledare fungerar i kombination med scrum-metodik.

\subsection{Syfte}
Rollen som teamledare innebär att du ska coacha projektgruppen och se till att de mål som finns för projektet uppfylls. Enligt scrum-metodik är det gruppen som tillsammans ska bestämma över de beslut som tas och hur problem ska lösas. Syftet med denna studie är att utreda om det går att kombinera rollen som teamledare med scrum-metodik.

\subsection{Frågeställningar}
Utefter det syfte som denna studie har ska följande frågor försöka besvaras:

\begin{enumerate}
	\item Hur ser teamledarrollen ut?
	\item Hur fungerar scrum-metodik?
	\item Går det att kombinera teamledarrollen i en scrum-metodik och fortfarande bibehålla syftet med de båda? 
\end{enumerate}

\section{Bakgrund}
I början av projektet skulle gruppen tilldela samtliga medlemmar en roll, däri fanns rollen teamledare som författaren av denna studie tog på sig. När projektet hade fortgått i några veckor togs det upp vilken metodik gruppen skulle använda för projektet. Kunden föreslog att vi kunde använda oss av scrum, så teamledaren läste på om denna metodik och höll en redovisning för gruppen. Efter redovisningen diskuterades om detta verkade vara en bra metod. Då scrum är en välkänd agil utvecklingsmetod med enkla och strukturerade riktlinjer tog gruppen beslutet att använda sig av denna. När gruppen började använda sig av metodiken började jag fundera över hur detta skulle påverka rollen som teamledare då teamledaren är en typ av ledarroll men enligt scrum är det gruppen som tar beslut om vad som ska göras och hur problem ska lösas.

\section{Teori}
Detta avsnitt går igenom den teori som finns till grund för denna studie.

\subsection{Scrum} \label{adam_scrum}
Scrum är en arbetsmetodik som främst används för mjukvaruutveckling. Metodiken appliceras på mindre team (3-9 personer) och går ut på att teamet arbetar i tidsbestämda sprintar där det existerar ett antal uppgifter som ska genomföras under respektive sprint.
I scrum existerar ett fåtal specifika roller som används för att beskriva de inblandade i projektet. De roller som existerar är:

\begin{itemize}
	\item \textbf{Produktägare}
	
	Produktägaren är den person som representerar beställaren, vanligtvis aktieägarna i större bolag, och förmedlar till utvecklingsteamet vilka krav som finns och vilka av dessa som har högst prioritet. En viktig egenskap för en produktägare är empati, du arbetar med aktieägare som ofta har olika bakgrund och värderingar. Produktägaren måste också vara bra på att samarbeta med utvecklarna så att arbetet med produkten går bra och att krav uppfylls.
	
	\item \textbf{Scrummästare}
	
	Scrummästaren skiljer sig från den klassiska teamledaren eller projektledaren. Dess roll är att se till att det inte existerar några hinder för utvecklingsteamet så att de kan uppnå projektmål och leverera en produkt. En annan viktig uppgift för scrummästaren är att se till att de processer som man valt att applicera i scrum-metoden utförs och efterföljs. Samarbete med produktägaren finns också för att underhålla backloggen så att utvecklingsteamet förstår vad som ska göras och så att arbetet med produkten kan fortgå.
	
	\item \textbf{Utvecklingsteam}
	
	Utvecklingsteamet är själva projektgruppen och den ansvarar för att göra framsteg i arbetet och utföra de uppgifter som existerar i respektive sprint. Trots att det kan finnas olika typer av roller så kallas alla i teamet för utvecklare. För att inte skapa förvirring kallas utvecklingsteamet för leveransteam och personerna för teammedlemmar.	
\end{itemize}

Utöver de roller som existerar i scrum använder man sig också av fyra olika mötestyper som inträffar med olika frekvens. Det viktiga med dessa möten är att de är tidsbestämda och den maximala tid som ska läggas på dessa beror i flera fall på hur lång en sprint i projektet är, detta för att man inte ska lägga alltför lång tid på möten. De mötestyper som existerar är:

\begin{itemize}
	\item \textbf{Sprintplanering (maximalt 4 timmar för en tvåveckors-sprint)}
	
	Sprintplaneringen är uppdelad i två delar, den första delen går ut på att produktägaren, scrummästaren och utvecklingsteamet väljer de uppgifter från backloggen som man tror kommer hinnas med under den kommande sprinten. När detta är gjort kommer planeringens andra del, utvecklingsteamet diskuterar och kommer mer specifikt fram till vilka arbetsuppgifter som behöver göras för att uppgifterna från backloggen ska kunna utföras och uppfyllas. Denna andra iteration över uppgifterna kan leda till att man delar upp vissa uppgifter eller att vissa uppgifter läggs tillbaka i backloggen då man inte tror att dessa kommer hinnas med under kommande sprint.
	
	\item \textbf{Daglig scrum (maximalt 15 minuter)}
	
	Varje dag hålls ett dagligt scrum-möte för att uppdatera utvecklingsteamet om vilka framsteg och hinder som finns i sprinten så att teamet kan öka sina chanser att nå målen med nuvarande sprint. Dessa möten är tidsbestämda till 15 minuter och de kan utföras stående i en ring, så att man inte fokuserar på annat och så att man håller sig inom de 15 minuterna. Man kan utgå ifrån tre frågor under en daglig scrum:
	
	\begin{itemize}
		\item Vad har jag gjort sedan igår?
		\item Vad ska jag åstadkomma till imorgon?
		\item Vad hindrar mig?
	\end{itemize}
	
	Om det skulle komma fram att det finns några hinder i gruppen ska scrummästaren notera detta och man bör sätta en person på att lösa problemet så att sprinten kan fortgå utan hinder.
	
	\item \textbf{Sprintgenomgång (maximalt 2 timmar för en tvåveckors-sprint)}
	
	När en sprint är avslutad hålls en sprintgenomgång, där man tittar på vilka arbetsuppgifter utvecklingsteamet har, och inte har, hunnit med att slutföra. Man håller också en demo för beställaren där man visar vilka framsteg som har gjorts under sprinten. Efter genomgång av utförda och icke utförda arbetsuppgifter samt demo diskuterar utvecklingsteamet och beställaren vad teamet ska arbeta med härnäst.
	 
Det som är viktigt att tänka på under en sprintgenomgång är att man inte ska visa upp arbetsuppgifter som inte är färdiga, t.ex. ska man inte visa upp funktionalitet i ett program om funktionaliteten inte är helt färdig.

\item \textbf{Sprintåterblick (maximalt 90 minuter för en tvåveckors-sprint)}

När en sprintgenomgång har utförts är det dags för en sprintåterblick. Sprintåterblick innebär att man utvärderar den sprint som har varit genom att utgå från två frågor:

\begin{itemize}
	\item Vad gick bra under sprinten?
	\item Vad kan förbättras till nästa sprint?
\end{itemize}

Från dessa två frågor identifierar man förbättringar som kan göras. Man väljer ut några av dessa punkter och arbetar med dem i nästa sprint.

\end{itemize}

För att scrum ska fungera så effektivt som möjligt behöver man en grupp som är geografiskt närvarande så att man får känsla av tillhörighet och har möjlighet att närvara vid dagliga scrum-möten. En annan viktig punkt är att medlemmarna i gruppen måste vara ärliga mot varandra, om man har några hinder i sitt arbete är det viktigt att detta kommer upp på det dagliga scrum-mötet. Detta så att scrummästaren eller utvecklingsteamet kan röja undan hindret så att arbetet kan fortsätta.\cite{scrum}

\subsection{Teamledarrollen}
Teamledaren är den person som ska coacha teamet mot de mål som är uppsatta. Du är lyhörd och ser till att alla känner sig delaktiga i processen. Teamledaren är inte en projektledare och kan därför ta ett steg tillbaka i ledarrollen för att få medlemmar i gruppen att ta plats och utvecklas.\cite{teamledare} Trots att teamledaren kan ta ett steg tillbaka är personen ändå närvarande i gruppen och ser till att ha täta avstämningar för att ha koll på vad som händer och för att skapa samhöriget inom gruppen.\cite{teamguide}

\section{Metod}
För att genomföra denna studie har tre olika metoder använts. Informationssökning genom att leta efter vetenskapliga artiklar som analyserar scrum-metoden, egna erfarenheter genom att analysera hur arbetet i projektgruppen har fortgått och intervjuer av medlemmar i andra projektgrupper som också har använt sig av scrum-metoden.

\subsection{Informationssökning}
Det finns en artikel som handlar om att förstå och arbeta med delat ledarskap i agil utveckling.

I den studien har man följt ett företag som nyligen har börjat använda sig av scrum-metoden för att se hur det fungerar i praktiken att dela upp ledarskap bland alla inblandade parter. Det man kunde observera var att medlemmarna i det projekt som observerades inte kunde lägga så mycket tid på projektet som man planerat, detta för att de var involverade i andra projekt som var beroende av dem. Ett annat problem man kunde se var att det saknades kompetens eller vilja att ta ledarrollen, en roll som ska roteras i scrum till den som har mest kompetens. \cite{sharedleader}

\subsection{Egna erfarenheter}
Under projektet som genomförts hade jag rollen som både teamledare och scrummästare. Det gav både erfarenhet i att leda en grupp i ett projekt och att hålla koll på de olika processer som används inom scrum.

Då en teamledare inte ska agera projektledare och eftersom det är väsentligt i scrum-metodik att det är utvecklingsteamet som bestämmer valde jag att angripa en mer passiv ledarroll. Jag bokade lokaler till gruppen, kallade till möten, höll övergripande koll på mål och deadlines och ledde diverse möten gruppen hade som t.ex. daglig scrum, sprint planering och handledarmöten. Men när det kom till beslut och planering arbetade jag för att gruppen skulle föra diskussion och komma fram till vad som skulle göras. När personer med rätt kompetens behövdes lät jag dem kliva fram och leda, detta för att få fram bäst resultat och för att följa scrum.

\subsection{Intervjuer}
Intervjuer har utförts på ett antal studenter som läser kursen TDDD96. För att få in ett bredare spektrum av erfarenheter om hur det är att arbeta med scrum och en teamledare har intervjuerna utförts på personer från olika projektgrupper.

Intervjuerna gick ut på det sättet att jag presenterade den studie jag arbetade och sedan ställde jag ett antal frågor. Till att börja med ställde jag frågor om vilken roll de hade i projektet, sedan följde frågor om varför de valt scrum, om personen var scrum-master och hur det fungerat med att upprätthålla dessa processer. Slutligen så ställde jag frågor om teamledarrollen, hur den tillämpats i projektet och hur det har gått att kombinera den med scrum-metodik. De frågor som ställdes till till de som intervjuades var:

\begin{enumerate}

\item Vilken roll har du i projektet?

\item Hur kommer det sig att ni valde att arbeta med scrum?

\item Vilka delar av scrum har ni tillämpat?

\item Vilken roll i er projektgrupp är scrummästare?

\item Varför valde ni just den rollen som scrummästare?

\item Hur har det gått med att upprätthålla scrum-processerna?

\item Vilka uppgifter har teamledaren i er projektgrupp?

\item Hur har kombinationen av teamledare och scrum-metodik fungerat?

\item Har rollen som teamledare och scrum-metodik gynnats av varandra?

\item Finns det något man kan ändra på i teamledarrollen eller i scrum-metodik för att arbetet skulle kunna förbättras?

\item Har du något övrigt att tillägga?

\end{enumerate}

Eftersom att de personer som intervjuades alla var medlemmar i olika projektgrupper fick man insyn i olika tillämpningar av scrum vilket var väldigt givande.

\section{Resultat}

\section{Diskussion}

\section{Slutsatser}

