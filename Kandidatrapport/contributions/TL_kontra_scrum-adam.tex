\chapter{Teamledarens roll kombinerat med scrum-metodik av Adam Andersson}

\section{A.1 Inledning}
Scrum modellen är en arbetssätt som används ofta inom mjukvaruutveckling och är en bevisat fungerande metod. I en grupp som använder sig utav Scrum-metoden så finns det ingen specifik ledare, om man bortser från scrum master som har ansvaret för att se till att Scrum-metodiken används, då alla är utvecklare och gruppen bestämmer tillsammans vad som ska göras och hur problem ska lösas.
I detta projektet har även alla gruppmedlemar en projektroll varav en person innehar rollen som teamledare. Denna studie ska utreda hur rollen som teamledare fungerar i kombination med scrum-metodik.

\section{A.1.1 Syfte}
Rollen som teamledare innebär att du ska coacha projektgruppen och se till att de mål som finns för projektet uppfylls. Enligt scrum-metodik så är det gruppen som tillsammans ska bestämma över de beslut som tas och hur problem ska lösas. Syftet med denna studie är att utreda om det går att kombinera rollen som teamledare med scrum-metodik.

\section{A.1.2 Frågeställning}
Utefter det syfte som denna studie har ska följande frågor försöka besvaras:

\begin{enumerate}
	\item Hur ser teamledarrollen ut?
	\item Hur fungerar scrum-metodik
	\item Går det att kombinera teamledarrollen i en scrum-metodik och fortfarande bibehålla syftet med de båda? 
\end{enumerate}

\section{A.2 Bakgrund}
I början av projektet skulle gruppen tilldela samtliga medlemmar en roll, däri fanns rollen teamledare som författaren av denna studie tog på sig. När projektet hade fortgått i några veckor så togs det upp vilken metodik gruppen skulle använda för projektet. Kunden föreslog att vi kunde använda oss av scrum, så teamledaren läste på om denna metodik och höll en redovisning för gruppen. Efter redovisningen diskuterades om detta verkade vara en bra metod. Då scrum är en välkänd agil utvecklingsmetod med enkla och strukturerade riktlinjer så tog gruppen beslutet att använda sig av denna. När gruppen började använda sig av metodiken så började teamledaren fundera över hur detta skulle påverka den roll som han redan innehade då teamledaren är en typ av ledarroll men enligt scrum så är det gruppen som tar beslut om vad som ska göras och hur problem ska lösas.

\section{A.3 Teori}
Detta avsnitt går igenom den teori som finns till grund för denna studie.

\subsection{A.3.1 Scum}
Scrum är en agil utvecklingsmetod som består av tre organ, produktägaren, scrummästaren och utvecklingsteamet. Produktägaren är den person som kommer med en beställning till utvecklingsteamet. Scrummästaren arbetar för att scrum-metodiken ska upprätthållas och ser till, tillsammans med produktägaren, att underhålla backloggen så att utvecklingsteamet förstår vad som ska göras och så att arbetet kan fortgå. Utvecklingsteamet är själva kärnan i scrum-metodiken, det är teamet som bestämmer vilka uppgifter som ska utföras under en sprint och hur de ska lösas. För en utförligare beskrivning av scrum-metodiken, se avsnitt "Scrum" i kapitel 3 - Teori i kandidatrapporten \textbf{(Ska fixa referering till scrum-avsnittet)}

\section{A.4 Metod}

\section{A.5 Resultat}

\section{A.6 Diskussion}

\section{A.7 Slutsatser}

