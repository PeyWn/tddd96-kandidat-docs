\chapter{Teamledarens roll kombinerat med scrum-metodik av Adam Andersson}

\section{A.1 Inledning}
Scrum modellen är en arbetssätt som används ofta inom mjukvaruutveckling och är en bevisat fungerande metod. I en grupp som använder sig utav Scrum-metoden så finns det ingen specifik ledare, om man bortser från scrum master som har ansvaret för att se till att Scrum-metodiken används, då alla är utvecklare och gruppen bestämmer tillsammans vad som ska göras och hur problem ska lösas.
I detta projektet har även alla gruppmedlemar en projektroll varav en person innehar rollen som teamledare. Denna studie ska utreda hur rollen som teamledare fungerar i kombination med scrum-metodik.

\section{A.1.1 Syfte}
Rollen som teamledare innebär att du ska coacha projektgruppen och se till att de mål som finns för projektet uppfylls. Enligt scrum-metodik så är det gruppen som tillsammans ska bestämma över de beslut som tas och hur problem ska lösas. Syftet med denna studie är att utreda om det går att kombinera rollen som teamledare med scrum-metodik.

\section{A.1.2 Frågeställning}
Utefter det syfte som denna studie har ska följande frågor försöka besvaras:

\begin{enumerate}
	\item Hur ser teamledarrollen ut?
	\item Hur fungerar scrum-metodik?
	\item Går det att kombinera teamledarrollen i en scrum-metodik och fortfarande bibehålla syftet med de båda? 
\end{enumerate}

\section{A.2 Bakgrund}
I början av projektet skulle gruppen tilldela samtliga medlemmar en roll, däri fanns rollen teamledare som författaren av denna studie tog på sig. När projektet hade fortgått i några veckor så togs det upp vilken metodik gruppen skulle använda för projektet. Kunden föreslog att vi kunde använda oss av scrum, så teamledaren läste på om denna metodik och höll en redovisning för gruppen. Efter redovisningen diskuterades om detta verkade vara en bra metod. Då scrum är en välkänd agil utvecklingsmetod med enkla och strukturerade riktlinjer så tog gruppen beslutet att använda sig av denna. När gruppen började använda sig av metodiken så började jag fundera över hur detta skulle påverka rollen som teamledare då teamledaren är en typ av ledarroll men enligt scrum så är det gruppen som tar beslut om vad som ska göras och hur problem ska lösas.

\section{A.3 Teori}
Detta avsnitt går igenom den teori som finns till grund för denna studie.

\subsection{A.3.1 Scrum}
Scrum är en agil utvecklingsmetod som består av tre organ, produktägaren, scrummästaren och utvecklingsteamet. Produktägaren är den person som kommer med en beställning till utvecklingsteamet. Scrummästaren arbetar för att scrum-metodiken ska upprätthållas och ser till, tillsammans med produktägaren, att underhålla backloggen så att utvecklingsteamet förstår vad som ska göras och så att arbetet kan fortgå. Utvecklingsteamet är själva kärnan i scrum-metodiken, det är teamet som bestämmer vilka uppgifter som ska utföras under en sprint och hur de ska lösas. För en utförligare beskrivning av scrum-metodiken, se avsnitt \ref{scrum}.

\subsection{A.3.2 Teamledarrollen}
Teamledaren är den person som ska coacha teamet mot de mål som är uppsatta. Du är lyhörd och ser till att alla känner sig delaktiga i processen. Teamledaren är inte en projektledare och kan därför ta ett steg tillbaka i ledarrollen för att få medlemmar i gruppen att ta plats och utvecklas.\cite{teamledare} Trots att teamledaren kan ta ett steg tillbaka så är personen ändå närvarande i gruppen och ser till att ha täta avstämningar för att ha koll på vad som händer och för att skapa samhöriget inom gruppen.\cite{teamguide}

\section{A.4 Metod}
För att genomföra denna studie har tre olika metoder använts. Informationssökning genom att leta efter vetenskapliga artiklar som analyserar scrum-metoden, egna erfarenheter genom att analysera hur arbetet i projektgruppen har fortgått och intervjuer av medlemmar i andra projektgrupper som också har använt sig av scrum-metoden.

\subsection{A.4.1 Informationssökning}
Det finns en artikel som handlar om att förstå och arbeta med delat ledarskap i agil utveckling. \citep{sharedleader}

I den studien har man följt ett företag som nyligen har börjat använda sig av scrum-metoden för att se hur det fungerar i praktiken att dela upp ledarskap bland alla inblandade parter. Det man kunde observera var att medlemmarna i det projekt som observerades inte kunde lägga så mycket tid på projektet som man planerat, detta för att de var involverade i andra projekt som var beroende av dem. Ett annat problem man kunde se var att det saknades kompetens eller vilja att ta ledarrollen, en roll som ska roteras i scrum till den som har mest kompetens. \citep{sharedleader}

\subsection{A.4.2 Egna erfarenheter}
Under projektet som genomförts så hade jag rollen som både teamledare och scrummästare. Det gav både erfarenhet i att leda en grupp i ett projekt och att hålla koll på de olika processer som används inom scrum.

\subsection{A.4.3 Intervjuer}
Intervjuer har utförts på ett antal studenter som läser kursen TDDD96. För att få in ett bredare spektrum av erfarenheter om hur det är att arbeta med scrum och en teamledare har intervjuerna utförts på personer från olika projektgrupper.

Intervjuerna gick ut på det sättet att jag presenterade den studie jag arbetade och sedan ställde jag ett antal frågor. Till att börja med ställde jag frågor om vilken roll de hade i projektet, sedan följde frågor om varför de valt scrum, om personen var scrum-master och hur det fungerat med att upprätthålla dessa processer. Slutligen så ställde jag frågor om teamledarrollen, hur den tillämpats i projektet och hur det har gått att kombinera den med scrum-metodik.

\section{A.5 Resultat}

\section{A.6 Diskussion}

\section{A.7 Slutsatser}

