\chapter{Teamledarens roll kombinerat med scrum-metodik - Adam}

Scrum modellen är en arbetssätt som används ofta inom mjukvaruutveckling och är en bevisat fungerande metod. I en grupp som använder sig utav Scrum-metoden så finns det ingen specifik ledare, om man bortser från scrum master som har ansvaret för att se till att Scrum-metodiken används, då alla är utvecklare och gruppen bestämmer tillsammans vad som ska göras och hur problem ska lösas. I TDDD96 så har man en teamledarroll och då är tanken att mitt individuella bidrag till kandidaten ska vara en beskrivning av teamledarrollen, scrum-metodiken och hur det fungerar i kombination med vartannat, blir det konflikt i det arbetsätt och de uppgifter som ska utföras eller fungerar det att kombinera dessa?

Jag kommer främst att arbeta på två sätt för att införskaffa material till denna studie, det ena är att använda sig av LiU’s biblioteks hemsida för att finna vetenskapliga artiklar om scrum och teamledarrollen, det andra är att använda sig av diverse webbsidor då det finns mycket information gällande ledarskap och scrum-metodik. I det andra fallet av informationssökning så är det viktigt att jag är extra kritisk mot de källor jag hittar eftersom det är större chans att informationen inte är helt korrekt jämfört med vetenskapliga artiklar.
