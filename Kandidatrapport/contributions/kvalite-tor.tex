\chapter{Kvalitetsförsäkrande metoder i ett småskaligt mjukvaruprojekt}


Inviduellt bidrag Kvalitetssamordnare. Tor Utterborn
\section{Inledning}
I dagens industri sätter företagare mer och mer fokus på hållbarhetstänk. Hållbarhetstänk i dels metoder för utvecklandet av produkten, men också hållbarhetstänk hos produkten i ekonomiska,  miljö och samhällsaspekter.
Om man använder metoder specificerade i kvalitetsplanen för att kvalitetsförsäkra produkten. Kommer vi då uppnå en högkvalitativ produkt i dessa tre ämnesområden?
\section{Syfte}
Det finns många kvalitetsförsäkrande arbetsmoment inom småskalig mjukvaruutveckling. Kan några av dessa metoder föra vårt arbete mot en mer kvalitativ produkt? Och ur ett större perspektiv, vad kan man dra för slutsatser kring de arbetsmetoder, standard etc. som undersökts. 
Jag tycker det kan vara intressant att undersöka detta i ett projekt som ska utveckla ett verktyg som i teorin ska boka in och användas för att boka in tusentals operationer om året.
Jag tänkte samla information från Linköpings publicerade artiklar i deras bibliotek, möjligtvis artiklarna vi blev tilldelade i hållbarhetsseminariet. Sedan får jag i min roll som kvalitetssamordnare samla och diskutera erfarenheter med projektmedlemmar under projektets gång.
