\chapter{För/nackdelar med TypeScript jämfört med JavaScript}


TypeScript är ett språk utvecklat av Microsoft som i princip är en påbyggnad av JavaScript som lägger till bland annat strikt typning och klasser. TypeScript kompilerar också till JavaScript.

I webbapplikationen som kommer att utvecklas i projektet kommer det finnas både en klientdel och en serverdel. Klientdelen kommer att bygga på ramverket Angular, som använder TypeScript, och serverdelen på Node.js där JavaScript används. Alltså kommer alla medlemmar att få testa på båda språken. Jag tycker därför att det skulle vara intressant att undersöka för och nackdelar med den strikta typningen och klasserna i TypeScript jämfört med JavaScript, där typningen är dynamisk och klasser saknas. Speciellt av intresse är hurvida strikt typning hjälper till att minska antalet fel som görs under projektet.

För att hitta material om detta har jag främst tänkt söka på Liu:s biblioteks hemsida men även på internet. Jag ska också samla in projektmedlemmarnas erfarenheter med de två språken under utvecklingens gång.
