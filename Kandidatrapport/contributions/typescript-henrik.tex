\chapter{Jämförelse mellan TypeScript och JavaScript i projektet - Henrik Lindström}

\section{Inledning}
Vid utveckling av webbapplikationer är språkvalen något begränsade. I grunden är det dynamiskt typade JavaScript ofta det enda alternativet eftersom det är det enda scriptingspråk med ett brett stöd av webbläsare. Vill man ha statisk typning så finns det dock alternativ som uppnår samma webbläsarstöd genom att kompileras till JavaScript. Ett av dessa är TypeScript som är en typad utökning av JavaScript. Under projektet användes delvis TypeScript vid utveckling av webbklienten och delvis JavaScript vid utveckling av webbservern.

I denna rapport utreds hur de två olika språken och dess olika typsystem påverkade utvecklingsarbetet i projektet Schemaläggningsstöd för kirurgi där en webbapplikation utvecklades.
\section{Syfte}
Denna rapport undersöker positiva och negativa effekter med användning av TypeScript jämfört med JavaScript i projektet. Mer specifikt är de två språkens olika typsystem som ligger i fokus. Syftet med detta är för att komma fram till vilka effekter statisk jämfört med dynamisk typning hade på utvecklingsarbetet. Detta innefattar effekter på feldetektering, kodförståelse och utvecklingsfart. 

\section{Frågeställning}
För att uppfylla syftet kommer följande frågeställningar att besvaras i rapporten:
\begin{enumerate}
\item Bidrog den statiska typningen i TypeScript till att fel hittades tidigare under utvecklingen?
\item Bidrog den statiska typningen i TypeScript till kod som var lättare att läsa och förstå?
\item Hur påverkades produktiviteten vid utveckling av statisk jämfört med dynamisk typning?
\end{enumerate}
\section{Bakgrund}
I projektet utvecklades en webbapplikation bestående av två delar, en webbklient och en webbserver. Eftersom det bestämdes att klienten skulle skrivas med hjälp av plattformen Angular medförde det också språket TypeScript som i plattformen. På serversidan bestämdes det istället att utvecklingen skulle ske med vanlig JavaScript. Detta innebar alltså att båda språken användes i projektet och att gruppmedlemmarna fick testa på och få erfarenheter av dem båda. Alla medlemmar i projektet hade också svaga eller inga kunskaper om de båda språken innan projektet. Detta innebar att projektet lämpade sig rätt så bra för denna undersökning.

Jämförelse mellan strikt och dynamiskt typade språk är relevant eftersom det finns en massa språk på båda sidor som flitigt används inom mjukvaruutveckling. Detta tyder på att båda typsystemen är mycket relevanta och det är därför intressant att jämföra dem, i det här fallet inom webbutveckling.
\section{Teori}
\subsection{Dynamisk typning}
\subsection{Statisk typning}
\subsection{JavaScript}
\subsection{TypeScript}
\section{Metod}
\section{Resultat}
\section{Diskussion}
\section{Slutsatser}


TypeScript är ett språk utvecklat av Microsoft som i princip är en påbyggnad av JavaScript som lägger till bland annat strikt typning och klasser. TypeScript kompilerar också till JavaScript.

I webbapplikationen som kommer att utvecklas i projektet kommer det finnas både en klientdel och en serverdel. Klientdelen kommer att bygga på ramverket Angular, som använder TypeScript, och serverdelen på Node.js där JavaScript används. Alltså kommer alla medlemmar att få testa på båda språken. Jag tycker därför att det skulle vara intressant att undersöka för och nackdelar med den strikta typningen och klasserna i TypeScript jämfört med JavaScript, där typningen är dynamisk och klasser saknas. Speciellt av intresse är hurvida strikt typning hjälper till att minska antalet fel som görs under projektet.

För att hitta material om detta har jag främst tänkt söka på Liu:s biblioteks hemsida men även på internet. Jag ska också samla in projektmedlemmarnas erfarenheter med de två språken under utvecklingens gång.
