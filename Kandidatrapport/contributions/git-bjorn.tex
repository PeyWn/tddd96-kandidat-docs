

\chapter{Versionshantering för ett mindre mjukvaruutvecklingsprojekt.}



\vspace{1.5em}
\section{Inledning}
Inför ett mjukvaruprojekt så är det kritiskt att välja rätt versionshanteringsprogram. Programmet måste på ett bra sätt skala till projektets storlek, den funktionalitet som planeras användas och inlärningskurva för programmet måste vara lämpligt. Den här rapporten tillhandahåller information erhållen genom analys av intervjuer angående valet och effektiviteten av olika versionshanteringsprogram för mindre mjukvaruutvecklingsprojekt.

\vspace{1.5em}
\section{Syfte}
Syftet med den här rapporten är att undersöka hur versionshantering kan användas på ett effektivt sätt under ett mindre mjukvaruutvecklingsprojekt. Rapporten redogör för vilka olika verktyg som kan användas i den här typen av projekt. Vidare beskrivs även arbetsmetodiker för verktygen.

\vspace{1.5em}
\section{Frågeställning}
\begin{itemize} \vspace{1em}
    \item Vilka olika verktyg och arbetsmetodiker finns att tillgå för versionshantering?

    \item Går det att säkerställa effektivt för en mindre agil projektgrupp?
    \item Går det att säkerställa att en projektgrupp arbetar effektivt?

    %\item Jämförelser mellan olika versionshanteringsarbetssätt. Går att jämföra, verktyg (t.ex. git (github, gitlab), docker, subversion, dropbox etc) och metod (t.ex. Git workflow, vi använder feature barnch).
    %\item Kommunikation gällande arbetssättet, hur delar medlemmarna upp arbetet. (t.ex. Individuella branches), hur används commitmeddelanden.
    %\item Hur ser man till att koden fungerar och ser bra ut (pull requests.)
\end{itemize}


\vspace{1.5em}
\section{Bakgrund}
Inför kandidatprojektet så beslutade gruppen att använda versionshanteringsprogrammet Git. Att använda Git möjliggör för det första att gruppen på en enkelt och familjärt sätt kan monitorerna ändringar i koden samt dokument. För det andra så har Git också stöd för utveckling av olika versioner, av samma kod, parallellt. Det gör att gruppens medlemmar kan arbeta på olika uppgifter samtidigt utan att vara till besvär för varandra.

\vspace{1.5em}
\section{Teori}
  \subsection{Versionshantering}
Versionshantering är i grunden ett sätt att hantera förändringar och säkerhetskopiering av data. Ingen skulle idag lämna data som är viktig utan att säkerhetskopiera den. Vissa typer av data tillexempel programkod, har en tendens att ändras mycket under sin livstid. Det blir då väldigt viktigt att kunna hantera alla ändringar. Då versionshantering medför flera fördela så har det idag blivit en naturlig del av mjukvaruutvecklares vardag. \cite{VersionControlGit}

\vspace{1.5em}
\section{Metod}
För att undersöka vilka olika verktyg och arbetsmetodiker som är attraktiva i mindre mjukvaruutvecklingsprojekt så används intervjuer. Till intervjuerna kommer studenter som går eller har gått kandidaten kursen eller liknande att intervjuas i sökas första hand. Även ingenjörer på mindre företag som arbetar i mindre projekt kommer att sökas.


Intervjuerna kommer att fokusera på att samla in information angående vilket versionshanteringsprogram som gruppen använder och varför. Vidare så är det också av intresse att samla in information angående vilken arbetsmetodik som guppen använder när det gäller versionshantering. Det är också intressant varför just den metodiken valdes och om den passar gruppens övriga arbetsmetodiker.

%Finns mycket teori angående workflow, förhoppningsvis även en del studier.
\subsection{ Potentiella Intervjufrågor } \vspace{1em}
\begin{enumerate}

  \item Kan du kort beskriva ett mindre mjukvaruutvecklingsprojekt som du har medverkat i? Vad var syftet med projektet?

  \item Vilket versionshanteringsprogram användes under projektet?

  \item Hade ni möjligheten att välja?

  \item Följdfråga om föregående fråga sant: Hur kommer det sig att ni valde just det ni gjorde?
  \item Följdfråga om föregående fråga falskt: Hur påverkade det projekt?

  \item I projekt hade ni en arbetsmetodik för versionshantering, om så var fallet hur fungerade den?

  \item Använde sig gruppen av metodiken så som det var tänkte i teorin?

  \item Passade metodiken storleken på gruppen?

  \item Använde gruppen något speciellt ramverk eller arbetssätt för att hantera arbetsprocessen och planera arbetet, t.ex. Veckomöten, Scrum?

  \item Följdfråga: Vävde ni samman arbetssättet med versionshanteringen och arbetsmetodiken på något sätt?

  \item Skulle du säga att arbetsmetodiken för versionshanteringen som ni använde var effektiv? Varför?

  \item Hur tycker du att programmet i sig fungerade i projektet?

  \item Var det svårt att lära sig versionshanteringsprogrammet? Fanns det programvara som gjorde versionshanteringen blev lättare att hantera?

\end{enumerate}

\vspace{1.5em}
\section{Resultat}
Vid undersökningen så intervjuades sex olika studenter som hade tidigare erfarenhet av att arbeta i mindre mjukvaruutvecklingsprojekt där för projektgruppen för respektive projekt använde sig av versionshantering. Det klara majoriteten av de som blev intervjuade använde sig utav versionshanteringsprogrammet git. Vilken webbtjänst som de använde sig av för att lagra och hantera den data som versionshanterats skiljde sig dock åt. 

\vspace{1.5em}
\section{Disskution}


\vspace{1.5em}
\section{Slutsatser}
