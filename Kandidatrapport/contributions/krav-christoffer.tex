\chapter{Betydelsen av att samla krav från en varierad grupp aktörer - Christoffer}

Jag avser skriva om hur kravframtagningsprocessen påverkas positivt av att samla och värdera krav från olika aktörer. Detta är något vi har möjlihet att göra i vårt projekt då vi har bra kontakt med systemarkitekter, projektledare för det större projekt som vårt är en slags förstudie till, UI och användbarhetsspecialist och givetvis även ett urval av de användarna som i slutet av utvecklingen ska använda produkten.

Genom projektets gång utgår jag från att jag kommer att ha god insyn i hur varierade åsikter och synvinklar bidrar till att skapa en bättre totalbild av vad som behövs i mjukvaran som vi skapar. Det lär också dyka upp scenarion där motstridiga uppgifter skapar potentiella problem och jag tror att det kan leda till en intressant diskussion om fördelar och nackdelar med metoden.

Informationen till min individuella del i rapporten tänker jag i första hand ta ifrån två källor, dels de upplevelser, de dokument och möten som jag tar del av under projektet och dels från vetenskapliga artiklar som berör relevant information för ämnet.
