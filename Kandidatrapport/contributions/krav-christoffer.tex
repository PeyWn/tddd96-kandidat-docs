\chapter{Hur kan man på ett effektivt sätt välja rätt aktörer för insamling av krav - Christoffer}

\section{Inledning}
I början av ett mjukvaruprojekt är det avgörande för projektets framgång att framställa en lista med krav som projektet ska uppnå. Detta är ett komplext ämne med mycket olika delar och metoder. Denna text diskuterar valet av stakeholders och vilken prioritet dessa ges i processen.

\section{Syfte}
Syftet är att diskutera och jämföra den metod som vi använt i vårt projekt för framställning av krav med en dokumenterad metod. Anledningen till detta är att kunna dra slutsattser om hur en mer sturkturerad approach för att välja stakeholders skulle påverkat vårt projekt och främst då kravframtagningsprocessen och dess resultat.

\section{Frågeställning}
Hur väljer och prioriterar man bäst stakeholders i ett mindre mjukvaruprojekt? (tillfällig)

\section{Avgränsningar}
Denna jämförelse kommer enbart att beröra den metod som använts i projektet för kravframställning och val av stakeholders med den metod beskriven i  \todo{referens till källa här}

\section{Definitioner}
\begin{itemize}
	\item \textbf{Stakeholder} - Aktör med intresse av projektet.
	\item \textbf{LoFi-prototyp} - Snabb designprotoyp oftast tillverkad med penna och papper.
\end{itemize}


\section{Bakgrund}
Som analysansvarig i detta projekt har det varit mitt ansvar att kommunicera med båder den egna projektgruppen och kunden för att framställa en kravspecifikation. Då detta är ett mycket komplext område valde jag att lägga mest fokus på metoden för att ta fram olika krav mer än av vilka aktörer kraven skulle tas ifrån. Senare i projektet fick vi möjlighet att intervjua tre potentiella slutanvändare. Denna intervju vände i stor utsträckning upp och ner på vår kravlista Vilket fick mig att undersöka olika metoder för att välja och prioritera krav från olika stakeholders i ett projekt. Resultatet av denna undersökning presenteras i denna rapport.  

\section{Teori}
I denna del pressenteras en del teori som behövs för att senare kunna förstå och jämföra med den metod vi använt i projektet med andra metoder och tekniker.
\subsection{Kravframställning}
Vid starten av ett mjukvaruprojekt börjar man analysera behoven för projektet för att formalisera en lista över krav som slutprodukten ska uppfylla. För att göra detta behövs det oftast att berörda aktörer så kallade stakeholders berättar vad de vill att produkten ska ha för funktioner och kvaliteter. En stakeholder kan vara vilken person som helst som blir påverkad av eller kan påverka utgången av projektet. Detta innebär att stakeholders kan vara  utvecklare, beställare, organisationsledare, olika experter och inte minst slutanvändare. 

Olika metoder såsom brainstorming, intervjuer och användartester kan användas för att locka fram kraven från de olika aktörerna.

Olika stakeholders kan ha mycket olika krav på produkten. Detta skapar två olika problem, dels kan olika stakeholders ha luckor i kunskap om vilka krav som finns, och dels så kan olika aktörer ha olika prioriteringar eller åsikter om vilka krav som finns. 

Lösningarna till detta är delvis motstridiga, för att inte missa några krav måste alla relevanta aktörer behandlas. Men ju fler aktörer som blir inblandade ju fler motstridiga krav riskerar att uppstå. Detta leder till att kraven måste analyseras och prioriteras beroende på vem som givit dem.
Till exempel kan det vara så att en utvecklare av programmet vill att akuta händelser markeras med lila för att det passar bäst in i programmets övriga färgschema. En av slutanvändrana tycker dock att det ska vara rött för det syns bättre. I detta fallet är det givetvis slutanvändarens krav som ska väga tyngre då det är denna person som kommer använda programmet. 
Om det gäller ett krav på vilket programmeringsspråk som ska användas däremot så lär utvecklarens förmodligen större kunskap inom ämnet göra att dennes åsikt väger tyngre.

\subsection{Urval av berörda aktörer}
För att effektivt framställa krav är det en bra idé att göra ett urval av aktörer som på bästa sätt kan skapa en komplett bild av projektets krav. Detta görs inte bara för att identifiera alla aktörer som är viktiga att ha med för att inte missa viktiga krav utan även för att se till att inte ta med aktörer vars åsikter och kunskap inte är nödvändig för att skapa en komplett bild av projektet.


\subsection{Brainstorming}
Brainstorming är en metod för kravframtagning som görs i grupp. Syftet med brainstorming är att komma fram till en stor mängd ideer. Detta görs oftast tidigt i projektet eller när en helt ny del eller funktion ska utvecklas. I vårt projekt så har vi delat upp våra brainstorming tillfällen i två delar. Del ett är till för att storma ideer. Här är det öppet för alla ideer och inga ideer får skjutas ner eller kritiseras. Alla ideer skrivs upp i en lista oftast på en whiteboard. I del 2 så diskuteras de olika ideerna och filtreras ner till en eller några få ideer som sedan utvärderas vidare på olika sätt. Till exempel kan de gå vidare och bli prototyper eller efterforskas vidare genom att läsa på om hur andra projekt har använt liknande ideer. 


\subsection{Intervjuer}
Intervjuer kan läggas upp på flera olika sätt men ett av de vanligaste sätten är att förbereda ett antal frågor som behöver besvaras för att kunna begränsa området för projektet. Dessa frågor ställs sedan till de aktörer som har kunskap om det området. Frågorna kan antingen vara smala och direkta eller mer öppna. De smalare frågorna är bra för att få svar på specifika oklarheter. Till exempel ‘Behöver användare vara inloggade för att se schemat i programmet’. Den frågan kan bara besvaras med ‘ja’ eller ‘nej’ och skapar samtidigt ett tydligt krav som kan vidareutvecklas med följdfrågor. Mer öppna frågor är mer användbara när den som ställer frågorna behöver ett mer berättande svar.

\subsection{Användartest}
En metod som vi har använt en del är användartest utförda på Lo-Fi prototyper. Vad vi då typiskt gjort är att vi har presenterat ett antal olika prototyper för olika aktörer där de har kunnat navigera runt och testa olika designkoncept. Oftast har vi låtit användaren testa mer än en prototyp där det har varit ett antal medvetna design skillnader för att efter testet kunna få reda på vilket alternativ som användare tyckte var bäst. Detta är utöver att hitta nya krav också ett bra sätt att validera de krav som redan samlats in.



\section{Metod}
\subsection{Vår framtagning}
Vår process för framtagning illustreras i en sorts tidslinje i figuren.

\subsubsection{Inledande möte}
På det första mötet var vår teamledare samt analysanvarige med och kundens beställare vilka är två personer som är anställda på avdelningen för test och innovation på Region Östergötlan. Under detta möte avhandlas projektets övergripande syfte vad i breda drag.
Vi får även reda på vilka olika kontaktpersoner vi kommer ha för olika frågor. De två deltagarna från kunden på detta mötet hade främst koll på tekniska detaljer och endast de övergripande detaljerna när det kom till mer funktionella krav.

\subsubsection{Första mötet med hela projektet}
På detta mötet var hela vår projektgrupp med samt, utöver de två beställarna från förra mötet, också en projektansvarig för ett större projekt i vilket vårt ingår som en mindre del. Denna projektansvarige hade större kunskap i de funktionella kraven för projektet 

Vi använde intervjutekniken beskriven längre med breda öppna frågor för att få en inledande idé om vad systemet skulle kunna. De som deltog på dessa möten dels en projektledare för ett större projekt där vårt projekt är en liten del samt två från kundens it-avdelning. Personerna från IT är dels de officiella beställarna men var även där för att kunna svara på frågor om tekniska och systemrelaterade frågor. Projektledaren som även jobbar inom kirurgi på sjukhuset var där för att förklara systemets krav på funktionalitet. Efter dessa mötet så skrev projektledaren för det större projektet upp en lista med krav utifrån deras behov och det vi pratat om på mötena. Denna lista diskuterades sedan igenom i utvecklingsgruppen och efter detta så sammanställde analysansvarig feedback och ideer som utvecklingsgruppen haft. 

\section{Jämförd metod}
\emph{här kommer en mer sturkturerat metod att beskrivas}

\section{Resultat}
\emph{Här kommer de olika metodernas resultat besrivas, främst med fokus på vår metod}


\section{Diskussion}
\emph{Här ska för och nackdelar med de olika metoderna diskuteas}

\section{Slutsatser}





